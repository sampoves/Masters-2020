% Sampo notes
% Tämä on mun gradu. Tässä alussa vähän löpinää siitä, mistä kyse ja varmaan jotain linkkejä

%%% LAITOKSEN VANHAT GRADUOHJEET
% https://blogs.helsinki.fi/geotiede/files/2008/03/Pro-gradu-ohjeet.pdf
% FLAMMAN GRADUOHJEET:
%%% https://flamma.helsinki.fi/portal/units/matlu?_nfpb=true&_pageLabel=P9806621561360920311893&contentId=HY358060&placeId=HY357991&lang=fi

\documentclass[a4paper,11pt]{article}

% marginaalit
\usepackage{geometry}
\geometry{
    a4paper,
    total={170mm,257mm},
    left=20mm,
    right=20mm,
    }

% fontti- ja asetteluasetukset
\renewcommand{\baselinestretch}{1.5}        % aseta riviväli
\usepackage[english]{babel}                 % aseta kieli
\usepackage[utf8]{inputenc}
\usepackage[font=small,labelfont=bf]{caption}
\setlength{\abovecaptionskip}{15pt plus 3pt minus 2pt}
\setlength{\belowcaptionskip}{15pt plus 3pt minus 2pt}
\usepackage[document]{ragged2e}             % tasaa rivit
\usepackage[ddmmyyyy]{datetime}             % muokkaa datetimen ominaisuuksia
\renewcommand{\dateseparator}{.}
\newdateformat{monthyeardate}{\monthname[\THEMONTH], \THEYEAR} %Abstraktin mm/yyyy date

% tiedostojen sijainnit
\usepackage{graphicx}
\graphicspath{ {Images/} }

% referenssit ja referenssityylit
\usepackage{natbib}
\bibliographystyle{agsm}
\setcitestyle{authoryear,open={},close={}}  % Removes citation parentheses

% muut lisäosat
\usepackage{alphabeta}
\usepackage{amsmath} 
\usepackage{subfigure}
\usepackage{gensymb}
\usepackage[colorinlistoftodos]{todonotes}
\usepackage{hyperref} 
\usepackage{amsthm} 
\usepackage{float}
\usepackage{pdfpages}
\usepackage{minted}
\usepackage{enumitem}
\usepackage{listings}
\usepackage{color}

% tables
\usepackage[showframe]{geometry}
\usepackage{tabularx}
\usepackage{makecell}                       % enable thicker vline
\usepackage{cellspace}
\setlength{\cellspacetoplimit}{6pt}
\setlength{\cellspacebottomlimit}{6pt}
\addparagraphcolumntypes{X}
\newcolumntype{?}{!{\vrule width 1.5pt}}    % thicker vline for tabularx
\newcommand\Tstrut{\rule{0pt}{2.2ex}}       % "top" strut
\newcommand\Bstrut{\rule[-4.0ex]{0pt}{0pt}} % "bottom" strut
\newcommand{\TBstrut}{\Tstrut\Bstrut}       % top & bottom struts

% titlen asetukset
\usepackage{setspace}                       % Eriävä riviväli kuin dokumentin oletus
\author{Sampo Vesanen}
\date{\today}
\title{Parking of private cars and spatial accessibility in Helsinki Capital Region}

% Begin document proper
% MUISTA KATTOO \begin{spacing}{1} \end{spacing} GRADUKIRJALLISUUS-TIEDOSTOSTA
\begin{document}

% TITLE PAGE
\begin{titlepage}
{
\setstretch{1.0}
\centering
\begin{figure}[t]
\centering
\includegraphics[scale=0.4]{HY_logo}
\end{figure}

Master's thesis \par
Geography \par
Geoinformatics \par

\bigskip
PARKING OF PRIVATE CARS AND SPATIAL ACCESSIBILITY IN HELSINKI CAPITAL REGION

\bigskip
Sampo Vesanen

\today

% Fills empty space between top and bottom
\vfill 

Supervisor: \par
Tuuli Toivonen \par
\bigskip
\bigskip
UNIVERSITY OF HELSINKI\par
FACULTY OF SCIENCE\par
DEPARTMENT OF GEOSCIENCES\par
GEOGRAPHY\par
PL 64 (Gustaf Hällströmin katu 2)\par
00014 Helsingin yliopisto\par
}
\end{titlepage}

% abstraktisivut, ensin ENG, sitten FI
% Katso apua
% https://tex.stackexchange.com/questions/186046/table-with-different-widths-and-rows
\newpage
\thispagestyle{empty}
\noindent
\begin{spacing}{1}
\footnotesize % set table font size to small
% Tabularx table notes:
%%% Circumvent having to use halves of columns by using six
%%% In this table '?' means thicker vline. See definitions for tables in the beginning of the document
%%% Idea from: https://tex.stackexchange.com/questions/236155/tabularx-and-multicolumn
%%% TBstrut is also defined above. See following address for the full explanation:
%%%%% https://tex.stackexchange.com/questions/126539/padding-at-the-top-of-a-table-cell-in-latex
\begin{tabularx}{\linewidth}{|S{X}S{X}|S{X}|S{X}|S{X}|S{X}|}
    \Xhline{3\arrayrulewidth}
    \multicolumn{3}{?>{\hsize=\dimexpr3\hsize-+3\tabcolsep\relax}S{X}|}{\textsf{Tiedekunta/Osasto  Fakultet/Sektion – Faculty}\newline Faculty of Science} & \multicolumn{3}{>{\hsize=\dimexpr3\hsize-+3\tabcolsep\relax}S{X}?}{\textsf{Laitos/Institution– Department}\newline Department of Geosciences and Geography} \TBstrut\\
    \hline
    \multicolumn{6}{?>{\hsize=\dimexpr6\hsize-+6\tabcolsep\relax}S{X}?}{\textsf{Tekijä/Författare – Author }\newline Sampo Vesanen} \TBstrut\\
    \hline
    \multicolumn{6}{?>{\hsize=\dimexpr6\hsize-+6\tabcolsep\relax}S{X}?}{\textsf{Työn nimi / Arbetets titel – Title}\newline Parking of private cars in Helsinki Capital Region} \TBstrut\\
    \hline
    \multicolumn{6}{?>{\hsize=\dimexpr6\hsize-+6\tabcolsep\relax}S{X}?}{\textsf{Oppiaine /Läroämne – Subject}\newline Geography} \TBstrut\\
    \hline
    \multicolumn{2}{?>{\hsize=\dimexpr2\hsize + 2\tabcolsep\relax} S{X}|}{\textsf{Työn laji/Arbetets art – Level}\newline Master's thesis} & \multicolumn{2}{>{\hsize=\dimexpr2\hsize + 2\tabcolsep\relax} S{X}|} {\textsf{Aika/Datum – Month and year}\newline \monthyeardate\today} & \multicolumn{2}{>{\hsize=\dimexpr2\hsize + 2\tabcolsep\relax} S{X}?} {\textsf{Sivumäärä/ Sidoantal – Number of pages}\newline 1337 pages and annexes} \TBstrut\\
    \hline    
    \multicolumn{6}{?>{\hsize=\dimexpr6\hsize+6\tabcolsep\relax} S{X}?} {\textsf{Tiivistelmä/Referat – Abstract}\newline Olipa kerran gradu} \\ [540pt]
    \hline
    \multicolumn{6}{?>{\hsize=\dimexpr6\hsize-+6\tabcolsep\relax}S{X}?}{\textsf{Avainsanat – Nyckelord – Keywords}\newline parking, GIS, löysää} \TBstrut\\
    \hline
    \multicolumn{6}{?>{\hsize=\dimexpr6\hsize-+6\tabcolsep\relax}S{X}?}{\textsf{Säilytyspaikka – Förvaringställe – Where deposited}\newline Library of University of Helsinki} \TBstrut\\
    \hline
    \multicolumn{6}{?>{\hsize=\dimexpr6\hsize-+6\tabcolsep\relax}S{X}?}{\textsf{Muita tietoja – Övriga uppgifter – Additional information}\newline -} \TBstrut\\
    \Xhline{3\arrayrulewidth}
\end{tabularx}
\end{spacing}

\newpage
abstrakt FI

% Table of contents
\newpage
\tableofcontents

% INTRODUCTION AND RESEARCH QUESTIONS
\newpage
\section{Introduction}

% THEORETICAL BACKGROUND
\newpage
\section{Theoretical background}

% DATA AND METHODS
\newpage
\section{Data and methods}

% RESULTS
\newpage
\section{Results}

% DISCUSSION
\newpage
\section{Discussion}

% Yhdistä Mendeley-lähteet tähän .bib-tiedostoon
\newpage
\section{References}
\renewcommand{\refname}{}
\bibliography{references.bib}

% ANNEXES
\newpage
\section{Annexes}

\end{document}
