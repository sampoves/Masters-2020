
% Welcome to the University of Helsinki Master's thesis of Sampo Vesanen. This is the main document of this thesis, which sets up the software packages and document properties and calls individual tex files (thesis chapters) to complete this multifile LaTeX project. Please find below some links which I used as a basis in the technical structure of the thesis. Additionally, I have extensively commented the thesis in appropriate places to establish programmatical aspects of this LaTeX project.



% VIESTI HENKALLE, PETTERILLE JA TUULILLE:
% tilanne tiistai-iltapäivänä 6.10.2020: 
%- minulta puuttuu aukikirjoitettu teksti enää kappaleesta "2 Background", kohdasta PPGIS. Työstän sitä juuri nyt
%- Backgroundin jälkeen heti tiivistelmien kimppuun
%- Henkka antanut kommentteja 4.10., teen muutokset mahdollisimman nopeasti täyden tekstimitan saavuttamisen jälkeen


% Multifile LaTeX project:
%%% https://www.overleaf.com/learn/latex/Multi-file_LaTeX_projects
%%% https://en.wikibooks.org/wiki/LaTeX/Modular_Documents
% How to write a thesis in LaTeX
%%% https://www.overleaf.com/learn/latex/How_to_Write_a_Thesis_in_LaTeX_(Part_1):_Basic_Structure
% Microtype for thesis: 
%%% http://www.khirevich.com/latex/microtype/
% Multilanguage code adaptation from:
%%% https://tex.stackexchange.com/q/42784/182586

%  NB: It can be argued that the \documentclass should be {report} for a thesis
\documentclass[a4paper,11pt]{article}

% Set margins -----------------------------------------------------------
\usepackage{geometry}
\geometry{a4paper,total={170mm,257mm},left=20mm,right=20mm,}


% Font and layout settings ----------------------------------------------
\usepackage[activate={true,nocompatibility},tracking=true,kerning=true,factor=1100,stretch=10,shrink=10]{microtype}
\usepackage[finnish,english]{babel}         % Set languages. Last in list is main language
\usepackage[T1]{fontenc}
\usepackage[utf8]{inputenc}
\usepackage[font=small,labelfont=bf,labelsep=period]{caption}
\usepackage{ragged2e}                       % Justify text. Command: \justify
\usepackage{lastpage}                       % Get amount of pages
\usepackage{color}                          % Enable \textcolor{}{}
\usepackage{minted}                         % Format and highlight code
\usepackage{enumitem}                       % Control layout of itemize, enumerate, description
\usepackage{pdfpages}                       % pdf graphics
\renewcommand{\baselinestretch}{1.5}        % Set line spacing
\setlength{\abovecaptionskip}{15pt plus 3pt minus 2pt}
\setlength{\belowcaptionskip}{15pt plus 3pt minus 2pt}
\usepackage{csquotes}                       % Enable displayquote

% define \code{}: StackExchange style inline code ----------------------
\usepackage{soul}                           % typesetting with colour, supports linebreak (\colorbox on its own does not)
\usepackage[table]{xcolor}                  % use in horizontal lines in tables

% \ctext colorbox height
\makeatletter
    \def\SOUL@hlpreamble{%
        \setul{}{3.33ex}%                   default is 2.5ex
        \let\SOUL@stcolor\SOUL@hlcolor
        \SOUL@stpreamble
    }
\makeatother

% make note of thinspaces, enable thin whitespace inside \ctext colorbox
\newcommand{\ctext}[3][RGB]{%               Define \ctext
    \begingroup
        \definecolor{hlcolor}{#1}{#2}\sethlcolor{hlcolor}%
        \hl{\thinspace #3\thinspace}%
    \endgroup
}
\newcommand{\code}[1]{\texttt{\ctext[RGB]{242,242,242}{#1}}} % define \code command


% Generate list of abbreviations ---------------------------------------
\usepackage{xurl}                               % xurl linebreaks long links
\usepackage[breaklinks=true]{hyperref}          % Hyperlinks. Need to be activated before glossaries
\usepackage[acronym,style=super4col]{glossaries}% Glossaries and lists of acronyms


% Abbreviations --------------------------------------------------------
% Note the definition of plural forms where applicable
\newacronym{anova}{ANOVA}{Analysis of Variance test}
\newacronym[plural=IDEs,firstplural=Integrated development environments (IDE)]{ide}{IDE}{Integrated Development Environment}
\newacronym{php}{PHP}{Hypertext preprocessor}

% Nomenclature for glossary --------------------------------------------
% Note the definition of plural forms.
% \glssetwidest together with \printglossary parameters helps prevent overlong nomenclature lines
\glssetwidest[0]{...................}

\newglossaryentry{CPU}{
  name = {CPU},
  description = {Central Processing Unit},
  plural = {CPUs},
  firstplural = {\glsentrydesc{CPU}s (\glsentryplural{CPU})}
}
\newglossaryentry{GIS}{
  name = {GIS},
  description = {Geographic Information System},
}
\newglossaryentry{GB}{
  name = {GB},
  description = {Gigabyte is a multiple of the unit used to store digital information. A gigabyte is 1 000 megabytes},
}
\newglossaryentry{mysql}{
  name = {MySQL},
  description = {Open-source relational database management system},
}
\newglossaryentry{RAM}{
  name = {RAM},
  description = {Random Access Memory},
}
\makeglossaries

% Datetime settings  ------------------------------------------------------
\usepackage[ddmmyyyy]{datetime}             % Change the format of datetime

\newdateformat{mydate}{%
    {\THEDAY}.{\THEMONTH}.\THEYEAR}
\newdateformat{monthyeardate}{%
    \monthname[\THEMONTH] \THEYEAR}         % mm/yyyy date format for English abstract
\newcommand*{\printfindate}{%               % Get Finnish months and year for the Finnish abstract
   \ifcase \month\or Tammikuu\or Helmikuu\or Maaliskuu\or Huhtikuu\or Toukokuu\or Kesäkuu\or Heinäkuu\or Elokuu\or Syyskuu\or Lokakuu\or Marraskuu\or Joulukuu\fi \space \number\year}


% Images related packages and settings ------------------------------------
\usepackage{rotating, graphicx}
\graphicspath{ {images/} }


% References and reference styles -----------------------------------------
\usepackage{natbib}                         % Flexible bibliography support
\bibliographystyle{apalike2}                
\setcitestyle{authoryear,open={},close={}}  % Removes citation parentheses

% Define new command \fullref to reference both section number and name
\newcommand*{\fullref}[1]{\hyperref[{#1}]{\autoref*{#1}, \nameref*{#1}}}

% Define \citeintext to get citeauthor and citeyear in the same command
\newcommand*{\citeintext}[1]{\citeauthor{#1} (\citeyear{#1})}


% Other packages ----------------------------------------------------------
%\usepackage{alphabeta}                      % Greek alphabet
\usepackage{amsmath}                        % AMS mathematical facilities, equations
\usepackage{subfig}                         % Enable collections of subfigures as a single figure
\usepackage{gensymb}                        % Generic symbols for both text and math mode
\usepackage{float}                          % Improved interface for floating objects
\usepackage{pdfpages}                       % Include PDF documents in LaTeX
\usepackage{listings}                       % Typeset source code listings using LaTeX
\usepackage{import}                         % Establish input relative to a directory
\usepackage{xfp}                            % Math processing such as \fpeval

% Calculate increase in percentage. #1 new number, #2 original number
\newcommand\calcin[2]{\FPeval\result{round(((#1 - #2) / #2) * 100, 1)}\result~\%}


% Tables -----------------------------------------------------------------
\usepackage{array}                          % For table properties such as p{2cm}
\usepackage{tabularx}                       % Tabulars with adjustable-width columns
\usepackage{makecell}                       % Thicker vline
\usepackage{chngpage}                       % Allows for temporary adjustment of side margins
\usepackage{longtable}                      % Allows tables to span multiple pages
\usepackage{cellspace}                      % Ensure minimal spacing of table cells
\usepackage{booktabs}                       % Extra commands for tables
\usepackage{multirow}                       % Enable table cells which can span multiple rows
\usepackage{colortbl}                       % set custom colors for midrules
\usepackage{tablefootnote}                  % footnotes in tabular environment
\setlength{\cellspacetoplimit}{6pt}
\setlength{\cellspacebottomlimit}{6pt}
\addparagraphcolumntypes{X}
\newcolumntype{?}{!{\vrule width 1.5pt}}    % thicker vline for tabularx
\newcommand\Tstrut{\rule{0pt}{2.2ex}}       % Define "top" strut
\newcommand\Bstrut{\rule[-4.0ex]{0pt}{0pt}} % Define "bottom" strut
\newcommand{\TBstrut}{\Tstrut\Bstrut}       % Run top & bottom struts for rows
\newcommand{\ra}[1]{\renewcommand{\arraystretch}{#1}}

% \usepackage[table]{xcolor} and \usepackage{booktabs} to define a lighter midrule
\newcommand{\greyrule}{\arrayrulecolor{black!30}\midrule\arrayrulecolor{black}}

% Define extra wide column separators "C" (centering) and "L" (left-align)
\newlength\tbspace
\setlength\tbspace{0.5cm}
\newcolumntype{C}{c<{\hspace{\tbspace}}}%
\newcolumntype{L}{l<{\hspace{\tbspace}}}%


% Translator command setup -----------------------------------------------
% this enables us to use multilanguage strings which works with package babel
\newcommand{\newtranslatorcommand}[3]{%
    \newcommand{#1}{%
        \iflanguage{english}{#2}{%
        \iflanguage{finnish}{#3}{%
        }}%
    }%
}


% Title page settings -----------------------------------------------------
\usepackage{setspace}                       % Different line spacing than document default
\newcommand{\myname}{                       % This string represents thesis author in the document
    Sampo Vesanen%
}
\newcommand{\mysupervisors}{%
    Tuuli Toivonen \\
    Petteri Muukkonen \\
    Henrikki Tenkanen%
}
\newcommand{\addressFirst}{%
    PL 64 (Gustaf Hällströmin katu 2)
}
\newcommand{\addressSecond}{%
    00014 Helsingin yliopisto%
}


% Abstract page settings --------------------------------------------------
% THESIS MULTILANGUAGE STRINGS

\newtranslatorcommand{\mytitle}{%
    Parking private cars and accessibility in Helsinki Capital Region%
}{%
    Henkilöauton pysäköinti ja spatiaalinen saavutettavuus pääkaupunkiseudulla%
}

% Choose correct date format for abstract.
% Dates are now frozen for thesis release. Remove the date and restore the corresponding command if needed.
\newtranslatorcommand{\translatedate}{%
    October 2020%
    %\monthyeardate\today%
}{%
    Lokakuu 2020%
    %\printfindate%
}

% PAGE COUNT
\newtranslatorcommand{\mypages}{%
    \pageref{myLastPage} pages and appendices%
}{%
    \pageref{myLastPage} sivua ja liitteet%
}

% UNIVERSITY NAME
\newtranslatorcommand{\myuni}{%
    University of Helsinki%
}{%
    Helsingin yliopisto%
}

% FACULTY NAME
\newtranslatorcommand{\myfaculty}{%
    Faculty of Science%
}{%
    Matemaattisluonnontieteellinen tiedekunta%
}

% DEPARTMENT NAME
\newtranslatorcommand{\mydept}{%
    Department of Geosciences and Geography%
}{%
    Geotieteiden ja maantieteen osasto%
}

% SUBJECT NAME
\newtranslatorcommand{\mysubject}{%
    Geography
}{%
    Maantiede
}

% SPECIFIC SUBJECT NAME
\newtranslatorcommand{\mySpecSubject}{%
    Geoinformatics%
}{%
    geoinformatiikka%
}

% THESIS TYPE
\newtranslatorcommand{\mythesis}{%
    Master's thesis%
}{%
    Pro gradu -tutkielma%
}

% CHOOSE CORRECT ABSTRACT LANGUAGE
\newtranslatorcommand{\myabstract}{%
    \import{content/}{abstract_en.tex}%
}{%
    \import{content/}{abstract_fi.tex}%
}

% KEYWORDS
\newtranslatorcommand{\mykeywords}{%
    GIS, R, parking, Python, accessibility%
}{%
    GIS, R, pysäköinti, Python, saavutettavuus%
}

% LIBRARY TYPE
\newtranslatorcommand{\mylibrary}{%
    E-Thesis / Helda%
}{%
    E-Thesis / Helda%
}

% OTHER INFO
\newtranslatorcommand{\myotherinfo}{%
    -%
}{%
    -%
}

\title{\mytitle}
\author{\myname}
\date{8.10.2020} % Date is now frozen. Use \today to get, today.


% BEGIN DOCUMENT PROPER ---------------------------------------------------
\begin{document}

% TITLE PAGE
\import{elements/}{titlepage.tex}

%%% 1ST ABSTRACT PAGE, ENGLISH
\newpage
\import{elements/}{abstract.tex}

%%% 2ND ABSTRACT PAGE, FINNISH
\newpage
\begin{otherlanguage*}{finnish}
\import{elements/}{abstract.tex}
\end{otherlanguage*}

% TABLE OF CONTENTS
\newpage
\selectlanguage{english}
\thispagestyle{empty} % suppress page numbering
\renewcommand*\contentsname{Table of contents} % Set custom name for ToC
\tableofcontents

% LIST OF FIGURES
\newpage
\listoffigures
\thispagestyle{empty}

% LIST OF TABLES
\listoftables

% LIST OF ABBREVIATIONS AND NOMENCLATURE
% NB, nomenclature labels will not line break if parameters nonumberlist and style=alttree are not combined with \glssetwidest.
\newpage
\printglossary[type=\acronymtype,title=List of abbreviations]
\printglossary[nonumberlist,style=alttree,title=Nomenclature]
\thispagestyle{empty}

% INTRODUCTION AND RESEARCH QUESTIONS
\newpage
\setcounter{page}{1} % set page numbering to start from here
\section{Introduction}
\justify
% --- what to include in intro ---
% Glimpse to essence of the current scientific discussion \par
% terminology \par
% viewpoint of previous studies \par
% gaps in current scientific knowledge \par
% how study relates to existing ones \par
% main themes and challenges dealt with the present study \par

% In the center of this thesis is the work of \citeauthor{Salonen2013} where they presented their door-to-door approach (\citeyear{Salonen2013}). 

Amidst the raised awareness of climate change, the amount of private cars is globally on the rise. According to one estimation, the world reached one billion cars in 2020 (\cite{Sperling2009}). \textcolor{red}{OICA sanoo, että 1 mrd kaaraa 2015}. With no slowing in sight for the production of private cars, eyes must turn to managing the vast quantity of personal transportation in cities and in their surroundings. It is a question of mitigation of climate change but also maximising the quality of life for urban citizens everywhere. (\cite{StatisticsFinland2019})

Cities face many challenges, particularly in relation to mobility of people and structure of land use. Parking management is a way to link land use and transportation (\cite{Marsden2006}; \cite{Diallo2015}).

Goals of parking policy are numerous, for example optimal accessibility and traffic flow and maximising turn-over for shops (\cite{Marsden2006}).

Travel time is considered an intuitive measure to indicate accessibility and a strong predictor of mode choice (\cite{Frank2008}).

Donald Shoup's Crusing for Parking and The High Cost of Free Parking, Beyond Mobility book?

To realistically model travel times, one needs to take into account the whole process of private car journeys, including the entire parking process (\cite{Salonen2013}).

This thesis is connected with the work of Digital Geography Lab (\textcolor{blue}{\url{https://www.helsinki.fi/en/researchgroups/digital-geography-lab}}) and Accessibility Research Group (\textcolor{blue}{\url{https://blogs.helsinki.fi/accessibility}}) within University of Helsinki.

This thesis strives for maximum transparency and repeatability. All parts of this thesis are available online at GitHub (\textcolor{blue}{\url{https://github.com/sampoves/Masters-2020}} and \textcolor{blue}{\url{https://github.com/sampoves/Msc_thesis_data_analysis}}). This includes the entire thesis in LaTeX format, the parking survey programmed in JavaScript, instructions to set up the web server as used in this research survey, and the survey data and analysis scripts programmed in Python and R. Complete development histories of all components are included. All of the work is released using \textcolor{red}{MIT or something} license. In addition to the work proper, as side products, an empty template of this thesis (\textcolor{blue}{insert link}) and a point based variant of the park survey (\textcolor{blue}{insert link}) have been made available.

I would like to thank Harri Lampi, Johannes Nyman, Samuli Pitzén and Panu Vesanen for their invaluable help in various aspects of this thesis.

\bigskip
\noindent
The research questions for this thesis are:

\begin{spacing}{1}
    % use Roman numerals, uses enumitem
    \begin{enumerate}[label=\Roman*]
      \item What are the spatial differences in the time that it takes to find a parking spot and park one’s car in the study area?
      \item If spatial differences are detected, what explains them?
      \item What is the significance of the parking process to the overall travel time?
    \end{enumerate}
\end{spacing}
\bigskip
In addition to these research questions this thesis explores how well the map survey created for this thesis worked in collecting user data in a spatial manner.

% THEORETICAL BACKGROUND
\newpage
\import{content/}{c2_background.tex}%

% DATA AND METHODS
\newpage
\import{content/}{c3_data.tex}%

% RESULTS
\newpage
\import{content/}{c4_results.tex}%

% DISCUSSION
\newpage
\import{content/}{c5_discussion.tex}%

% REFERENCES
% Connect Mendeley references to Overleaf bib file
\newpage
\label{myLastPage} % This label defines the last page to be included in page count
\section{References}
\renewcommand{\refname}{}
\bibliography{references.bib}

% APPENDICES
\newpage
\pagenumbering{gobble} % Suppress appendices page numbering from page and table of contents
\appendix
\import{content/}{appendices.tex}%

\end{document}