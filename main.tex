% Sampo notes
% Tämä on mun gradu. Tässä alussa vähän löpinää siitä, mistä kyse ja varmaan jotain linkkejä

%%% LAITOKSEN VANHAT GRADUOHJEET
% https://blogs.helsinki.fi/geotiede/files/2008/03/Pro-gradu-ohjeet.pdf
% FLAMMAN GRADUOHJEET:
%%% https://flamma.helsinki.fi/portal/units/matlu?_nfpb=true&_pageLabel=P9806621561360920311893&contentId=HY358060&placeId=HY357991&lang=fi
% Multifile LaTeX project:
%%% https://www.overleaf.com/learn/latex/Multi-file_LaTeX_projects
%%% https://en.wikibooks.org/wiki/LaTeX/Modular_Documents
% How to write a thesis in LaTeX
%%% https://www.overleaf.com/learn/latex/How_to_Write_a_Thesis_in_LaTeX_(Part_1):_Basic_Structure

\documentclass[a4paper,11pt]{article}

% Set margins
\usepackage{geometry}
\geometry{
    a4paper,
    total={170mm,257mm},
    left=20mm,
    right=20mm,
    }

% Font and layout settings
% tutki microtypen käyttöö: http://www.khirevich.com/latex/microtype/
%\usepackage[activate={true,nocompatibility},tracking=true,kerning=true,factor=1100,stretch=10,shrink=10]{microtype}
\renewcommand{\baselinestretch}{1.5}        % Set line spacing
\usepackage[finnish,english]{babel}         % Set languages. Last in list is main language
\usepackage[T1]{fontenc}
\usepackage[utf8]{inputenc}
\usepackage[font=small,labelfont=bf]{caption}
\setlength{\abovecaptionskip}{15pt plus 3pt minus 2pt}
\setlength{\belowcaptionskip}{15pt plus 3pt minus 2pt}
\usepackage{ragged2e}                       % Enables justifying of text. Command: \justify
\usepackage{lastpage}                       % Get amount of pages
%\usepackage{subfiles}                      % Enable multi-file .tex support
\usepackage{color}                          % Enable \textcolor{}{}
\usepackage{minted}                         % Format and highlight code
\usepackage{enumitem}                       % Control layout of itemize, enumerate, description

% Generate list of abbreviations
\usepackage{hyperref}                       % Enable hyperlinks in thesis. Need to be activated before glossaries
\usepackage[acronym,style=super4col]{glossaries} % Glossaries and lists of acronyms

\newacronym{php}{PHP}{Hypertext preprocessor}% abbreviations

\newglossaryentry{mysql}{                    % nomenclature
  name = $MySQL$,
  description = Open-source relational database management system,
}

\makeglossaries

% Datetime settings
\usepackage[ddmmyyyy]{datetime}             % Change format of datetime
%\newdateformat{mydate}{%
%    \onedigit{\THEDAY}.\onedigit{\THEMONTH}.\THEYEAR} % remove unnecessary zeroes in date
\newdateformat{mydate}{%
    {\THEDAY}.{\THEMONTH}.\THEYEAR}         % The one above causes undefined control sequence   
\newdateformat{monthyeardate}{%
    \monthname[\THEMONTH] \THEYEAR}         % mm/yyyy date format for English abstract
\newcommand*{\printfindate}{%               % Get Finnish months and year for Finnish abstract
   \ifcase \month\or Tammikuu\or Helmikuu\or Maaliskuu\or Huhtikuu\or Toukokuu\or Kesäkuu\or Heinäkuu\or Elokuu\or Syyskuu\or Lokakuu\or Marraskuu\or Joulukuu\fi \space \number\year}

% Images related packages and settings
\usepackage{rotating, graphicx}
\graphicspath{ {images/} }

% references and reference styles
\usepackage{natbib}                         % Flexible bibliography support
\bibliographystyle{apalike2}                
\setcitestyle{authoryear,open={},close={}}  % Removes citation parentheses



% Other packages
%\usepackage{alphabeta}                      % Greek alphabet
%\usepackage{amsmath}                        % AMS mathematical facilities
\usepackage{subfigure}
\usepackage{gensymb}                        % Generic symbols for both text and math mode
%\usepackage[colorinlistoftodos]{todonotes}  % Marking things to do in a LATEX document
%\usepackage{amsthm}                         % Typesetting theorems (AMS style)
\usepackage{float}                          % Improved interface for floating objects
\usepackage{pdfpages}                       % Include PDF documents in LATEX
\usepackage{listings}                       % Typeset source code listings using LATEX
\usepackage{import}                         % Establish input relative to a directory

% Tables
%\usepackage[showframe]{geometry}           % double load of geometry? what does showframe do?
\usepackage{tabularx}                       % Tabulars with adjustable-width columns
\usepackage{makecell}                       % enable thicker vline
\usepackage{chngpage}                       % allows for temporary adjustment of side margins
\usepackage{longtable}                      % allows tables to span multiple pages
\usepackage{cellspace}                      % Ensure minimal spacing of table cells
\setlength{\cellspacetoplimit}{6pt}
\setlength{\cellspacebottomlimit}{6pt}
\addparagraphcolumntypes{X}
\newcolumntype{?}{!{\vrule width 1.5pt}}    % thicker vline for tabularx
\newcommand\Tstrut{\rule{0pt}{2.2ex}}       % "top" strut
\newcommand\Bstrut{\rule[-4.0ex]{0pt}{0pt}} % "bottom" strut
\newcommand{\TBstrut}{\Tstrut\Bstrut}       % top & bottom struts

% Translator command setup
% this enables us to use multilanguage strings which works with package babel
\newcommand{\newtranslatorcommand}[3]{%
    \newcommand{#1}{%
        \iflanguage{english}{#2}{%
        \iflanguage{finnish}{#3}{%
        }}%
    }%
}

% Title page settings
\usepackage{setspace}                       % Different line spacing than document default
\newcommand{\myname}{                       % THIS STRING REPRESENTS THESIS AUTHOR IN THE DOCUMENT
    Sampo Vesanen%
}
\newcommand{\mysupervisor}{%
    Tuuli Toivonen%
}
\newcommand{\addressFirst}{%
    PL 64 (Gustaf Hällströmin katu 2)
}
\newcommand{\addressSecond}{%
    00014 Helsingin yliopisto%
}

% Abstract page settings
% THESIS MULTILANGUAGE STRINGS
% Multilanguage code is adapted from here:
% https://tex.stackexchange.com/questions/42784/language-dependent-custom-command
\newtranslatorcommand{\mytitle}{%
    Parking private cars and accessibility in Helsinki Capital Region%
}{%
    Henkilöauton pysäköinti ja spatiaalinen saavutettavuus pääkaupunkiseudulla%
}

% Choose correct date format for abstract
\newtranslatorcommand{\translatedate}{%
    \monthyeardate\today%
}{%
    \printfindate%
}

% PAGE COUNT MULTILANUAGE STRING
\newtranslatorcommand{\mypages}{%
    \pageref{myLastPage} pages and appendices%
}{%
    \pageref{myLastPage} sivua ja liitteet%
}

% UNIVERSITY MULTILANGUAGE STRING
\newtranslatorcommand{\myuni}{%
    University of Helsinki%
}{%
    Helsingin yliopisto%
}

% FACULTY MULTILANGUAGE STRING
\newtranslatorcommand{\myfaculty}{%
    Faculty of Science%
}{%
    Matemaattisluonnontieteellinen tiedekunta%
}

% DEPARTMENT MULTILANGUAGE STRING
\newtranslatorcommand{\mydept}{%
    Department of Geosciences and Geography%
}{%
    Geotieteiden ja maantieteen osasto%
}

% GEOGRAPHY MULTILANGUAGE STRING
\newtranslatorcommand{\mysubject}{%
    Geography
}{%
    Maantiede
}

% GEOINFORMATICS MULTILANGUAGE STRING
\newtranslatorcommand{\mySpecSubject}{%
    Geoinformatics%
}{%
    geoinformatiikka%
}

% THESIS TYPE MULTILANGUAGE STRING
\newtranslatorcommand{\mythesis}{%
    Master's thesis%
}{%
    Pro gradu -tutkielma%
}

% CHOOSE CORRECT ABSTRACT LANGUAGE
\newtranslatorcommand{\myabstract}{%
    \import{content/}{abstract_en.tex}%
}{%
    \import{content/}{abstract_fi.tex}%
}

% KEYWORDS MULTILANGUAGE STRING
\newtranslatorcommand{\mykeywords}{%
    Parking, accessibility, GIS, Python%
}{%
    Pysäköinti, saavutettavuus, GIS, Python%
}

% LIBRARY MULTILANGUAGE STRING
\newtranslatorcommand{\mylibrary}{%
    Kumpula science library%
}{%
    Kumpulan tiedekirjasto%
}

% OTHER INFO MULTILANGUAGE STRING
\newtranslatorcommand{\myotherinfo}{%
    -%
}{%
    -%
}

\title{\mytitle}
\author{\myname}
\date{\today}


% BEGIN DOCUMENT PROPER
\begin{document}

% TITLE PAGE
\import{elements/}{titlepage.tex}

%%% 1ST ABSTRACT PAGE, ENGLISH
\newpage
\import{elements/}{abstract.tex}

%%% 2ND ABSTRACT PAGE, FINNISH
\newpage
\begin{otherlanguage*}{finnish}
\import{elements/}{abstract.tex}
\end{otherlanguage*}

% TABLE OF CONTENTS
\newpage
\selectlanguage{english}
\thispagestyle{empty} % suppress page numbering
\renewcommand*\contentsname{Table of contents} % Set custom name for ToC
\tableofcontents

% LIST OF FIGURES
\newpage
\listoffigures
\thispagestyle{empty}

% LIST OF TABLES
\listoftables

% LIST OF ABBREVIATIONS
\printglossary[type=\acronymtype,title=List of abbreviations]
\printglossary[title=Nomenclature]

% INTRODUCTION AND RESEARCH QUESTIONS
\newpage
\setcounter{page}{1} % set page numbering to start from here
\section{Introduction}
\justify
contentkansio Lorem ipsum dolor sit amet (\cite{Geurs2004}), consectetur adipiscing elit. Sed dapibus nisl nec nisi sagittis iaculis at vel tortor. Donec a lacus tristique, commodo leo a, viverra risus. In nec dolor facilisis, egestas ipsum quis, mattis lectus. Donec mattis quam tellus, ut consectetur magna posuere at. Donec nec malesuada est, et porta ex. Fusce pellentesque, felis et congue lobortis, lectus diam imperdiet libero, ac dapibus nulla purus quis est. Vivamus mauris dolor, iaculis nec vestibulum eu, consectetur sollicitudin tellus. Cras commodo suscipit dui, a vehicula ligula hendrerit at. Vivamus semper venenatis nulla ac finibus. Maecenas sagittis ornare sapien at venenatis. In justo urna, molestie sed sem nec, suscipit ornare orci.

Cras sed tincidunt enim. Donec sagittis dapibus velit, sit amet egestas dui hendrerit condimentum. Nunc non ipsum tristique, varius neque in, faucibus ante. Ut scelerisque facilisis turpis, nec eleifend orci efficitur et. Nam et ante quis dui iaculis mattis. Quisque ultrices cursus urna, eget venenatis dui iaculis ut. Suspendisse feugiat elementum nisl, ut pellentesque lacus tincidunt ut. Phasellus quis magna dui. Donec pharetra ultrices elit, id interdum est pharetra interdum. Nullam justo est, volutpat et tristique vitae, fringilla nec leo. Donec pharetra mollis sem quis bibendum. Aenean sodales ultrices fermentum.

Donec fringilla rhoncus tellus auctor venenatis. Nullam consequat ex libero, quis rutrum neque tincidunt ut. Donec est odio, vehicula sit amet felis eget, facilisis mattis nisi. Quisque vitae mattis odio. Pellentesque sapien lorem, fermentum ut blandit ut, ultricies non lacus. Etiam tristique accumsan nunc, id tincidunt ligula commodo vitae. Fusce commodo arcu arcu, id tempor tortor mattis id. Vivamus eget feugiat quam, ut ornare ante.

The specific research questions for this thesis are:

\begin{spacing}{1}
\begin{enumerate}[label=\Roman*] % use Roman numerals, uses enumitem package
  \item What are the spatial differences in the time that it takes to find and park one’s car in the study area?
  \item If spatial differences are detected, what explains them?
  \item What is the significance of the process of parking one’s car to the overall travel time?
\end{enumerate}
\end{spacing}

% THEORETICAL BACKGROUND
\newpage
\import{content/}{c2_background.tex}%

% DATA AND METHODS
\newpage
\import{content/}{c3_data.tex}%

% RESULTS
\newpage
\import{content/}{c4_results.tex}%

% DISCUSSION
\newpage
\import{content/}{c5_discussion.tex}%

% REFERENCES
% Connect Mendeley references to Overleaf bib file
\newpage
\label{myLastPage} % This label defines the last page to be included in page count
\section{References}
\renewcommand{\refname}{}
\bibliography{references.bib}

% APPENDICES
\newpage
\pagenumbering{gobble} % Suppress appendices page numbering from page and table of contents 
\import{content/}{appendices.tex}%

\end{document}