% Sampo notes
% Tämä on mun gradu. Tässä alussa vähän löpinää siitä, mistä kyse ja varmaan jotain linkkejä

%%% LAITOKSEN VANHAT GRADUOHJEET
% https://blogs.helsinki.fi/geotiede/files/2008/03/Pro-gradu-ohjeet.pdf
% FLAMMAN GRADUOHJEET:
%%% https://flamma.helsinki.fi/portal/units/matlu?_nfpb=true&_pageLabel=P9806621561360920311893&contentId=HY358060&placeId=HY357991&lang=fi

\documentclass[a4paper,11pt]{article}

% Set margins
\usepackage{geometry}
\geometry{
    a4paper,
    total={170mm,257mm},
    left=20mm,
    right=20mm,
    }

% Font and layout settings
\renewcommand{\baselinestretch}{1.5}        % set line spacing
\usepackage[finnish,english]{babel}         % set languages. Last in list is main language
\usepackage[T1]{fontenc}
\usepackage[utf8]{inputenc}
\usepackage[font=small,labelfont=bf]{caption}
\setlength{\abovecaptionskip}{15pt plus 3pt minus 2pt}
\setlength{\belowcaptionskip}{15pt plus 3pt minus 2pt}
\usepackage[document]{ragged2e}             % package enables justifying of text. \justify
\usepackage{lastpage}                       % get amount of pages

% Datetime settings
\usepackage[ddmmyyyy]{datetime}             % change format of datetime
%\renewcommand{\dateseparator}{.}
\newdateformat{mydate}%
    {\onedigit{\THEDAY}.\onedigit{\THEMONTH}.\THEYEAR} % remove unnecessary zeroes in date
\newdateformat{monthyeardate}%
    {\monthname[\THEMONTH] \THEYEAR}        % mm/yyyy date format for English abstract
\newcommand*{\printfindate}{%               % Get Finnish months and year for Finnish abstract
   \ifcase \month\or Tammikuu\or Helmikuu\or Maaliskuu\or Huhtikuu\or Toukokuu\or Kesäkuu\or Heinäkuu\or Elokuu\or Syyskuu\or Lokakuu\or Marraskuu\or Joulukuu\fi \space \number\year}

% location of files
\usepackage{graphicx}
\graphicspath{ {Images/} }

% references and reference styles
\usepackage{enumitem}% http://ctan.org/pkg/enumitem
\usepackage{natbib}
\bibliographystyle{agsm}
\setcitestyle{authoryear,open={},close={}}  % Removes citation parentheses

% other packages
\usepackage{alphabeta}
\usepackage{amsmath} 
\usepackage{subfigure}
\usepackage{gensymb}
\usepackage[colorinlistoftodos]{todonotes}
\usepackage{hyperref} 
\usepackage{amsthm} 
\usepackage{float}
\usepackage{pdfpages}
\usepackage{minted}
\usepackage{enumitem}
\usepackage{listings}
\usepackage{color}

% tables
\usepackage[showframe]{geometry}
\usepackage{tabularx}
\usepackage{makecell}                       % enable thicker vline
\usepackage{cellspace}
\setlength{\cellspacetoplimit}{6pt}
\setlength{\cellspacebottomlimit}{6pt}
\addparagraphcolumntypes{X}
\newcolumntype{?}{!{\vrule width 1.5pt}}    % thicker vline for tabularx
\newcommand\Tstrut{\rule{0pt}{2.2ex}}       % "top" strut
\newcommand\Bstrut{\rule[-4.0ex]{0pt}{0pt}} % "bottom" strut
\newcommand{\TBstrut}{\Tstrut\Bstrut}       % top & bottom struts

% Title page settings
\usepackage{setspace}                       % different line spacing than document default
\date{\today}
\newcommand{\myname}{                      % THIS STRING REPRESENTS THESIS AUTHOR IN THE DOCUMENT
    Sampo Vesanen
}
\newcommand{\mytitle}{                      % THIS STRING REPRESENTS THESIS TITLE IN THE DOCUMENT
    Parking of private cars and spatial accessibility in Helsinki Capital Region
}
\title{ \mytitle}
\author{ \myname}

% Begin document proper
\begin{document}

% TITLE PAGE
\begin{titlepage}{
    \setstretch{1.0}
    \centering
    \begin{figure}[t]
    \centering
    \includegraphics[scale=0.4]{HY_logo}
    \end{figure}
    
    Master's thesis \par
    Geography \par
    Geoinformatics \par
    
    \bigskip
    \MakeUppercase{\mytitle}
    
    \bigskip
    \myname
    
    \mydate\today
    
    % Fills empty space between top and bottom of page
    \vfill
    
    Supervisor: \par
    Tuuli Toivonen \par
    \bigskip
    \bigskip
    UNIVERSITY OF HELSINKI\par
    FACULTY OF SCIENCE\par
    DEPARTMENT OF GEOSCIENCES\par
    GEOGRAPHY\par
    PL 64 (Gustaf Hällströmin katu 2)\par
    00014 Helsingin yliopisto\par
}
\end{titlepage}

%%% 1ST ABSTRACT PAGE, ENGLISH
\newpage
\thispagestyle{empty} % hides page number
\noindent
\begin{spacing}{1}
\footnotesize % set table font size to small

% Tabularx table notes:
%%% Circumvent having to use halves of columns by using six
%%% In this table '?' means thicker vline. See definitions for tables in the beginning of the document
%%% Idea from: https://tex.stackexchange.com/questions/236155/tabularx-and-multicolumn
%%% TBstrut is also defined above. See following address for the full explanation:
%%%%% https://tex.stackexchange.com/questions/126539/padding-at-the-top-of-a-table-cell-in-latex
\begin{tabularx}{\linewidth}{|S{X}S{X}|S{X}|S{X}|S{X}|S{X}|}
    \Xhline{3\arrayrulewidth}
    % FIRST ROW ENGLISH
    \multicolumn{3}{?>{\hsize=\dimexpr3\hsize-+3\tabcolsep\relax}S{X}|}{\textsf{Tiedekunta/Osasto  Fakultet/Sektion – Faculty}\newline Faculty of Science} & %
    \multicolumn{3}{>{\hsize=\dimexpr3\hsize-+3\tabcolsep\relax}S{X}?}{\textsf{Laitos/Institution– Department}\newline Department of Geosciences and Geography} \TBstrut\\
    \hline
    % SECOND ROW
    \multicolumn{6}{?>{\hsize=\dimexpr6\hsize-+6\tabcolsep\relax}S{X}?}{\textsf{Tekijä/Författare – Author }\newline \myname} \TBstrut\\
    \hline
    % THIRD ROW
    \multicolumn{6}{?>{\hsize=\dimexpr6\hsize-+6\tabcolsep\relax}S{X}?}{\textsf{Työn nimi / Arbetets titel – Title}\newline \mytitle} \TBstrut\\
    \hline
    % FOURTH ROW
    \multicolumn{6}{?>{\hsize=\dimexpr6\hsize-+6\tabcolsep\relax}S{X}?}{\textsf{Oppiaine /Läroämne – Subject}\newline Geography (geoinformatics)} \TBstrut\\
    \hline
    % FIFTH ROW
    \multicolumn{2}{?>{\hsize=\dimexpr2\hsize + 2\tabcolsep\relax} S{X}|}{\textsf{Työn laji/Arbetets art – Level}\newline Master's thesis} & %
    \multicolumn{2}{>{\hsize=\dimexpr2\hsize + 2\tabcolsep\relax} S{X}|} {\textsf{Aika/Datum – Month and year}\newline \monthyeardate\today} & %
    \multicolumn{2}{>{\hsize=\dimexpr2\hsize + 2\tabcolsep\relax} S{X}?} {\textsf{Sivumäärä/ Sidoantal – Number of pages}\newline \pageref{myLastPage} pages and appendices} \TBstrut\\
    \hline    
    % SIXTH ROW
    \multicolumn{6}{?>{\hsize=\dimexpr6\hsize+6\tabcolsep\relax} S{X}?} {\textsf{Tiivistelmä/Referat – Abstract}\newline Once upon a gradu} \\ [540pt]
    \hline
    % SEVENTH ROW
    \multicolumn{6}{?>{\hsize=\dimexpr6\hsize-+6\tabcolsep\relax}S{X}?}{\textsf{Avainsanat – Nyckelord – Keywords}\newline parking, accessibility, GIS, Python} \TBstrut\\
    \hline
    % EIGHTH ROW
    \multicolumn{6}{?>{\hsize=\dimexpr6\hsize-+6\tabcolsep\relax}S{X}?}{\textsf{Säilytyspaikka – Förvaringställe – Where deposited}\newline Kumpula science library} \TBstrut\\
    \hline
    % LAST ROW
    \multicolumn{6}{?>{\hsize=\dimexpr6\hsize-+6\tabcolsep\relax}S{X}?}{\textsf{Muita tietoja – Övriga uppgifter – Additional information}\newline -} \TBstrut\\
    \Xhline{3\arrayrulewidth}
\end{tabularx}
\end{spacing}

%%% 2ND ABSTRACT PAGE, FINNISH
\newpage
\noindent
\thispagestyle{empty}
\begin{spacing}{1}
\footnotesize
\begin{otherlanguage*}{finnish}
\begin{tabularx}{\linewidth}{|S{X}S{X}|S{X}|S{X}|S{X}|S{X}|}
    \Xhline{3\arrayrulewidth}
    % FIRST ROW FINNISH
    \multicolumn{3}{?>{\hsize=\dimexpr3\hsize-+3\tabcolsep\relax}S{X}|}{\textsf{Tiedekunta/Osasto  Fakultet/Sektion – Faculty}\newline Matemaattisluonnontieteellinen tiedekunta} & %
    \multicolumn{3}{>{\hsize=\dimexpr3\hsize-+3\tabcolsep\relax}S{X}?}{\textsf{Laitos/Institution– Department}\newline Geotieteiden ja maantieteen osasto} \TBstrut\\
    \hline
    % SECOND ROW
    \multicolumn{6}{?>{\hsize=\dimexpr6\hsize-+6\tabcolsep\relax}S{X}?}{\textsf{Tekijä/Författare – Author }\newline \myname} \TBstrut\\
    \hline
    % THIRD ROW
    \multicolumn{6}{?>{\hsize=\dimexpr6\hsize-+6\tabcolsep\relax}S{X}?}{\textsf{Työn nimi / Arbetets titel – Title}\newline Henkilöauton pysäköinti ja spatiaalinen saavutettavuus pääkaupunkiseudulla} \TBstrut\\
    \hline
    % FOURTH ROW
    \multicolumn{6}{?>{\hsize=\dimexpr6\hsize-+6\tabcolsep\relax}S{X}?}{\textsf{Oppiaine /Läroämne – Subject}\newline Maantiede (geoinformatiikka)} \TBstrut\\
    \hline
    % FIFTH ROW
    \multicolumn{2}{?>{\hsize=\dimexpr2\hsize + 2\tabcolsep\relax} S{X}|}{\textsf{Työn laji/Arbetets art – Level}\newline Pro gradu -tutkielma} & %
    \multicolumn{2}{>{\hsize=\dimexpr2\hsize + 2\tabcolsep\relax} S{X}|} {\textsf{Aika/Datum – Month and year}\newline \printfindate} & %
    \multicolumn{2}{>{\hsize=\dimexpr2\hsize + 2\tabcolsep\relax} S{X}?} {\textsf{Sivumäärä/ Sidoantal – Number of pages}\newline \pageref{myLastPage} sivua ja liitteet} \TBstrut\\
    \hline    
    % SIXTH ROW
    \multicolumn{6}{?>{\hsize=\dimexpr6\hsize+6\tabcolsep\relax} S{X}?} {\textsf{Tiivistelmä/Referat – Abstract}\newline Olipa kerran gradu} \\ [500pt]
    \hline
    % SEVENTH ROW
    \multicolumn{6}{?>{\hsize=\dimexpr6\hsize-+6\tabcolsep\relax}S{X}?}{\textsf{Avainsanat – Nyckelord – Keywords}\newline pysäköinti, saavutettavuus, GIS, Python} \TBstrut\\
    \hline
    % EIGHTH ROW
    \multicolumn{6}{?>{\hsize=\dimexpr6\hsize-+6\tabcolsep\relax}S{X}?}{\textsf{Säilytyspaikka – Förvaringställe – Where deposited}\newline Kumpulan tiedekirjasto} \TBstrut\\
    \hline
    % LAST ROW    
    \multicolumn{6}{?>{\hsize=\dimexpr6\hsize-+6\tabcolsep\relax}S{X}?}{\textsf{Muita tietoja – Övriga uppgifter – Additional information}\newline -} \TBstrut\\
    \Xhline{3\arrayrulewidth}
\end{tabularx}
\end{otherlanguage*}
\end{spacing}

% Table of contents
\newpage
\selectlanguage{english}
\thispagestyle{empty}
\renewcommand*\contentsname{Table of contents} % Set custom name for ToC
\tableofcontents

% INTRODUCTION AND RESEARCH QUESTIONS
\newpage
\setcounter{page}{1} % set page numbering to start from here
\section{Introduction}
\justify
Tässä kappaleessa ei alaotsikoita.

The specific research questions for this thesis are:

\begin{spacing}{1}
\begin{enumerate}[label=\Roman*] % use Roman numerals, uses enumitem package
  \item What are the spatial differences in the time that it takes to find and park one’s car in the study area?
  \item If spatial differences are detected, what explains them?
  \item What is the significance of the process of parking one’s car to the overall travel time?
\end{enumerate}
\end{spacing}

% THEORETICAL BACKGROUND
\newpage
\section{Background}
\subsection{Parking studies}
\subsection{Parking time estimations}

% DATA AND METHODS
\newpage
\section{Data and methods}
\subsection{Parking survey}
\justify
To test if parking really is hard in Helsinki, I created a survey in JavaScript and Leaflet. It was actually the second time I was doing the survey as the first one, done with Survey123 didn't get approved. That really sucked man.

% RESULTS
\newpage
\section{Results}

% DISCUSSION
\newpage
\section{Discussion}

% REFERENCES
% Connect Mendeley references to Overleaf bib file
\newpage
\label{myLastPage} % this label defines the last page to be included in page count
\section{References}
\renewcommand{\refname}{}
\bibliography{references.bib}

% APPENDICES
\newpage
\pagenumbering{gobble} % suppress appendices page numbering from page and table of contents 
\section{Appendices}

\end{document}
