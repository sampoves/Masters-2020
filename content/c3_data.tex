\section{Data and methods}
\subsection{General workflow}
\justify

For this thesis, a spatial survey web application was designed and programmed. It was a long, multistage process involving changing technologies and research objectives. In this chapter the process to create the research survey is presented, from the design board to a functional web application to the end stage of data processing and visualisation. 

Essential data for this research was provided by Statistics Finland, The research group Digital Geography Lab of University of Helsinki, municipalities of Helsinki Capital Region, Finnish Environment Institute, and European Environment Agency. The research area of this thesis is based on PAAVO open data by postal code area (\cite{StatisticsFinland2019a}. Data such as Urban Atlas 2012 forest polygons and areas of urban structure were utilised to provide additional explanatory variables in the survey data (\cite{EuropeanEnvironmentAgency2016}; \cite{Ristimaki2017}). For visualisation of the survey results, various data by Statistics Finland was used (\cite{StatisticsFinland2012}).

Several programming languages and their libraries were used in the survey web application and subsequent data processing, analysis and visualisation. The survey itself was created with HTML, JavaScript and PHP, with the essential help of Leaflet, a JavaScript mapping library (\cite{Agafonkin2019}). Data processing was shared between Python and R, with most of the processing done in Python and most of the analysis and visualisation in R. In Python, the library GeoPandas was in central role, while the interactive analysis and visualisation application was created with Shiny (\cite{GeoPandasDevelopers2019}, \cite{Chang2019}). The thesis itself was programmed with LaTeX, using the online LaTeX editor Overleaf (\textcolor{red}{cite}).

Initially the research survey strove to collect parking event data on the precise temporal and spatial resolutions of one parking event and the exact coordinates, respectively. This first version of the survey was released to a small group of people, but it was quickly decided that a spatial survey of a more general resolution was needed. After consideration, one was programmed from the ground up using HTML and JavaScript. In this new survey, which was rolled out in May 2019, the respondent would answer about their parking activities in specific postal code areas generally. The reduction in data resolution was substantial, but would still have more fidelity than the existing parking time data in Travel-time Matrix 2018.

The survey was carried out in the four municipalities of Helsinki Capital Region. An invitation to participate in the survey was mostly spread in Facebook city district and neighborhood groups. Officially the response collection ended in June 2019, but the survey remained open until October 2019. The survey gathered, in total, 5222 responses from 4309 unique IP addresses.

After finishing the data collection phase, the data was processed, analysed and visualised using Python and R programming languages. The process started with anonymisation of IP address data, and moved on to the processing proper. The objective of data processing in Python was to bind the survey data into a selection of open spatial data, in the effort to create new explanatory variables for the analysis phase.

Most of the survey result analysis and visualisation was carried out in R. Utilising Shiny, a web application framework library for R, a data analysis and visualisation application was programmed for efficient and flexible analysis, but also to release the tool to the public, maintaining the mission of openness and transparency of this thesis.

The general workflow of survey data processing can be viewed in figure~\ref{fig:gen_workflow}.

\begin{figure}[H]
    \centering
    \includegraphics[trim=0.5cm 4.5cm 0.5cm 0.5cm,width=\columnwidth,scale=0.5]{thesis_workflow.pdf}
    \caption{General workflow of survey data processing in Python and R. \textcolor{red}{lisää "insert data back into records"}} 
    \label{fig:gen_workflow}
\end{figure}
\pagebreak

\subsection{Study area}
\justify
% https://www.hel.fi/hel2/Helsinginseutu/HS_tunnusluvut/liikennemaara_ja_autonomistus.pdf
% https://www.hsl.fi/sites/default/files/19_2016_auton_omistus_helsingin_seudulla.pdf
% https://www.hsl.fi/tutkimukset/muut-selvitykset
% http://pxnet2.stat.fi/PXWeb/pxweb/fi/StatFin/StatFin__lii__mkan/

\textcolor{red}{kappaleen pituus ei vielä lopullinen} The study area of this thesis is the Helsinki Capital Region. It comprises of municipalities of Helsinki, Espoo, Vantaa and Kauniainen. According to \textcolor{red}{LÄHDE}, the total population of the metropolitan area is 1.495 million. In practice the whole area amalgamates as one complete functional area with borders of the municipalities indistinguishable at the street level. The Helsinki Capital Region faces increasing pressure to manage its traffic because \textcolor{red}{LÄHDE}. Of these four municipalities Helsinki is the hub, and considered to contain the only inner city features of the municipalities (\textcolor{red}{syke-urbanareas}). Espoo, Vantaa and Kauniainen mostly consists of suburban areas with occasional industrial areas and large shopping complexes placed throughout the area. The Helsinki Capital Region is served with a high performance public transport system comprised of buses, train, subway, and tram in Helsinki. Recently, in 2017, the subway expanded from Helsinki to Espoo, triggering a new phase of quickly evolving cityscape in the surroundings of the new stations.

\begin{figure}[H]%
    \includegraphics[width=\textwidth]{images/thesis_resarea.png}
    \caption[Research area map]{Map of the Helsinki Capital Region, research area of this thesis. Specifically this research focuses on PAAVO postal code areas. The data has municipality value for each postal code area (see colouring of areas) but they do not completely align with the official municipality borders, as seen above. \textcolor{red}{osmcite}}%
    \label{fig:thesis_resarea}%
\end{figure}

Despite the extensive service level of Helsinki Capital Region public transport, households especially in Espoo, Vantaa, and Kauniainen remain dependent on their personal vehicles (\textcolor{red}{LÄHDE}). 

\subsection{Data}
\justify

The foundation of this research is the dataset \textit{PAAVO -- open data by postal code area} (abbreviated \textit{postal} in this thesis) (figure~\ref{fig:paavo_resarea}). For every postal code area in Finland, this data provides a large selection of data regarding the population. This includes demographics and data about employment by field. However, this thesis only utilises the spatial definitions of postal code areas, using these polygons in the web survey. This research makes use of the PAAVO 2018 dataset, released in January 2019. \textcolor{red}{discussioniin mahdollisuudet käyttää paavo-dataa laajemmin}

% https://land.copernicus.eu/news/product-update-2013-urban-atlas-2018-lc-lu %newslink ua, released Apr 17, 2020, not stated in the news article
In this thesis \textit{Urban Atlas Land Cover/Land Use 2012} (abbreviated \textit{Urban Atlas}, figure~\ref{fig:datalayers}) is used to locate forests in the Helsinki Capital Region (\cite{EuropeanEnvironmentAgency2016}). Provided by Copernicus Land Monitoring Service, which is coordinated by European Environment Agency, Urban Atlas consists of polygonal data about land cover and land use in populated areas which Urban Atlas calls Functional Urban Areas. For Finland, Urban Atlas data is available for Helsinki, Turku, Tampere, Lahti, Jyväskylä, Kuopio, and Oulu. In each dataset 27 different types of features are identified, forests being one of them.

In Urban Atlas, forests include broad leaved, coniferous, and mixed forest. Counted as forest are also transitional woodland and shrub and areas with over 30 \% tree canopy and tree height over five meters, including shrub at the edges of forests. Furthermore, plantations such as Christmas tree plantations and forest regeneration area are included. As an exception Urban Atlas does not consider forests within urban areas or subject to high human pressure to belong in this class. (\cite{CopernicusLandMonitoringService2016})

During the time of the writing of this thesis an updated version of Urban Atlas, Urban Atlas 2018 LCLU, was in the process of being launched. Unfortunately, the new dataset for Helsinki Functional Urban Area was not finished in time to be included in this work.

A main focus in this thesis was to compare the thesis survey results with \textit{Helsinki Region Travel Time Matrix 2018} (abbreviated \textit{TTM}), a dataset provided by Digital Geography Lab, a research group based in the University of Helsinki, the department of geosciences and geography (\cite{Tenkanen2018}). Their newest dataset provides travel times for public transport, private car, walking, and bicycling between all MetropAccess-YKR-grid cells (n=13231). All travel times were calculated using the door-to-door approach, which incorporates all parts of a journey from place A to place B into the travel time, be it walking from home door to the car or bus stop, or the time spent searching for parking. This thesis is only interested in journeys made by private car.

\textit{MetropAccess-YKR-grid} (abbreviated \textit{grid}, figure~\ref{fig:datalayers}) is a spatial dataset consisting of cells with the dimensions of 250 by 250 meters (\cite{Toivonen2014a}). The dataset is used in the MetropAccess project of Digital Geography Lab and is based on the YKR data provided by Finnish Environment Institute and Statistics Finland (\cite{StatisticsFinland2020}). \textit{Grid} is a simple dataset and contains the spatial coordinates of cells and their identifiers, called the YKR ID. Using the YKR ID it is easy to connect \textit{TTM} data with the statistical data provided by Statistics Finland, allowing wide-ranging possibilities for further research.

\textcolor{red}{huom, aineiston alkuperäiset yhdistelmäluokat poistettu/yhdistetty, muista mainita}
All postal code areas in the survey results would be classified with the \textit{zones of urban structures} (officially \textit{Yhdyskuntarakenteen vyöhykkeet}, abbreviated \textit{YKR zones}, figure~\ref{fig:datalayers}) (\cite{Ristimaki2017}). Utilising the same statistical grid of 250 x 250 meters as MetropAccess-YKR-grid, \textit{YKR zones} classifies grid cells to produce pedestrian, public transport, and automobile zones in and around Finland's urban regions using the theory of urban fabrics according to which these three zones have developed during different times in the urban region's history (\cite{Newman2016}). In this thesis every postal code area is assigned with a class defined in the \textit{YKR zones} based on which class has the largest presence. Adding this data into the survey results aimed to deepen the possibilities to explain the hypothetical dissimilarity of survey results in different parts of the Helsinki Capital Region. 

\textit{The regional division maps of Helsinki Capital Region} (officially \textit{Pääkaupunkiseudun aluejakokartat}, abbreviated \textit{subdivisions}, figure~\ref{fig:subdiv_placement}) was used in this thesis to analyse and visualise the survey results by subdivisions of Helsinki Capital Region (\cite{HelsinginEspoonVantaanjaKauniaistenmittausorganisaatiot2011}). Dividing the survey results into subdivisions would potentially give rise to local phenomena which would not be perceptible in the finest available level of resolution, the postal code areas.

The dataset \textit{Regional population density 2012} (figure~\ref{fig:subdiv_placement}) was used in this thesis to visualise borders of municipalities in Helsinki Capital Region (\cite{StatisticsFinland2012}). 

\begin{figure}[H]%
    \centering
    % use percentage of pagewidth with the syntax ".00\textwidth"
    \subfloat[Urban Atlas 2012 forests. \textcolor{red}{liian pieni selite, mieti kokoja}]{{\includegraphics[width=.5\textwidth]{images/thesis_data_ua_forest.png} }}%
    \subfloat[Zones of urban structure.]{{\includegraphics[width=.5\textwidth]{images/thesis_data_ykr_zones.png} }}%
    \quad
    \subfloat[MetropAccess-YKR-grid.]{{\includegraphics[width=\textwidth]{images/thesis_data_grid.png} }}%
    \caption[Spatial data]{Some of the spatial data used in the thesis.}%
    \label{fig:datalayers}%
\end{figure}

% hyphenrules: prevent hyphenation temporarily
\begin{hyphenrules}{nohyphenation}
    \begin{table}[H]
        \centering
        \caption[Thesis data]{Data utilised in the thesis.} 
        \label{tab:used_data}
        % use \scalebox{}{} to control table size. Note the additional curly brackets enveloping the tabular
        \scalebox{0.8}
        % use \arraystretch to add whitespace between rows. \setlength\tabcolsep for columns
        {\def\arraystretch{1.2} 
        \setlength\tabcolsep{1.2ex}
        % One unit here is ">{\raggedright\arraybackslash}p{4cm}". \raggedright prevents justification of text and conveniently allows flush right or flush left, which is not possible with column command p{4cm} alone.
        \begin{tabular}{ @{} >{\raggedright\arraybackslash}p{4cm} >{\raggedright\arraybackslash}p{4cm} >{\raggedright\arraybackslash}p{4.5cm} >{\raggedright\arraybackslash}p{3.5cm} >{\raggedleft\arraybackslash}p{3.5cm} @{} }
            \toprule
            Data & Description & Purpose in thesis & Abbreviation in thesis & Citation \\
            \midrule
            Paavo -- Open data by postal code area 2018 & Helsinki Capital Region postal code areas & Thesis research area & \textit{postal} & \cite{StatisticsFinland2019a} \\
            Urban Atlas LCLU 2012 & Land use and land cover data in vector format for functional urban area of Helsinki & Source of forest data & \textit{Urban Atlas} & \cite{EuropeanEnvironmentAgency2016} \\
            Zones of urban structure (Yhdyskuntarakenteen vyöhykkeet) 2017 & Delineation of urban areas based on the theory of urban fabrics & Source of data on spatial structure of urban areas & \textit{YKR zones} & \cite{Ristimaki2017} \\
            Helsinki Region Travel Time Matrix 2018 & Travel time and distance information for routes between all MetropAccess-YKR-grid cell centroids in the Capital Region of Helsinki & Use on travel time comparison calculations & \textit{Travel Time Matrix} & \cite{Tenkanen2018} \\
            MetropAccess-YKR-grid & Statistical grid of 250 x 250 meter cells for monitoring urban structure, Helsinki Capital Region area & Use on travel time comparison calculations & \textit{grid} & \cite{Toivonen2014a}, \cite{StatisticsFinland2020} \\
            Regional population density 2012 & Population density with municipality boundaries & Visualisation in R & \textit{hcr\_muns} & \cite{StatisticsFinland2012} \\
            Subdivisions in Helsinki Capital Region & Subdivisions in the municipalities of Helsinki Capital Region & Visualisation in R & \textit{subdivisions} & \cite{HelsinginEspoonVantaanjaKauniaistenmittausorganisaatiot2011} \\
            \bottomrule
        \end{tabular}}
    \end{table} 
\end{hyphenrules}

\subsection{Used software}
\justify

A wide variety of software was used in this thesis. The thesis strives for maximum openness and transparency and therefore much of the software employed in this work is free, open-source or both. The research survey application utilised several essential web technologies such as JavaScript, HTML, CSS and PHP. Using the web mapping library Leaflet, with the assistance of jQuery and other libraries, a modern and easy-to-use survey web application was created. Server-side, the programming language PHP was used to verify received records.

Data processing was carried out with Python 3.7.6 and R for Windows 3.6.3, with most of the processing done in Python and most of the analysis and visualisation in R. Much of the work was supported by libraries available for the programming languages. In Python Anaconda version 2020.02 -- a distribution for Python for statistical computing -- provides majority of the needed software libraries, with the exception of GeoPandas, a library for geospatial pandas DataFrames in Python. In R, many libraries are available for a comprehensive set of descriptive statistics. In addition to libraries helping with the analysis, libraries Shiny, ggplot, and ggiraph formed the basis of the visualisation of the survey results.

% latex is not a programming language for the most part: https://www.quora.com/Is-LaTeX-a-programming-language
The thesis was programmed using LaTeX and the online LaTeX editor Overleaf. LaTeX is a document preparation system, used to create documents such as scientific articles. LaTeX adheres to the WYSIWYM (what you see is what you mean) approach, as opposed to the "what you see is what you get" approach of text editors such as Microsoft Word, meaning that after establishing parameters LaTeX will compute document formatting, while the user can concentrate on the document content. While LaTeX can be considered a programming language, it is more closely related to HTML, a markup language (\textcolor{red}{cite}) In this LaTeX document, the LaTeX distribution TeX Live 2017 was used. Overleaf supports GitHub integration and as a result the complete thesis is available for viewing in the GitHub repository \textcolor{blue}{\url{https://github.com/sampoves/Masters-2020}} alongside with its entire development history. Additionally, the template of this thesis is provided at \textcolor{blue}{link}.

In addition to the aforementioned technologies, the flowcharts in this thesis were created with the web application diagrams.net (\textcolor{red}{cite}). Most of the map visualisations of this thesis were made using the geographic information system application QGIS version 3.12.2, while the many code repositories associated with this thesis are hosted in GitHub, a service which hosts Git software development version control systems.

% Consider! Removing \raggedright and hyphenrules will enable nice even table cells. Could be worth it to look into
\begin{hyphenrules}{nohyphenation}
    \begin{table}[H]
        \centering
        \caption[Thesis programming languages]{Programming languages and \glspl{ide} utilised in the thesis.} 
        \label{tab:used_langs}
        \def\arraystretch{1.3}
        \setlength\tabcolsep{1.2ex}
        \begin{tabular}{ @{} >{\raggedright\arraybackslash}p{3cm} >{\raggedright\arraybackslash}p{4.5cm} >{\raggedright\arraybackslash}p{3cm} >{\raggedleft\arraybackslash}p{3.5cm} @{} }
            \toprule
            Programming language and \gls{ide} & Description & Purpose in thesis & Citation \\
            \midrule
            HTML, CSS, JavaScript (NetBeans 8.2.0) & Web technologies which enable interactive web sites & Research survey programming  & \cite{WHATWG2020}, \cite{W3C2020}, \cite{ECMA2019}, \cite{ApacheSoftwareFoundation2016} \\
            Python 3.7.6, Anaconda 2020.02 (Spyder 4.0.1) & Anaconda is a Python distribution for scientific computing & Survey data processing & \cite{Python3Reference}, \cite{AnacondaInc.2020}, \cite{SpyderProjectContributors2020} \\
            R for Windows 3.6.3 (RStudio 1.2.5033) & Programming language environment for statistical computing & Survey data analysis and visualisation & \cite{RCoreTeam2020}, \cite{RStudioTeam2015} \\
            LaTeX (Overleaf) & LaTeX is a document preparation system. In this thesis, LaTeX distribution TeX Live 2017 is used. & Thesis format, structure, and writing & \textcolor{red}{latexcite, overleafcite} \\
            \bottomrule
        \end{tabular}
    \end{table} 
\end{hyphenrules}

\begin{hyphenrules}{nohyphenation}
    \begin{table}[H]
        \centering
        \caption[Essential software packages in thesis]{Essential software packages used in the thesis. The complete list is available at the script repositories in GitHub. \textcolor{red}{kuuluuko citee kaikki libraryt?}} 
        \label{tab:used_soft}
        \def\arraystretch{1.2}
        \setlength\tabcolsep{1.2ex}
        \begin{tabular}{ @{} >{\raggedright\arraybackslash}p{2.5cm} >{\raggedright\arraybackslash}p{3cm} >{\raggedright\arraybackslash}p{4cm} >{\raggedleft\arraybackslash}p{3cm} @{} }
            \toprule
            Programming language & Software package & Purpose in thesis & Citation \\
            \midrule
            JavaScript & Leaflet 1.4.0 & Web mapping library for the research survey & \cite{Agafonkin2019} \\
            Python & pandas 1.0.1 & Data analysis and manipulation & \cite{McKinney2011a} \\
                & GeoPandas 0.5.0 & Geographic data operations & \cite{GeoPandasDevelopers2019} \\
                & Shapely 1.6.4.post1 & Geometric objects, predicates, and operations & \cite{Gillies2019} \\
                & rtree 0.8.3 & Spatial indexing & \cite{Gillies2014} \\
            R & Shiny 1.4.0 & Web application framework for R & \cite{Chang2019} \\
                & ggplot2 3.3.0 & Data visualisation & \cite{Wickham2016} \\
                & ggiraph 0.7.0 & Interactive ggplot2 graphics & \cite{Gohel2019} \\
                & dygraphs 1.1.1.6 & Interactive time series charting & \cite{Vanderkam2018} \\
            \bottomrule
        \end{tabular}
    \end{table} 
\end{hyphenrules}

\subsection{Methods}
\subsubsection{Considering options for the survey}
\justify

To collect the areal parking data, the study required an interactive survey which respondents could use to submit their parking habits in a spatial fashion. To attract a largest possible number of submissions, the survey also needed to be of modern design, easy to use and its purpose easy to understand. The survey would have to be clear-cut, effortless to internalise and short in length as to prevent users getting frustrated and leaving before submitting answers. Design-wise, the spatial resolution of the survey was in question. The particular concern was that in the case of insufficient amount of answers, what kind of area delineation would be at the same time detailed enough but also streamlined enough to realistically reach the good quality results? This chapter strives to describe the process that would lead to the implemented survey of the thesis to accentuate the challenges this kind of research entail.

Once the consideration into options to produce the survey for this study had started, it quickly became apparent that there were few alternatives available and even fewer free, sufficiently customisable alternatives. Out of the proprietary options, Maptionnaire by the Finnish company Mapita was considered. They offer tailored map survey products with discounts for students. In return for the fee a subscriber receives a time window in which to carry out their survey accompanied with tailored features and customer support -- all according to the price plan. This price was considered too steep for the thesis and Maptionnaire was passed on. 

Next Survey123 for ArcGIS was evaluated. An Esri operated service, Survey123 is used to create and analyse form based surveys (\cite{Esri}). It is included in the contract between University of Helsinki and Esri and thus was free to use for the study. One can design a survey at the Survey123 website and share it immediately to respondents. Alternatively, the service is available as a desktop client in the form of Survey123 Connect, where Survey123 offers a range of possibilities for customisation with its adherence to the XLSForm standard. XLSForm is a standard to make authoring forms in Excel easier. With the customisability of XLSForm users can design Survey123 surveys to the dot while employing the support for Excel style scripting for complex survey behaviour (figure~\ref{fig:survey123_xlsform}). Furthermore, Survey123 provides online tools for collaboration, analysis and data viewing with many options for exporting collected data. Some months were used to perfect the parking survey with Survey123. 

\begin{figure}[H]%
    \includegraphics[width=\textwidth]{images/survey123_xlsform.png}
    \caption[Survey123 XLSForm view]{Survey123 XLSForm view in Microsoft Excel. Some parameter columns are hidden to provide a view to the essential inner workings of the Survey123 form.}%
    \label{fig:survey123_xlsform}%
\end{figure}

% \ref{}: tilde (~) indicates a non-breaking space
In January 2019 the parking survey developed with Survey123 was deployed to friends and family, with a large scale marketing push on social media platforms planned for later. At its core, this survey asked respondents for specific parking events in Helsinki Capital Region they had had (figure~\ref{fig:survey123}). Respondents would pick an exact location on a map view for the location of their parked car and separately on a second map view the location of their final destination. In addition, respondents would fill the date and time of this parking event, how long it took for them to find a parking spot, how often they had parked to that area, and what kind of a parking spot they had taken. Respondents were asked repeat this process as many times as they had the will to do so.

The Survey123 survey was designed to reach the same spatial resolution as Travel Time Matrix 2018 with its MetropAccess-YKR-grid (grid cell 250 x 250 meters). Using exact coordinates of parkings and final destinations it would have been possible to allocate each event to possibly two different MetropAccess-YKR-grid cell codes, reaching excellent spatial resolution. As MetropAccess-YKR-grid contains 13 231 grid cells, there was not enough resources for this master's thesis research survey to accumulate events for every grid cell, or even for most grid cells. If the data gathering campaign would have ended with insufficient amount of parking events the backup plan was to employ an interpolation algorithm to generate approximate boundaries for the hypothetically varying parking times in Helsinki Capital Region. It was also considered that the exact coordinates of the parking events could be generalised to other boundaries, such as administrative areas like municipality subdivisions or postal code areas. 

% utilises package subfig
\begin{figure}[H]%
    \centering
    \subfloat[Survey introduction and the date and time for the parking event.]{{\includegraphics[width=5.5cm]{survey123_1.png} }}%
    \subfloat[Map panels for the parking location and the final destination.]{{\includegraphics[width=5.5cm]{survey123_2.png} }}%
    \subfloat[Final questions of the survey.]{{\includegraphics[width=5.5cm]{survey123_3.png} }}%
    \caption[The unused parking survey created with Survey123]{An example parking event entered into the unused parking research survey made with Survey123 Connect.}%
    \label{fig:survey123}%
\end{figure}

The survey was released even after Survey123 had proved itself unwieldly for the purposes of this research. The software was difficult to use because of an assortment of inconvenient design choices, unfinished functionality and a helping of bugs. It was not possible, for example, to have respondents enter multiple parking events at once in a full screen map view. They would have to reload the survey, something a majority of people would not do. Survey123 Connect version available at the time, version 3.1.126, did not allow customisation of the post-submission message and therefore it would not be possible to efficiently direct respondents back to the form. In addition, recording coordinates from two map views was only possible through a bypass. The coordinates of the final destination would have to be printed on the form (hence the section "Coordinates" on the form in figure~\ref{fig:survey123}) and then these second set of coordinates could be saved into the survey data table in string format. The technical limitations of Survey123 as a spatial survey was witnessed also in the fact that it was not possible to add custom polygons on top of the map views. It was therefore impossible to delineate the research area for the respondents and accurately detect attempts to add parking events outside of Helsinki Capital Region.

The functionality of the released form was not reliable on the most popular web browsers such as Google Chrome and Apple Safari. Survey123 supported multi-language strings but it proved problematic to ensure that the form would open in the system language of most respondents, Finnish. In addition to this, the field for entering the specific time of parking was restricted to the US preferred 12 hour clock -- a system Finnish nationals would frown upon in the survey. To make matters worse, at that time there was a long persisting bug in Survey123 which produced unexpected behaviour, in some cases, with the use of "constraint", the parameter that controls which entered values are deemed illegal and which are not (\cite{GeoNet-TheEsriCommunity2018}). If any type of constraint statement was added, the finalised form would always claim that the related question input was invalid. The parameter would have to be left empty and therefore it was not possible to automatically prevent insertion of parking events happening in the future and excessively long times for searching for parking, reducing the quality of the survey data and making the survey form more confusing for the respondent.

\begin{figure}[H]%
    \includegraphics[width=\textwidth]{images/survey123_dataview.png}
    \caption[Survey123 data tab]{Survey123 for ArcGIS, with open "data" tab. The research survey made with Survey123 received in total 104 parking events. The red dots are the final destinations of each parking event.}%
    \label{fig:survey123_dataview}%
\end{figure}

Despite the many technical incertainties of Survey123, the survey gathered more than one hundred parking events in one month (figure~\ref{fig:survey123_dataview}). This amount was achieved for the most part without advertising. Soon after the publication of the Survey123 form it was decided, however, that the required spatial resolution for this research would need to be lower than exact points in an attempt to gather more responses from the entire research area. An additional deciding factor was the fact that with Survey123 respondents could not send multiple parking events with one survey session, making the form unwieldy and outdated in its rigid structured format. It was argued that a more general scale would still be accurate enough to provide good data and a more generalised scale would make the survey easier to answer to and a more pleasant experience for the respondent. Postal code areas were deemed an acceptable compromise in spatial accuracy.

After careful consideration it was decided that the survey for this thesis would have to be programmed from the ground up.

\subsubsection{Programming the parking survey}
\justify
To achieve maximum transparency and repeatability for this research, in addition to freedom in survey content and appearance, a survey web application was programmed from the ground up utilising HTML, JavaScript and PHP. The survey and its supporting infrastructure was installed on a virtual machine in CSC's -- the state owned ICT solutions company -- Taito supercluster. CSC offers virtual machines in several different hardware configurations, or flavors. The virtual machine flavor picked for this survey was \textit{standard.medium}, a flavor with 3.9 \gls{GB} \gls{RAM}, three virtual \gls{CPU}s and 80 GB of disk space. Running on the Linux distribution Ubuntu version 16.04, the backbone of the survey ecosystem was a LAMP stack, a software bundle which incorporates the Linux operating system, Apache web server software, \gls{mysql} relational database management system and the PHP programming language environment for server-side scripting. The public component of the survey is the front-end, the only component of the survey system a respondent would interact with (figure~\ref{fig:js_survey_welcome}). One may use additional software in a LAMP stack for extended functionality or can replace some of the components with a wide array of alternatives. This thesis utilises the components described in the table~\ref{tab:survey_components}.

\begin{hyphenrules}{nohyphenation}
    \begin{table}[H]
        \centering
        \def\arraystretch{1.2}
        \setlength\tabcolsep{1.2ex}
        \caption{Survey web application components} 
        \label{tab:survey_components}
        \begin{tabular}{ @{} >{\raggedright\arraybackslash}p{3cm} >{\raggedright\arraybackslash}p{3cm} >{\raggedright\arraybackslash}p{5.5cm} @{} }
            \toprule
            Component & Version & Description \\
            \midrule
            Ubuntu & 16.04.6 & Linux distribution, operating system for the virtual machine \\
            Apache HTTP Server & 2.4.18-2ubuntu3.9 & Web server software, manage website requests and responses \\
            MySQL & 5.7.25-0ubuntu0.16.04.2 & Relational database management system, survey database operations \\
            PHP & 7.0.33-0ubuntu0.16.04.1 & Programming language, used for server side scripting \\
            Parking survey front-end & 16.5.2019 & Survey visible to user, graphical user interface \\        
            \bottomrule
        \end{tabular}
    \end{table}
\end{hyphenrules}

Setting up the virtual machine for the use of the survey was a process of few stages. The LAMP stack was installed on the fresh virtual machine with the command \code{sudo apt install lamp-server\^}. \textcolor{red}{kuuluuko koodi, tällaiset komennot, graduun?} After the successful installation the MySQL tables were formed and relevant users created. The last step before a fully functioning web server was using root access to give the survey components permission to access relevant system directories. Please see the GitHub repository (\textcolor{blue}{\url{https://github.com/sampoves/parking-in-helsinki-region}}) for the full step-by-step install procedure used to set up the web server for this thesis.

\begin{figure}[H]%
    \includegraphics[width=\textwidth]{images/js_survey_welcome.png}
    \caption[Survey landing page]{This was the screen respondents would first see when arriving to the parking survey.}%
    \label{fig:js_survey_welcome}%
\end{figure}

The survey front-end was programmed in NetBeans \gls{ide} 8.2 in mostly JavaScript using an open-source mapping library Leaflet (software version 1.4.0) in January--May 2019. In the survey the respondent was presented with a map view of Helsinki Capital Region with its 167 postal code areas with the ability to drag the view, zoom in and out, search for places and addresses, choose the language between English and Finnish, and tweak various settings to their liking. In this setting, the respondent was asked to pick as many postal code areas as they could remember parking in in the last two years, and answer to five questions per postal code area (table~\ref{tab:js_survey_questions} and figure~\ref{fig:js_survey_questions}). In each question the respondent was asked to estimate their parking experience in that postal code area usually during the past two years. The last two years was chosen as the timeframe to allow respondents to comfortably recall parking events which happened during the subjective notion of "recent memory" while also forbidding the submission of out of date parking times. 

\textcolor{red}{lisää kappale, jossa selitän kysymysten sisällön auki} parktime, walktime: The maximum values in these fields were consciously placed to 99 in an effort for the range to not feel restrictive for user.

In the introduction to the survey it was explained to respondents that all answers were meant to be estimates as the survey was not about an exact time and place. To mitigate confusion and errors made by respondents a comprehensive help functionality and a location search tool were implemented in the parking survey. Once the respondent was finished with the survey, they would send their responses to the server. Respondents were welcomed to return to the survey to send additional data on any postal code areas they had missed the last time.

\begin{hyphenrules}{nohyphenation}
    \begin{table}[H]
        \centering
        \caption{Survey questions and question choices.} 
        \label{tab:js_survey_questions}
        \def\arraystretch{1.5}
        \setlength\tabcolsep{1.2ex}
        \begin{tabular}{ @{} >{\raggedright\arraybackslash}p{5.5cm} >{\raggedright\arraybackslash}p{5cm} >{\raggedright\arraybackslash}p{2.5cm} >{\raggedright\arraybackslash}p{2cm} @{} }
            \toprule
            Question & Question choices & Question type & Abbreviation \\
            \midrule
            How long does it usually take for you to find a parking spot and park your car in this postal code area (in minutes)? & 0--99 & Field, selection within range & parktime \\
            How long does it usually take for you to walk from your parking spot to your destination in this postal code area (in minutes)? & 0--99 & Field, selection within range & walktime \\
            How familiar are you with this postal code area? & 1 -- Extremely familiar\linebreak2 -- Moderately familiar\linebreak3 -- Somewhat familiar\linebreak4 -- Slightly familiar\linebreak5 -- Not at all familiar & Radio button group, likert-type scale & likert \\
            What kind of parking spot do you usually take in this postal code area? & 1 -- Parking space on the side of the street\linebreak2 -- Parking lot\linebreak3 -- Parking garage\linebreak4 -- Private or reserved spot\linebreak5 -- Other & Dropdown, selection & parkspot \\
            At what time of the day do you usually park in this postal code area? & 1 -- Weekday, rush hour (07.00--09.00 and 15.00--17.00)\linebreak2 -- Weekday, other than rush hour\linebreak3 -- Weekend\linebreak4 -- None of the above, no usual time & Dropdown, selection & timeofday \\
            \bottomrule
        \end{tabular}
    \end{table} 
\end{hyphenrules}

\begin{figure}[H]%
    \centering
    \subfloat[Survey questions in English.]{{\includegraphics[width=6.25cm]{js_survey_en.png} }}%
    \qquad
    \subfloat[Survey questions in Finnish.]{{\includegraphics[width=6.25cm]{js_survey_fi.png} }}%
    \caption[Research survey questions in the web application]{For each postal code area of their choosing the respondent would answer to these five questions. The survey was made available in English and Finnish.}%
    \label{fig:js_survey_questions}%
\end{figure}

When data was received from the respondent, a script written in \gls{php} verified the data contents. This was an effort to prevent attacks on the web server running the study survey. Only specific variables of specific types were accepted from the front-end. Additionally, the \gls{php} verification made sure falsified or incomplete data would not be accepted into the database containing the verified results. If the server-side verification test failed in any way, the respondent was informed about it. 

In addition to the data verification, a PHP script tracked the IP addresses which accessed the survey web server. By using the survey, respondents agreed that their IP addresses were recorded for the use of this thesis solely to identify falsified or overlapping data and detect unique visits. All IP addresses were anonymised with a Python script and original sensitive data deleted. The anonymisation script is available for viewing at the survey repository at GitHub.

As a final survey component the server side contained two MySQL datatables, one for received data (Table~\ref{tab:mysql_records}) and another for survey web page hits (Table~\ref{tab:mysql_visitors}). In the table \textit{records}, the following data was recorded: time of sending (column name \code{timestamp}), IP address (\code{ip}), postal area code (\code{zipcode}), a value in the sequence 1--5 for the likert question (\code{likert}), a value in the sequence 1--5 for the question what type of parking spot was used (\code{parkspot}), an integer value for how long it usually took to park in this location (\code{parktime}), an integer value for how long it usually took to walk from parking place to one's destination (\code{walktime}), and a value in the sequence 1--4 for the question at what time of the day one usually parks in the location (\code{timeofday}). In the table \textit{records} it is notable that in the case an respondent sent the web server data for multiple postal code areas each of the postal code areas would take up their own row in the data table. Consequently, it was theoretically possible for one respondent to simultaneously submit 167 rows of data.

In the table \textit{visitors} the following data was recorded: IP address (\code{ip}), the timestamp of the first visit of this IP address (\code{ts\_first}), the timestamp of latest visit of this IP address (\code{ts\_latest}), and the count of visits (\code{count}). In this table an IP address is only stored once. On the first visit of an IP address the row for that IP address is created in the data table with \code{ts\_first} and \code{ts\_latest} being identical. On further visits of that IP address the original row is appended with updated information in the columns \code{ts\_latest} and \code{count}.

\begin{hyphenrules}{nohyphenation}
    \begin{table}[H]
        \centering
        \setlength\tabcolsep{1.2ex}
        \caption[Structure of MySQL table records]{The structure of the survey MySQL table \textit{records} fetched with the statement \code{DESCRIBE records;}} 
        \label{tab:mysql_records_str}
        \begin{tabular}{ @{} >{\raggedright\arraybackslash}p{2cm} >{\raggedright\arraybackslash}p{2cm} >{\raggedright\arraybackslash}p{1cm} >{\raggedright\arraybackslash}p{1cm} >{\raggedright\arraybackslash}p{1.5cm} >{\raggedleft\arraybackslash}p{4cm} @{} }
            \toprule
            Field & Type & Null & Key & Default & Extra \\
            \midrule
            id & int(11) & No & PRI & NULL & AUTO\_INCREMENT \\
            timestamp & varchar(19) & Yes & & NULL & \\
            ip & TEXT & Yes & & NULL & \\
            zipcode & varchar(5) & Yes & & NULL & \\
            likert & int(1) & Yes & & NULL & \\
            parkspot & int(1) & Yes & & NULL & \\
            parktime & int(2) & Yes & & NULL & \\
            walktime & int(2) & Yes & & NULL & \\
            timeofday & int(1) & Yes & & NULL & \\
            \bottomrule
        \end{tabular}
    \end{table} 
\end{hyphenrules}

\begin{hyphenrules}{nohyphenation}
    \begin{table}[H]
        \centering
        \setlength\tabcolsep{1.2ex}
        \caption[Structure of MySQL table visitors]{The structure of the survey MySQL table \textit{visitors} fetched with the statement \code{DESCRIBE visitors;}} 
        \label{tab:mysql_visitors_str}
        \begin{tabular}{ @{} >{\raggedright\arraybackslash}p{2cm} >{\raggedright\arraybackslash}p{2cm} >{\raggedright\arraybackslash}p{1cm} >{\raggedright\arraybackslash}p{1cm} >{\raggedright\arraybackslash}p{1.5cm} >{\raggedleft\arraybackslash}p{4cm} @{} }
            \toprule
            Field & Type & Null & Key & Default & Extra \\
            \midrule
            id & int(11) & No & PRI & NULL & AUTO\_INCREMENT \\
            ip & TEXT & Yes & & NULL & \\
            ts\_first & DATETIME & Yes & & NULL & \\
            ts\_latest & DATETIME & Yes & & NULL & \\
            count & int(11) & Yes & & NULL & \\        
            \bottomrule
        \end{tabular}
    \end{table} 
\end{hyphenrules}

The parking survey was released to the public in May 2019 and the active phase of collecting data continued until 30th June 2019. However, the survey remained open after the active period, receiving the last row of data in October 2019. The majority of the respondents were found through Facebook. Invitations to participate in the survey were sent to 112 city district and neighborhood groups with a theoretical reach of tens of thousands of people. Of the 112 posts, 63 were Helsinki centric groups, while 22 were from Espoo, 15 from Vantaa, and 12 from municipalities bordering Helsinki Capital Region. In addition to these city district and municipal groups invitation to participate was sent to two other Facebook groups, "Lisää kaupunki Helsinkiin", a group for city planning ethusiasts in Helsinki, and the GIS profession group "GIS-velhot". It is not possible to conclusively differentiate from which group or city survey data originated from. A clue about the survey's popularity in each city, however, may be gained from the table \textit{visitors} due to the fact that invitation posts were sent over multiple days to the groups in roughly the order Espoo--Helsinki--Vantaa--bordering municipalities--reminders to all groups. In addition to Facebook, an effort was also made to get faculty members of geosciences and geography and students of University of Helsinki to participate in the survey. A small amount of answers were collected with a tweet sent from the Twitter account of Digital Geography Lab. After the initial invitation to participate, reminders were sent to the largest Facebook groups one month after the original posts.

The source code for the survey described in this chapter and step-by-step information to set up an identical system is available at GitHub (\textcolor{blue}{\url{https://github.com/sampoves/parking-in-helsinki-region}}). As a side product, a variant of this survey was created where respondents pick precise points instead of areas. This point-based survey template is, too, available at GitHub (\textcolor{blue}{\url{https://github.com/sampoves/leaflet-map-survey-point}}).

% \scalebox to prevent table going too wide
\begin{hyphenrules}{nohyphenation}
    \begin{table}[H]
        \centering
        \setlength\tabcolsep{2pt}
        \caption[MySQL table records]{The data content of the survey MySQL table \textit{records}.} 
        \label{tab:mysql_records}
        \scalebox{0.9}
        {\begin{tabular}{ @{} >{\raggedright\arraybackslash}p{1.5cm} >{\raggedright\arraybackslash}p{4cm} >{\raggedright\arraybackslash}p{2.5cm} >{\raggedright\arraybackslash}p{2cm} >{\raggedright\arraybackslash}p{1.5cm} >{\raggedright\arraybackslash}p{1.5cm} >{\raggedright\arraybackslash}p{1.5cm} >{\raggedright\arraybackslash}p{1.5cm} >{\raggedright\arraybackslash}p{1.5cm} @{} }
            \toprule
            id & timestamp & ip & zipcode & likert & parkspot & parktime & walktime & timeofday \\
            \midrule
            3245 & 2019-06-06 21:41:21 & wro4qo8hv4 & 00510 & 1 & 4 & 0 & 3 & 1 \\
            3246 & 2019-06-06 21:41:54 & aonm72lyx3 & 00520 & 2 & 1 & 10 & 5 & 1 \\
            3247 & 2019-06-06 21:46:19 & n1982i4i2v & 00100 & 1 & 1 & 20 & 4 & 1 \\
            3248 & 2019-06-06 21:46:22 & sbhfz0uvsl & 00210 & 1 & 1 & 5 & 3 & 3 \\
            3249 & 2019-06-06 21:46:22 & sbhfz0uvsl & 00220 & 2 & 2 & 5 & 5 & 2 \\        
            \bottomrule
        \end{tabular}}
    \end{table} 
\end{hyphenrules}

\begin{hyphenrules}{nohyphenation}
    \begin{table}[H]
        \centering
        \setlength\tabcolsep{1pt}
        \caption[MySQL table visitors]{The data content of the survey MySQL table \textit{visitors}.} 
        \label{tab:mysql_visitors}
        \begin{tabular}{ @{} >{\raggedright\arraybackslash}p{2cm} >{\raggedright\arraybackslash}p{3cm} >{\raggedright\arraybackslash}p{4cm} >{\raggedright\arraybackslash}p{4cm} >{\raggedleft\arraybackslash}p{1cm} @{} }
            \toprule
            id & ip & ts\_first & ts\_latest & count \\
            \midrule
            1780 & mvovd467a7 & 2019-05-26 15:25:23 & 2019-05-26 15:26:06 & 2 \\
            1781 & xgbgkkzxb3 & 2019-05-26 15:26:23 & 2019-05-26 15:26:23 & 1 \\
            1782 & c9qer4q99a & 2019-05-26 15:27:25 & 2019-05-26 15:27:25 & 1 \\
            1783 & cujhd0hng7 & 2019-05-26 15:27:29 & 2019-05-26 15:27:29 & 1 \\
            1784 & 3ja7gjtko6 & 2019-05-26 15:28:45 & 2019-05-26 15:29:20 & 2 \\        
            \bottomrule
        \end{tabular}
    \end{table} 
\end{hyphenrules}

\begin{figure}[H]%
    \centering
    \subfloat[Respondent arrives to the survey web application to see a map with the postal code areas of Helsinki Capital Region lined out.]{{\includegraphics[width=7cm]{js_survey_process1.png} }}%
    \quad
    \subfloat[Respodent proceeds to fill out their parking experiences in freely chosen postal code areas.]{{\includegraphics[width=7cm]{js_survey_process2.png} }}%
    \quad
    \subfloat[\textit{Submit records} button activates when all questions in all selected postal code areas are completed.]{{\includegraphics[width=7cm]{js_survey_process3.png} }}%
    \quad
    \subfloat[Respondent receives a prompt to confirm that their submission was successful.]{{\includegraphics[width=7cm]{js_survey_process4.png} }}%
    \caption[Steps to fill out the survey]{A respondent would follow these steps to submit data through the survey web application.}%
    \label{fig:survey_process}%
\end{figure}

\subsection{Processing survey data}
\label{sec:processdata} % labeling to enable hyperref to this chapter
\justify
%\begin{itemize} %processingin tärkeimmät kohdat
%    \item anonymisation of ip addresses
%    \item Read in spatial data sources
%    \item Read in survey data
%    \item Prepare source data (convert formats, remove some irregular erroneous answers from dataset)
%    \item Prepare shape files (remove islands not reachable by car)
%    \item give grid cells zipcodes (ykr grid does not have those of-the-shelf. Develop method to assign all cells zipcodes, take into account water and grid cells which are outside of research area)
%    \item respondent behaviour (see how each user has answered)
%    \item detect illegal data (first detect duplicate answers, produce report. Then remove data where parktime and/or walktime is 60 or over)
%    \item Add data to geodataframes (add columns for ykr\_vyoh, ua-forest, answer count, parktime and walktime mean
%    \item show statistics to user
%    \item Set percentage of urban zones and forest in each zipcode area (choose one urban zone and forest amount (jenks breaks) for every zipcode)
%    \item add subdivisions to data (all answer row gets corresponding subdivision value)
%    \item EXPERIMENTAL utilise travel-time matrix 2018, make comparisons
%    \item EXPERIMENTAL somehow create my own TTM18, with updated values
%    \item export results to R
%\end{itemize}

In this section various data are refered to with abbreviated names as this makes it easier to follow the data processing workflow. Please see table~\ref{tab:used_data} for the key. In this section, the dataframe containing survey responses is called \textit{records} and survey visits \textit{visitors}. \textcolor{red}{kaikki kohdat tässä kappaleessa ei käytä lyhenteitä (records, visitors, postal) vielä. en oo päättänyt yksityiskohtia vielä tähän liittyen}

The main objective of the thesis data processing was to merge \textit{records} with selected spatial data and prepare \textit{records},  \textit{visitors}, \textit{postal}, and \textit{grid} for later analysis in R programming language environment. Using a selection of open spatial data (table~\ref{tab:used_data}), new explanatory variables would be available for analysis. This opened opportunities to compare the newly gathered survey data against that in Helsinki Travel-time Matrix 2018. In this phase, all data processing was carried out in Python programming language version 3.7.6, using Anaconda, a free and open-source Python distribution for scientific programming. Anaconda version 2020.02 included all essential packages for carrying out the script, except for GeoPandas 0.5.0, the package for geospatial data manipulation in Python. GeoPandas and its dependencies -- GDAL 2.4.1, Fiona 1.8.6, pyproj 2.1.3, rtree 0.8.3, and Shapely 1.6.4.post1 -- were manually installed through Python package installer pip.

As the first step in the survey data processing, all IP addresses were anonymised and replaced with identifiers of ten characters consisting of numbers 0--9 and letters of English alphabet (figure~\ref{fig:gen_workflow}, section 1). The anonymisation was carried out in such a way that the random identifiers for respondents matched in both \textit{records} and \textit{visitors}, preserving the possibility to associate survey responses with survey visits. 

\begin{figure}[H]%
    \centering
    \subfloat[Unedited PAAVO postal code areas for Helsinki Capital Region.]{{\includegraphics[width=8cm]{resarea_unedited.png} }}%
    \quad
    \subfloat[PAAVO postal code areas for Helsinki Capital Region, islands unreachable by car removed.]{{\includegraphics[width=8cm]{resarea_edited.png} }}%
    \caption[Process to remove islands not reachable by car]{Islands unreachable by car were removed from the postal code area data in the Python data processing.}%
    \label{fig:paavo_resarea}%
\end{figure}

The data processing proper started with loading the open spatial data presented in table~\ref{tab:used_data} and selecting only areas relevant to the research (figure~\ref{fig:gen_workflow}, section 2). For \textit{Urban Atlas 2012} this meant selecting only areas marked 31000 Forests.\textit{YKR zones}, a dataset that covers the entirety of Finland, was clipped with spatial dimensions of PAAVO postal code areas data, \textit{postal}, that had been extended with a 500 meter buffer. \textit{postal} was processed to only include areas reachable by car from the mainland (figure~\ref{fig:paavo_resarea}). Islands not reachable by car were approximated visually using Google Maps and were removed from the data. However, some islands in Helsinki Capital Region are technically accessible with a car from the mainland, but in practice the access is limited. In these cases deliberation was used. For example, Suomenlinna islands and Korkeasaari were kept in the data. Conversely, some technically car-accessible islands like Staffan in Espoo, and Mustasaari and Seurasaari in Helsinki were removed from the data with the grounds of them containing only private property, or no public parking spaces. \textcolor{red}{selitä laajemmin miksi näitä saaria poistellaan (analyysi ja visualisointi). lisäksi, mieti toi logiikka "yksityisalue tai ei yleisiä parkkiksia"}

\begin{figure}[H]%
    \includegraphics[width=\textwidth]{images/paavo-ykr.png}
    \caption[Assigning MetropAccess-YKR-grid postal codes]{The MetropAccess-YKR-grid cell 6002625 (marked with the red square) is assigned postal code 02360 because in that grid cell, the largest segment of PAAVO open data (coloured warm purple and yellow) belongs in the postal code 02360 Soukka. \textcolor{red}{osm cite}}%
    \label{fig:paavo_ykr}%
\end{figure}

Helsinki Travel-time Matrix 2018 and the survey data of this thesis operate in different spatial units. Travel-time Matrix 2018 uses the MetropAccess-YKR-grid (\textit{grid}), a spatial dataset based on the Statistics Finland statistical grid with the cell size of 250 x 250 meters. The basic spatial unit of the survey data is the postal code area based on PAAVO open data (\textcolor{red}{lisää lähteet}). Using Python, postal codes were added to each \textit{grid} cell with the logic that the largest area in \textit{postal} (figure~\ref{fig:paavo_ykr}) assigns the postal code in each \textit{grid} cell. \textit{postal} polygons do not always intersect with the cells of \textit{grid} and because of this some cells were assigned a postal code of 99999 to denote missing data (\textcolor{red}{mieti vielä 99999:n käyttö}). As a side product of this postal code assignment, \textit{grid} was merged with data which tells how much of a cell was contained in the research area (\textit{postal}) and how large was the largest postal code area which dictated the postal code assignment of the current cell. \textcolor{red}{varmista että lukija ymmärtää PAAVO spatial datan ja tutkimusalueen yhteyden}

The data processing script created for this thesis contains detailed features to detect patterns in the survey data (figure~\ref{fig:gen_workflow}, section 3). To enhance pattern recognition \textit{records} and \textit{visitors} were purged of known false data, namely responses and visits made by the author. In addition, one survey respondent reported that they had sent erroneous data to the survey. These responses were identified and deleted.

The data processing script creates two distinct reports about \textit{records}. Firstly, the data processing script aggregates \textit{records} by IP address code, resulting in an Excel file where one row represents each respondent. It is then possible to review the behaviour of each respondent, in detail. In addition to this report, the data processing script writes a text file report about IP address codes which submitted multiple responses from the same postal code area. The text file report also identifies whether the duplicate responses for each postal code area per IP address code have identical values or if they have changed between responses. These two reports were used to determine what to do about the duplicates and values which appear anomalous.

It was decided that if the parking time or walking time value in a \textit{records} row was 60 minutes or greater, that data row would be deleted. This value is arbitrary. The research assumes that it is highly unlikely that anybody would generally park 60 minutes away from their final destination which they would reach on foot. A hour of searching for parking is plausible in the center of Helsinki but because it its unlikeliness the same 60 minutes limit was utilised in parktime. It is not possible to determine the reasoning behind the answers which in multiple cases reach the maximum value 99, but it can't be ruled out that these data rows are protest votes meant to declare that reliable parking is hard to find in certain parts of Helsinki Capital Region. In conclusion, even though the Python script has the capability to delete data rows deemed illegal, all of the illegal data in \textit{records} was preserved for later use in R.

Next in the survey data processing workflow the additional spatial data was added to \textit{postal}, the dataset with one row for each postal code area (figure~\ref{fig:gen_workflow}, section 4). Utilising \textit{records}, functions \code{sum}, \code{mean}, and \code{median} were used to produce answer count, and means and medians for parktime and walktime for all postal code areas. Each postal code area also received seven columns to depict the share of \textit{YKR zones} classes in percentage. Using \textit{Urban Atlas 2012} data, the percentage of forest in each postal code area was calculated. 

In the finalising section \textit{records} was prepared for analysis and visualisation in R (figure~\ref{fig:gen_workflow}, section 5).  The software library for plotting in R, \textit{ggplot2}, prefers data inputted in long format. To study charasteristics of postal code areas in this research it meant adding repetitive data columns in \textit{records}, where values for Urban Atlas 2012 forest, YKR zone and subdivision remained unchanged for all rows in the same postal code area. For Urban Atlas 2012 forest, a custom Jenks breaks function (GitHub user Drewda, \textcolor{red}{add cite}) with five classes were utilised to find the applicable Jenks breaks class for each postal code area. For YKR zones, the most common urban structure type in percentage was selected for each postal code area. In addition, \textit{records} was inserted with municipality subdivision information (figure~\ref{fig:subdiv_placement}). This was achieved by collecting data from the web sites of the municipalities of Helsinki Capital Region (\cite{Espoonkaupunki2020}, \cite{Helsinginkaupunkiymparistontoimiala2019}, \cite{Vantaankaupunki2019}). In these sources, each municipality broke the subdivisions down to city district level, from where it was possible to allot each postal code area with a subdivision. This was for the most part simplistic work, but in some cases the postal code areas and city districts did not align and author's own deliberation was used to help the placement. Some of the most glaring discrepancies between PAAVO postal code areas and subdivision borders occur in Espoo. In the case of Lippajärvi-Järvenperä, a postal code area north of Kaunianen, the subdivision Vanha-Espoo was chosen because Lippajärvi-Järvenperä as a whole does not fit into the charasteristics of Suur-Leppävaara, and at the same time the city districts Lippajärvi and Järvenperä do not fit into the distinctive features of the subdivision Pohjois-Espoo. In the same spirit the postal code area Sepänkylä-Kuurinniitty south of Kauniainen lies troublingly in the area of four subdivisions of Espoo. In the end Vanha-Espoo was chosen as Sepänkylä-Kuurinniitty lies for the most part in its area. Similar complications occurred in Helsinki and Vantaa (the partial placement of Kirkonkylä-Veromäki and Ruskeasanta-Ilola in subdivision of Tikkurila) and using author's best judgement, the classification shown in figure~\ref{fig:subdiv_placement} was used in the survey results analysis of this thesis.

Finally, a tool to compare the thesis research survey data and that in Travel-time Matrix 2018 was developed. In this tool it was possible to insert any number of origin points and destination points in \textit{grid} id codes, and receive a detailed comparison of the two data sources. \textcolor{red}{jatka tätä kappaletta}

The source code for the data processing described in this chapter is available at GitHub (\textcolor{blue}{\url{https://github.com/sampoves/Msc-thesis-data-analysis}}).

\begin{figure}[H]%
    \includegraphics[width=\textwidth]{images/thesis_subdiv_place.png}
    \caption[Placing postal code areas in subdivisions]{For the purposes of analysis in R, all postal code areas in \textit{postal} were assigned with subdivision information. In this figure, distinct colors depict the postal code areas with the subdivision classification chosen for this thesis.}%
    \label{fig:subdiv_placement}%
\end{figure}

\subsection{Conducting analyses}
\justify
% - Mostly R, but remember travelTimeComparison() in Pyytton
% - Compare a few different travel time chains in Helsinki Capital Region. A few starting points and a few finishing points
% - All of the statistics stuff to detect the variation 

%\begin{itemize}
%    \item Prepare data to R compliant format
%    \item ShinyApp descriptive statistics
%    \item shinyapp histogram for parktime and walktime
%    \item shinyapp boxplot, show outliers
%    \item shinyapp barplot, show amounts
%    \item shinyapp levene test
%    \item shinyapp one-way anova
%    \item shinyapp map, nice to have, not at all important
%    \item visitor shinyapp, see the accumulation of visits and received records
%\end{itemize}

\textcolor{red}{en ole vielä puhunut travelTimeComparison()ista}
Once the data processing in Python was completed, \textit{records} and \textit{visitors} were carried over to R for its easy to access statistical analysis functionality (namely packages \textit{onewaytests} for ANOVA and Brown-Forsythe test, \textit{plotrix} for standard error, and \textit{moments} for quantiles) \textcolor{red}{citations}. To help study the large datasets, two Shiny applications were written, one for \textit{records} and a second for \textit{visitors}. Benefits of this decision were twofold. Firstly, approaching the survey results from an interactive perspective allowed countless combinations of active and inactive variables -- without constant tweaking of code -- which would be beneficial for the analysis of \textit{records}. Secondly, Shiny applications can be deployed to the internet using the website shinyapps.io. Combination of these two factors made it effortless to analyse results of the survey in a visual way and at the same time, publish the results and used tools to the public, upholding the thesis' mission of openness and transparency.

\begin{figure}[H]%
    \centering
    \subfloat[A segment of the shinyapps.io deployment of \textit{records}. The web application provides a wide array of analysis and visualisation tools for the results of the thesis survey research.]{{\includegraphics[width=8.1cm]{images/shinyapps_analysis.png} }}%
    \quad
    \subfloat[The shinyapps.io deployment of \textit{visitors}. In this web application users may examine how the amounts of submitted responses and unique first visits of the survey web application developed over time.]{{\includegraphics[width=8.1cm]{images/shinyapps_visitors.png} }}%
    \caption[Survey results as shinyapps.io web applications]{shinyapps.io deployments of the two survey datasets.}%
    \label{fig:shinyapps}%
\end{figure}

In the Shiny application for \textit{records} users can view the survey responses from many different angles. Users are given control which variables are active at any moment (\hyperref[fig:shinyapps]{figure~\ref{fig:shinyapps}a}). Users control the variables through a side panel, with effects taking place in the larger main panel. The variables currently viewed are selected through two dropdown menus, Response (continuous) and Explanatory (ordinal). Continuous variables are \code{parktime} and \code{walktime} with their scale from 0--99. Ordinal variables include \code{likert}, \code{parkspot}, \code{timeofday}, \code{ua\_forest}, \code{ykr\_zone}, and \code{subdiv} with the values that can't be unequivocally ordered in a sequence in the same way as continuous variables. One variable from each variable group can be selected at the same time. Any and all groups of values in the ordinal variables can be deactivated to better understand the significance of each value. In addition to the selection of the continuous and ordinal variable, users can deactivate \textit{records} data rows based on their spatial location in municipality subdivisions assigned in \hyperref[sec:processdata]{\fullref{sec:processdata}}. Most importantly, the analysis application allows selection of maximum allowed value for parktime and walktime. The default value for both is set at 59 minutes, as discussed in the \fullref{sec:processdata}, but the user is free to choose any value between zero and 99.

% levene, anova, boxplot, lue: https://www.itl.nist.gov/div898/handbook/eda/section3/eda35a.htm. On legit lähde
\begin{table}[H]
    \centering
    \caption[Records Shiny application features]{Records Shiny application features \textcolor{red}{käy läpi outputsit erityisesti statistiikassa, katso \%-linkk}} 
    \label{tab:records_shiny_features}
    \scalebox{0.8}
    {\def\arraystretch{1.3}
    \setlength\tabcolsep{1.2ex}
    \begin{tabular}{ @{} >{\raggedright\arraybackslash}p{3cm} >{\raggedright\arraybackslash}p{2cm} >{\raggedright\arraybackslash}p{6cm} >{\raggedright\arraybackslash}p{6cm} @{} }
        \toprule
        Feature & Type & Outputs & Effective settings \\
        \midrule
        1 Descriptive statistics & Analysis, table & n, median, mean, standard deviation, standard error, confidence interval for mean, lower bound, confidence interval for mean, min, max, 25th quartile, 75th quartile, skewness, kurtosis & - Maximum permitted \code{parktime} and \code{walktime} values \linebreak - Currently active response and explanatory variable \linebreak - Inactive subdivisions \\
        2 Histogram & Analysis, chart & Histogram, kernel density estimate, mean, median & - Maximum permitted \code{parktime} and \code{walktime} values \linebreak - Currently active response and explanatory variable \linebreak - Histogram binwidth \linebreak - Inactive subdivisions \\
        3 Distribution of ordinal variables & Analysis, chart & Distribution plot by explanatory variable value group & - Maximum permitted \code{parktime} and \code{walktime} values \linebreak - Currently active explanatory variable \linebreak - Explanatory variable for the distribution plot Y axis \linebreak - Inactive subdivisions \\
        4 Boxplot & Analysis, chart & Quartile data & - Maximum permitted \code{parktime} and \code{walktime} values \linebreak - Currently active response and explanatory variable \linebreak - Inactive subdivisions \\
        5 Test of homogeneity of variances (Levene's test) & Analysis, table & Equality of variances for a variable calculated for the currently active response and explanatory variable & - Maximum permitted \code{parktime} and \code{walktime} values \linebreak - Currently active response and explanatory variable \linebreak - Inactive subdivisions \\
        6 Analysis of variance (ANOVA) & Analysis, table & Analysis of differences among group means in a sample & - Maximum permitted \code{parktime} and \code{walktime} values \linebreak - Currently active response and explanatory variable \linebreak - Inactive subdivisions \\
        7 Brown-Forsythe test & Analysis, table & Analysis of equality of group variances & - Maximum permitted \code{parktime} and \code{walktime} values \linebreak - Currently active response and explanatory variable \linebreak - Inactive subdivisions \\
        8 Context map & Visualisation, map & Gain visual context & - Inactive subdivisions \\
        9 Interactive map & Visualisation, map & Choropleth map with Jenks breaks classifiction, descriptive data per postal code area (answer count, mean and median for parktime and walktime, forest amount percentage, largest YKR zone percentage) & - Selection of active cities \linebreak - Jenks breaks parameter column \linebreak - Amount of Jenks breaks classes \\
        \bottomrule
    \end{tabular}}
\end{table} 

When the user has selected a continuous and an ordinal variable to compare, they are presented a thorough set of descriptive statistics for currently active data rows with n, median, mean, standard deviation, standard error, confidence interval for lower and upper bound, minimum and maximum, 25 \% and 75 \% quantiles, skewness, and kurtosis. For the continuous variables, a histogram is available to visualise the distribution of walktime and parktime. Distribution of ordinal variables likert, parkspot and timeofday can be compared against other ordinal variables in a barplot. To study quartiles, a boxplot is available. Importantly, users can test their selection of variables with the test of homogeneity of variables (Levene's test), analysis of variance (ANOVA), and the Brown-Forsythe test \textcolor{red}{laajenna selityksiä}. To provide a visual context, a map of the research area is included. The purpose of this map is to visualise which subdivisions the user has currently active and which are deactivated. Lastly, an interactive map of the research area is provided. This map, divided in postal code areas, is affected with the options presented earlier in this paragraph. In addition, this interactive map is controlled by some options of its own. The map offers six distinct parameters for viewing the research area through Jenks breaks, alongside with the possibility to select amount of classes in the current view. Hovering the cursor over the map reveals a tooltip to view mean, median and percentage data about each postal code area in Helsinki Capital Region.

In the Shiny application for \textit{visitors} users can examine events in the timeline of the survey research (figure~\ref{fig:shinyapps}). In an interactive view cumulative charts are presented for received responses and survey page first visits. The charts reveal the effect of advertisement on actual received responses and survey traffic. While not completely verifiable, the significance of different sources of responses can be viewed in the application.

Scripts described in this chapter were written in RStudio 1.2.5033 using R for Windows 3.6.3. The source code for the data analysis and visualisation scripts are available at GitHub (\textcolor{blue}{\url{https://github.com/sampoves/Msc_thesis_data_analysis}}). The interactive analysis tools are available for viewing on shinyapps.io (\textcolor{blue}{\url{https://sampoves.shinyapps.io/records}} and  \textcolor{blue}{\url{https://sampoves.shinyapps.io/visitors}}). 