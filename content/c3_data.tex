\section{Data and methods}
\subsection{General workflow}
\justify
Flowchart of methodology/data refinement process

\subsection{Study area}
\justify
Helsinki capital region

\subsection{Parking survey}
\justify
To collect the areal parking data, the study needs a survey which respondents can use to submit their parking habits in a spatial fashion. In actuality, such survey is a combination of many components. One needs a user interface, web server and a database for storing results. Such web applications are provided by a handful of enterprises in Finland and elsewhere but a no such solutions exist which are also free and state-of-the-art web applications. a web application was programmed with JavaScript and its open-source mapping library Leaflet. The source code for this survey is available at GitHub (\textcolor{blue}{Insert link here}). In the survey the respondent is presented with a map view of Helsinki Capital Region with its 167 postal code areas and is asked to fill out three questions about each area: 
\begin{spacing}{1}
\begin{enumerate}
  \item How long does it usually take for you to park your car and arrive at your destination by foot in this postal code area (in minutes)?
  \item How familiar are you with this postal code area?
  \item What kind of parking spot do you usually take in this postal code area?
\end{enumerate}
\end{spacing}

\noindent
User is asked to only fill out postal code areas they remember parking in. Each user only needs to fill out the survey one time but they can revisit it, in case they want to add in more postal code areas.

\subsection{Processing survey data}
\justify
Vastausdatan käsittely