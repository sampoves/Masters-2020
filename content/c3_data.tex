\section{Data and methods}
\subsection{General workflow}
\justify

% tilde "~" indicates a non-breaking space
A selection of web applications was designed and programmed for this thesis. In this chapter, the process to create these applications is presented, from the design board to a functional web application to the end stage of data processing and visualisation. Four applications are presented: a spatial web survey for data collection and three separate web applications for the analysis and visualisation of the survey results. These applications directly answer the research questions while providing a possibility for a vast array of additional results and calculations. The general workflow of the thesis data processing, analysis, and visualisation can be viewed in figure~\ref{fig:gen_workflow}.

\textcolor{red}{kerro mikä on ip-osoite} \\
In the thesis research survey web application, a respondent would send data about their parking activities in a specific postal code areas, in a general sense, summing up their experiences in the most recent two years (figure~\ref{fig:gen_workflow}, the top two parallelograms). Five questions were posed, and the numerical parking time and walking time questions pertained to the first research question and the following three single-choice questions aimed to answer to the second. The survey was carried out in the four municipalities of the Helsinki Capital Region -- Helsinki, Espoo, Vantaa and Kauniainen. The survey gathered 5222 data rows from 1060 unique IP addresses. 

After the conclusion of the data collection phase, the survey data was processed, analysed, and visualised using Python and R programming languages. The process started with anonymisation of the IP address data (figure~\ref{fig:gen_workflow}, section 1, Anonymise IP addresses), and moved on to the processing proper. In this next step, all input data was processed to better work together. Input data here refer to spatial GIS layers, which were to be used as explanatory variables in the analysis, or visualisation in the various interactive applications developed for the use of this thesis. Some of the spatial layers were streamlined by removing attribute data that was not necessary for this study. Other spatial data, such as the 250 by 250 meter spatial grid of the Helsinki Capital Region, was supplemented with additional data, the postal code area, for the use of the analysis applications in later stages of the workflow (figure~\ref{fig:gen_workflow}, section 2, Preprocessing).

In the general workflow section 3, Detect illegal data, the parking survey data was analysed for potentially problematic data. For example, the survey data was analysed for data entries, where the same IP address codes had 1) answered multiple times to same postal code areas, 2) entered identical values to some or all survey questions, and 3) entered anomalously high values in the numeric questions. This section also had the feature to remove any data rows deemed unwanted, but in the end, it was deemed a better solution to remove any problematic data later on in the workflow. 

The section 4, insert data into \textit{postal}, prepares the postal code areas data for later analysis. For example, each postal code area was supplemented with data about mean and median parking and walking times in that area, and indicator values of the spatial data were added. For example, in a geoprocessing calculation each postal code area was given a value in percentage how much of its area is artificial according to the CORINE land cover and land use dataset (\cite{FinnishEnvironmentInstitute2018}). In the thesis general workflow section 5, all of the newly calculated postal code area data was added to the dataframe containing the survey results. This process would help in data analysis and visualisation.

The survey result analysis and visualisation was carried out in R (figure~\ref{fig:gen_workflow}, sections 6, 7, and 8). In section 6, Preprocess Travel Time Matrix 2018, the cumbersome dataset was transformed for the specific needs of the analysis applications. Then, in sections 7 and 8, data analysis and visualisation applications were programmed for efficient and flexible data analysis for this thesis, but also to release the survey results to the public, maintaining the mission of openness and transparency of this thesis. The analysis application contains a range of tools for viewing the important properties of the survey data, such as descriptive statistics, different kinds of charts, such as histograms and boxplots and, finally, an interactive map to view the results in a spatial manner. A visitors application helps track the timeline of data collection, cumulatively showing the data rows as they were received. Finally, the travel time comparison application made it possible to analyse the parking process proportion in the total duration of travel chains in the reseach area. This application was designed to answer in the third research question.

\newpage
\begin{figure}[H]
    \centering
    %Cropping an image: trim={<left> <lower> <right> <upper>},clip
    \includegraphics[trim=0.4cm 2.0cm 0.4cm 0.25cm,clip,width=\columnwidth,scale=0.5]{thesis_workflow.pdf}
    \caption{The general workflow of the thesis survey data processing in Python and R.}
    \label{fig:gen_workflow}
\end{figure}

\newpage
\subsection{Study area}
\justify

% autojen käyttölaajennus:

%tsek helsingin raportit ja henkilöliikennetutkimus, sitten päätä mitä päätä mitä kirjoitat

% Potentiaalisesti hyödyllisiä linkkejä Helsingin autoilusta
%---liikennemäärät
%https://www.hsl.fi/tutkimukset/liikenne-ja-matkustajalaskennat-ja-mittaukset %arcgis-kartat!

%https://www.hsl.fi/tutkimukset/muut-selvitykset
%https://www.hel.fi/helsinki/fi/kartat-ja-liikenne/kadut-ja-liikennesuunnittelu/tutkimus-ja-tilastot/moottoriajoneuvoliikenteen-maarat/
%https://www.hel.fi/helsinki/fi/kartat-ja-liikenne/kadut-ja-liikennesuunnittelu/tutkimus-ja-tilastot/liikenteen-sujuvuus/
%http://pxnet2.stat.fi/PXWeb/pxweb/fi/StatFin/StatFin__lii__mkan/

%--HLT
%https://julkaisut.vayla.fi/pdf8/lti_2018-01_henkiloliikennetutkimus_2016_web.pdf

%--liikkumistottumukset
%\cite{Brandt2019}
%https://www.hsl.fi/sites/default/files/hsl_julkaisu_9_2019_netti.pdf 

%-- autojen määrä viime vuosikymmeninä suomessa
%https://www.stat.fi/til/mkan/2019/mkan_2019_2020-02-28_tie_001_fi.html

%-- HSL aukee-projekti
%\cite{ElinaBrandtMatleenaLindeqvist2016}
%Powerpoint, Matleena Lindeqvist: https://www.hel.fi/hel2/helsinginseutu/hs_tunnusluvut/liikennemaara_ja_autonomistus.pdf

The study area of this thesis is the Helsinki Capital Region in Finland (figure \ref{fig:thesis_resarea}). It comprises of the municipalities of Helsinki, Espoo, Vantaa, and Kauniainen and in August 2020 the area had a total population of 1.2 million (\cite{StatisticsFinland2020a}). In practice, the whole area amalgamates as one complete functional area with boundaries of the municipalities indistinguishable at the street level. Of these four municipalities Helsinki is the hub, and can be considered to contain the only inner city features of the municipalities (\cite{FinnishEnvironmentInstitute2013}). The urban fabric of Espoo, Vantaa, and Kauniainen mostly consist of suburban areas with an occasional industrial area here and there. Relatively recently, large shopping complexes have risen in important traffic junctions around the cities. The exact boundaries of the research area are based on the dataset \textit{PAAVO open data by postal code area} (\cite{StatisticsFinland2019a}).

\begin{figure}[H]%
    \includegraphics[width=\textwidth]{images/thesis_resarea.png}
    \caption[Research area map]{Map of the Helsinki Capital Region, the research area of this thesis. This research focuses on PAAVO postal code areas. The data has a municipality value for each postal code area (see the colouring of areas and map legend) but the areas do not completely align with the official municipality boundaries. (\cite{OpenStreetMap})}%
    \label{fig:thesis_resarea}%
\end{figure}

The Helsinki Capital Region has experienced considerable growth in the recent past and this trend is poised to continue. According to Statistics Finland (\citeyear{StatisticsFinland2020c}), the population of Helsinki has grown with more than 100 000 people in the last twenty years, and will grow with another 100 000 in the next twenty years. Since 2000, nearly 80 000 people moved to Espoo, and in the next twenty years, more than 50 000 people are forecast to move in. Vantaa follows suit, with an increase of about 100 000 people in 2000--2040. Also Kauniainen grows, albeit in much smaller scale than the three larger Helsinki Capital Region municipalities. In the next twenty years about 1 500 people will move into Kauniainen.

About two thirds of the households in the Helsinki Capital Region own at least a single car (64 \%) (figure~\ref{fig:cars_per_household}). Private cars are the most ubiquitous in Espoo, where 53 percent of households own one car and 19 percent own two cars. In the entire Helsinki Capital Region, Inner Helsinki households are the most car-free with 61 percent of households not in possession of a car (\cite{Liikennevirasto2018}).

\begin{figure}[H]%
    \includegraphics[width=\textwidth]{images/hlt_cars_per_household.png}
    \caption[The number of private cars in households in the Helsinki Capital Region]{Amount of private cars in households in the Helsinki Capital Region. Numbers represent percentage. Adapted from Henkilöliikennetutkimus 2016 (\cite{Liikennevirasto2018}).}%
    \label{fig:cars_per_household}%
\end{figure}

Even as the share of private car journeys of all travel modes has been on the decrease in recent times in the Helsinki Capital Region, the absolute amount of cars is slightly on the rise in the same area (\cite{Brandt2019}; table~\ref{tab:registered_cars}). At the same time, the traffic is increasing in the major roads, such as the beltways Ring I and III (\cite{Helsinginseudunliikenne2020}; figure~\ref{fig:traffic_amount_autumn2018}). However, the density of private cars (cars / 1000 residents) that are commissioned for traffic is decreasing, perhaps owing the efforts to make public transport more appealing as roads become increasingly congested and parking spaces scarcer (table~\ref{tab:car_density}).

\begin{figure}[H]%
    \includegraphics[width=\textwidth]{images/traffic_amount_autumn2018.png}
    \caption[Motor vehicle traffic on main roads in the Helsinki Capital Region]{Motor vehicle traffic on an average autumn weekday in the Helsinki Capital Region. Screen capture from the web map service, colors enhanced (\cite{Helsinginseudunliikenne2019}; \cite{OpenStreetMap}). \textcolor{red}{POISTA? Please note that in this screen capture smaller roads and their labels sometimes obscure the labels of larger roads, for example, Finnish national road 1 to Turku is obstructed by the labels of the old highway to Turku.}}%
    \label{fig:traffic_amount_autumn2018}%
\end{figure}

% \calcin is to calculate increase in percentage from two values. #1 new number, #2 original number
\begin{hyphenrules}{nohyphenation}
    \begin{table}[H]
        \centering
        \def\arraystretch{1.2}
        \setlength\tabcolsep{1.2ex}
        \caption[Amount of cars registered in the Helsinki Capital Region and in Finland total]{The number of private cars registered in the Helsinki Capital Region municipalities, in KUUMA municipalities, and in Finland total (\cite{StatisticsFinland2020b}). Private cars decommissioned from traffic are not included in this table.} 
        \label{tab:registered_cars}
        \begin{tabular}{ llll @{} >{\raggedleft\arraybackslash}p{2cm} @{} }
            \toprule
                                            & 2011      & 2015      & 2019      & Growth 2011--2019 \\
            \midrule
            Helsinki                        & 207 639   & 206 229   & 214 583   & \calcin{214583}{207639} \\
            Espoo                           & 107 833   & 115 446   & 122 185   & \calcin{122185}{107833} \\
            Vantaa                          & 91 844    & 98 963    & 109 068   & \calcin{109068}{91844} \\
            Kauniainen                      & 3 815     & 4 105     & 4 324     & \calcin{4324}{3815} \\
            \greyrule
            KUUMA municipalities*           & 149 930   & 157 984   & 169 760   & \calcin{169760}{149930} \\
            \greyrule
            Finland                         & 2 978 729 & 3 257 581 & 3 574 570 & \calcin{3574570}{2978729} \\
            \bottomrule
            \multicolumn{5}{ p{12cm} }{\textsuperscript{*}\footnotesize{KUUMA municipalities are the Greater Helsinki municipalities without the Helsinki Capital Region: Hyvinkää, Järvenpää, Kirkkonummi, Kerava, Mäntsälä, Nurmijärvi, Pornainen, Sipoo, Tuusula, and Vihti (\cite{KUUMA-seutuliikelaitos2020}).}}
        \end{tabular}
    \end{table}
\end{hyphenrules}

% use package xfp for density calculations. This makes the calculations much more transparent
\begin{hyphenrules}{nohyphenation}
    \begin{table}[H]
        \centering
        \def\arraystretch{1.2}
        \setlength\tabcolsep{1.2ex}
        \caption[Density of private cars in the Helsinki Capital Region in 2019]{Density of private cars in the Helsinki Capital Region municipalities, in KUUMA municipalities, and in the entire Finland in 2019 (\cite{StatisticsFinland2020b}, \citeyear{StatisticsFinland2020}). Private cars decommissioned from traffic are not included in this table. Density is formulated as cars/1000 inhabitants.} 
        \label{tab:car_density}
        \scalebox{0.9}
        {\begin{tabular}{ lllLlll }
            \toprule
            			            & \multicolumn{3}{c}{2011} & \multicolumn{3}{c}{2019} \\
						            \cmidrule(lr{\tbspace}){2-4} \cmidrule(lr){5-7}
                                    & Population & Private cars & Density & Population & Private cars & Density \\
            \midrule    
            Helsinki                & 595 384   & 207 639   & \fpeval{round(207639/(595384/1000),0)}  & 656 970   & 214 583   & \fpeval{round(214583/(656970/1000),0)} \\
            Espoo                   & 252 439   & 107 833   & \fpeval{round(107833/(252439/1000),0)}  & 291 490   & 122 185   & \fpeval{round(122185/(291490/1000),0)} \\
            Vantaa                  & 203 001   & 91 844    & \fpeval{round(91844/(203001/1000),0)}   & 236 434   & 109 068   & \fpeval{round(109068/(291490/1000),0)} \\
            Kauniainen              & 8 807     & 3 815     & \fpeval{round(3815/(8807/1000),0)}      & 9 990     & 4 324     & \fpeval{round(4324/(9990/1000),0)} \\
            \greyrule
            KUUMA municipalities*   & 315 094   & 149 930   & \fpeval{round(149930/(315094/1000),0)}  & 326 211   & 169 760   & \fpeval{round(169760/(326211/1000),0)} \\
            \greyrule
            Finland                 & 5 401 267 & 2 978 729 & \fpeval{round(2978729/(5401267/1000),0)}& 5 532 333 & 3 574 570 & \fpeval{round(3574570/(5532333/1000),0)} \\
            \bottomrule
            \multicolumn{7}{ p{16cm} }{\textsuperscript{*}\footnotesize{KUUMA municipalities are the Greater Helsinki municipalities without the Helsinki Capital Region: Hyvinkää, Järvenpää, Kirkkonummi, Kerava, Mäntsälä, Nurmijärvi, Pornainen, Sipoo, Tuusula, and Vihti (\cite{KUUMA-seutuliikelaitos2020}).}}
        \end{tabular}}
    \end{table}
\end{hyphenrules}

The City of Helsinki monitors inbound and outbound motor traffic on several fronts. On an average autumn week day in 2019, the boundary of the municipality was crossed by 662 000 vehicles, inner city boundary by 319 000 vehicles, and Helsinki peninsula boundary by 188 000 vehicles. During the last ten years, motor vehicle crossings of the boundary of Helsinki has increased 6 percent. Inner city boundary crossings have increased 11 percent while crossings of the Helsinki peninsula have decreased by 21 percent. (\cite{CityofHelsinki2019})

In Espoo, motor traffic is monitored on eastern and western municipality boundaries. Facing east, to Helsinki, on average 308 000 vehicles crossed the boundary daily. To the west, to Kirkkonummi and Vihti, daily crossings measured at 93 000 vehicles. (\cite{Espoonkaup2020}). Compared to the previous year, the cities of Espoo and Helsinki report increases in daily traffic along their boundaries in the range of 1--2 percent.

\textcolor{red}{POISTA? This study did not find corresponding traffic research for Vantaa.}

\newpage
\subsection{Data}
\justify

Essential data for this study was provided by Statistics Finland, the research group Digital Geography Lab of University of Helsinki, the municipalities of Helsinki Capital Region, and Finnish Environment Institute (table~\ref{tab:used_data}).

% hyphenrules: prevent hyphenation temporarily
\begin{hyphenrules}{nohyphenation}
    \begin{table}[H]
        \centering
        \caption[Thesis data]{Data utilised in the thesis.} 
        \label{tab:used_data}
        % use \scalebox{}{} to control table size. Note the additional curly brackets enveloping the entire tabular object
        \scalebox{0.8}
        % use \arraystretch to add whitespace between rows. \setlength\tabcolsep for columns
        {\def\arraystretch{1.5} 
        \setlength\tabcolsep{1.2ex}
        % One unit here is ">{\raggedright\arraybackslash}p{4cm}". \raggedright prevents justification of text and conveniently allows flush right or flush left, which is not possible with column command p{4cm} alone.
        \begin{tabular}{ @{} >{\raggedright\arraybackslash}p{4cm} >{\raggedright\arraybackslash}p{4cm} >{\raggedright\arraybackslash}p{4.5cm} >{\raggedright\arraybackslash}p{3.5cm} >{\raggedleft\arraybackslash}p{3.5cm} @{} }
            \toprule
            Data & Description & Purpose in thesis & Abbreviation in thesis & Citation \\
            \midrule
            CORINE land cover 2018 & Land use and land cover data in vector format & Artificial surface data & \textit{CORINE} & \cite{FinnishEnvironmentInstitute2018} \\
            Helsinki Region Travel Time Matrix 2018 & Travel time and distance information for routes between all \textit{grid} cell centroids in the Capital Region of Helsinki & Use in travel time comparison calculations between \textit{TTM} and thesis survey results & \textit{TTM} & \cite{Tenkanen2018} \\
            MetropAccess-YKR-grid & Statistical grid of 250 x 250 meter cells for monitoring urban structure, the Helsinki Capital Region area & Use in travel time comparison calculations between Travel Time Matrix 2018 and thesis survey results & \textit{grid} & \cite{Toivonen2014a}, \cite{StatisticsFinland2020} \\
            Paavo -- Open data by postal code area 2018 & Helsinki Capital Region postal code areas & Thesis research area and the basic unit of spatial resolution in the survey & \textit{postal} & \cite{StatisticsFinland2019a} \\
            Regional population density 2012 & Population density with municipality boundaries & Visualisation in R & \textit{hcr\_muns} & \cite{StatisticsFinland2012} \\
            Subdivisions of the Helsinki Capital Region & The subdivisions of the municipalities of the Helsinki Capital Region & Visualisation in R & \textit{subdivisions} & \cite{HelsinginEspoonVantaanjaKauniaistenmittausorganisaatiot2011} \\
            Zones of urban structure (Yhdyskuntarakenteen vyöhykkeet) 2017 & Delineation of urban areas based on the theory of urban fabrics & Data on spatial structure of urban areas & \textit{YKR zones} & \cite{Ristimaki2017} \\
            \bottomrule
        \end{tabular}}
    \end{table} 
\end{hyphenrules}

The foundation of this research is the dataset \textit{PAAVO -- open data by postal code area} (abbreviated \textit{postal} in this thesis) (figure~\ref{fig:paavo_resarea} \textcolor{red}{aika pitkällä tää kuva}) (\cite{StatisticsFinland2019a}). This data provides a large selection of data regarding the population of every postal code area in Finland. This includes detailed demographics and data about employment by field which follows the industrial classification TOL 2008 (\cite{Tilastokeskus2008}). However, this thesis only utilises the spatial definitions of the postal code areas, using these polygons to differentiate areas from each other in the web survey. This research makes use of the PAAVO 2018 dataset, released in January 2019.

In this thesis, the \textit{CORINE Land Cover 2018} (abbreviated \textit{CORINE}, figure~\ref{fig:datalayers_corine}) vector format dataset is used to locate built area, or artificial surface, in the Helsinki Capital Region (\cite{FinnishEnvironmentInstitute2018}). Provided by Finnish Environment Institute, \textit{CORINE} contains polygonal data about land cover and land use for the entire nation in different hierarchy levels. In this thesis, the hierarchy level 1 value \textit{Artificial surfaces} is used (\textcolor{red}{tää table oudossa paikassa}  table~\ref{tab:corine_artificial}). The minimum unit depicted in this dataset is 25 hectares in area or 100 meters in width. This slightly coarse data fits well with the spatially simplified nature of PAAVO postal code areas. \textit{CORINE} dataset is an integration of automated satellite image interpretation and existing digital map data. In this thesis, every postal code area is given the attribute value of how much, in percentaege, of the area is artificial, built surface.

\textit{MetropAccess-YKR-grid} (abbreviated \textit{grid}, figure~\ref{fig:datalayers_metropaccess_ykr}) is a spatial dataset which consists of cells with the dimensions of 250 by 250 meters ($n=13231$) (\cite{Toivonen2014a}). The dataset is used in the MetropAccess project of Digital Geography Lab and is based on the statistical grid dataset provided by Finnish Environment Institute and Statistics Finland (\cite{StatisticsFinland2020}). \textit{Grid} is a simple dataset and only contains the attribute data of spatial coordinates of cells and their identifiers, the YKR ID. Using the YKR ID it is effortless to join \textit{TTM} data with the statistical data provided by Statistics Finland, allowing wide-ranging possibilities for further research. The extent of the dataset is the Helsinki Capital Region.

A main focus in this thesis was to compare the thesis survey results with \textit{Helsinki Region Travel Time Matrix 2018} (abbreviated \textit{TTM}), a dataset provided by Digital Geography Lab, a research group based in the University of Helsinki, the department of geosciences and geography (\cite{Tenkanen2018}). The newest release of their dataset provides travel times for public transport, private car, walking, and bicycling between all \textit{grid} cells. All travel times in this dataset were calculated using the door-to-door approach, which incorporates all parts of a journey from place A to place B into the travel time, including walking from one's home door to the car or bus stop and the time spent searching for parking (\cite{Salonen2013}, figure~\ref{fig:door-to-door}). This thesis focuses on journeys made by private car.

All postal code areas in the survey results were classified with the \textit{zones of urban structure} (officially \textit{Yhdyskuntarakenteen vyöhykkeet}, abbreviated \textit{YKR zones}, figure~\ref{fig:datalayers_ykr}) (\cite{Ristimaki2017}). Utilising the same statistical grid of 250 x 250 meters as \textit{grid}, \textit{YKR zones} classifies the cells to produce pedestrian, public transport, and automobile zones in and around Finland's urban regions using the theory of urban fabrics. According to this theory, these three zones developed during different times in the urban region's history (\cite{Newman2016}). In this thesis, every postal code area is assigned with a class defined in the \textit{YKR zones} based on which class has the largest presence. Adding this data into the survey results aimed to provide more possibilities to explain the hypothetical dissimilarity of survey results in different parts of the Helsinki Capital Region.

\textit{The regional division maps of the Helsinki Capital Region} (officially \textit{Pääkaupunkiseudun aluejakokartat}, abbreviated \textit{subdivisions}, figure~\ref{fig:subdiv_placement}) was used in this thesis to analyse and visualise the survey results by subdivisions of the Helsinki Capital Region (\cite{HelsinginEspoonVantaanjaKauniaistenmittausorganisaatiot2011}). Dividing the survey results into subdivisions would potentially give rise to local phenomena which would not be perceptible in the finest available level of spatial resolution, the postal code areas.

The dataset \textit{Regional population density 2012} (figure~\ref{fig:subdiv_placement}) was used in this thesis to visualise the boundaries of the municipalities in Helsinki Capital Region (\cite{StatisticsFinland2012}). 

% \newpage to fit artificial and ykr_zones on the same page
\newpage
\begin{figure}[H]%
    \centering
    \includegraphics[width=.88\textwidth]{images/thesis_data_artificial.png}
    \caption[CORINE Land Cover 2018 artificial surfaces in Kauniainen and eastern Espoo.]{CORINE Land Cover 2018 artificial surfaces in Kauniainen and eastern Espoo (\cite{OpenStreetMap}).}%
    \label{fig:datalayers_corine}%
\end{figure}

\begin{figure}[H]%
    \centering
    \includegraphics[width=.88\textwidth]{images/thesis_data_ykr_zones.png}
    \caption[Zones of urban structure in eastern Helsinki.]{Zones of urban structure, portrayed here in eastern Helsinki (\cite{OpenStreetMap}).}%
    \label{fig:datalayers_ykr}%
\end{figure} 

\begin{figure}[H]%
    \includegraphics[width=\textwidth]{images/thesis_data_grid.png}
    \caption[MetropAccess grid]{MetropAccess grid GIS layer (\cite{OpenStreetMap}).}%
    \label{fig:datalayers_metropaccess_ykr}%
\end{figure}

\newpage
\subsection{Software}
\justify

A wide variety of software was used in the research for this thesis. The research strived for maximum openness and transparency and therefore much of the software employed in this work is free, open-source, or both. The research survey application utilised several essential web technologies such as JavaScript, HTML, CSS and PHP (table~\ref{tab:used_langs}). Using the web mapping library Leaflet, with the assistance of jQuery and other libraries, a modern and easy-to-use survey web application was created. Server-side, the programming language PHP was used to verify received data.

Data processing was carried out in Python 3.7.6 and R for Windows 3.6.3, with the initial processing done in Python and most of the analysis and visualisation in R (table~\ref{tab:used_langs}). Much of the work depended on additional software libraries available for the programming languages (table~\ref{tab:used_soft}). Python Anaconda version 2020.02 -- a distribution for Python for statistical computing -- provides the majority of the needed software libraries in the installation package, with the notable exception of GeoPandas, a library for geospatial pandas DataFrames in Python, and Shapely, a library for manipulation and analysis of planar geometric objects. In R, many libraries were used to achieve a comprehensive set of descriptive statistics. Libraries such as Shiny, ggplot2, and ggiraph formed the basis of the visualisation of the survey results.

% LaTeX is not a programming language for the most part: https://www.quora.com/Is-LaTeX-a-programming-language
\textcolor{red}{onks liian tarkasti} \\
The thesis was written and typeset with LaTeX using the online LaTeX editor Overleaf. LaTeX is a document preparation system, used to create documents such as scientific articles. LaTeX adheres to the WYSIWYM (what you see is what you mean) system, as opposed to the "what you see is what you get" (WYSIWYG) system of text editors such as Microsoft Word, meaning that after establishing a set of parameters LaTeX will automatically compute the document formatting, while the user can concentrate on the document content. While LaTeX can be considered a programming language, it is more closely related to markup languages such as Hypertext Markup Language (HTML). In this LaTeX document, the LaTeX distribution TeX Live 2019 was used. Overleaf supports GitHub integration and as a result the complete thesis is available for viewing in the GitHub repository \textcolor{blue}{\url{https://github.com/sampoves/Masters-2020}} alongside with its entire development history. Additionally, the template of this thesis is provided at \textcolor{blue}{\url{https://github.com/sampoves/msc-thesis-template}}.

In addition to the aforementioned technologies, the flowcharts in this thesis were created with the web application \textit{diagrams.net}. Most of the map visualisations of this thesis were made using the geographic information system application QGIS version 3.12.2.

% Consider! Removing \raggedright and hyphenrules will enable nice even table cells. Could be worth it to look into
\begin{hyphenrules}{nohyphenation}
    \begin{table}[H]
        \centering
        \caption[Thesis programming languages]{Programming languages, essential technologies, and \glspl{ide} utilised in the thesis, grouped by the function in this thesis.} 
        \label{tab:used_langs}
        \def\arraystretch{1.4}
        \setlength\tabcolsep{1.2ex}
        \begin{tabular}{ @{} >{\raggedright\arraybackslash}p{3cm} >{\raggedright\arraybackslash}p{4.5cm} >{\raggedright\arraybackslash}p{3.5cm} >{\raggedleft\arraybackslash}p{3.5cm} @{} }
            \toprule
            Programming language and \gls{ide} & Description & Purpose in thesis & Citation \\
            \midrule
            JavaScript, HTML, CSS (NetBeans 8.2.0) & Essential web technologies & Research survey programming, analysis and visualisation application programming & \cite{WHATWG2020}, \cite{W3C2020}, \cite{ECMA2019}, \cite{ApacheSoftwareFoundation2016} \\
            Python 3.7.6, Anaconda 2020.02 (Spyder 4.0.1) & Anaconda is a Python distribution for scientific computing & Survey data processing & \cite{Python3Reference}, \cite{AnacondaInc.2020}, \cite{SpyderProjectContributors2020} \\
            R for Windows 3.6.3 (RStudio 1.2.5033) & Programming language environment for statistical computing & Survey data analysis and visualisation & \cite{RCoreTeam2020}, \cite{RStudioTeam2015} \\
            LaTeX (Overleaf) & Document preparation system & Thesis formatting, structure, and writing & \textcolor{red}{latexcite, overleafcite} \\
            \bottomrule
        \end{tabular}
    \end{table} 
\end{hyphenrules}

\begin{hyphenrules}{nohyphenation}
    \begin{table}[H]
        \centering
        \caption[Essential software packages in thesis]{Essential software libraries used in the thesis. The complete list of libraries can be viewed in \textcolor{red}{appendix number something}.} 
        \label{tab:used_soft}
        \def\arraystretch{1.2}
        \setlength\tabcolsep{1.2ex}
        \begin{tabular}{ @{} >{\raggedright\arraybackslash}p{2.5cm} >{\raggedright\arraybackslash}p{3cm} >{\raggedright\arraybackslash}p{4cm} >{\raggedleft\arraybackslash}p{3cm} @{} }
            \toprule
            Programming language & Software package & Description & Citation \\
            \midrule
            % NB, manual positioning of the multirow label
            \multirow{3}{*}[-5ex]{JavaScript} & Leaflet 1.4.0 & Web mapping library for the research survey & \cite{Agafonkin2019} \\
            & jQuery 3.4.1 & Simplification of HTML DOM traversal and other features & \textcolor{red}{cite} \\
            & Font Awesome 5.13.0 & Font and icon collection & \textcolor{red}{cite} \\
            \greyrule
            \multirow{4}{*}[-6.5ex]{Python} & pandas 1.0.1 & Data analysis and manipulation & \cite{McKinney2011a} \\
            & GeoPandas 0.5.0 & Geographic data operations & \cite{GeoPandasDevelopers2019} \\
            & Shapely 1.6.4.post1 & Geometric objects, predicates, and operations & \cite{Gillies2019} \\
            & rtree 0.8.3 & Spatial indexing & \cite{Gillies2014} \\
            \greyrule
            \multirow{4}{*}[-7.0ex]{R} & Shiny 1.4.0.2 & Web application framework for R & \cite{Chang2019} \\
            & ggplot2 3.3.0 & Data visualisation & \cite{Wickham2016} \\
            & ggiraph 0.7.0 & Interactive ggplot2 graphics & \cite{Gohel2019} \\
            & dygraphs 1.1.1.6 & Interactive time series charting & \cite{Vanderkam2018} \\
            & fst 0.9.2 & High-performance writing and loading of data & \textcolor{red}{cite} \\
            \bottomrule
        \end{tabular}
    \end{table} 
\end{hyphenrules}

\newpage
\subsection{Methods}

\subsubsection{Programming the parking survey}
\justify

To achieve maximum transparency and repeatability for this research, in addition to freedom in survey content and appearance, a survey web application was programmed from the ground up utilising HTML, JavaScript and PHP. The survey and its supporting infrastructure was installed on a virtual machine in CSC's -- the state owned ICT solutions company -- Taito supercluster. CSC offers virtual machines in several different hardware configurations, or flavors. The virtual machine flavor picked for this survey was \textit{standard.medium}, a flavor with 3.9 gigabytes \acrshort{ram}, three virtual \acrshort{cpu}s and 80 GB of disk space. Running on the Linux distribution Ubuntu version 16.04, the backbone of the survey ecosystem was a \gls{lamp} (Linux, Apache, MySQL, PHP), a software bundle which incorporates the Linux operating system, Apache web server software, \gls{mysql} relational database management system and the PHP programming language environment for server-side scripting. The public component of the survey is the front-end, the only component of the survey system a respondent would interact with (figure~\ref{fig:js_survey_welcome}). One may use additional software in a \gls{lamp} for extended functionality or can replace some of the components with a wide array of alternatives. This thesis utilises the components described in the table~\ref{tab:survey_components}.

\begin{figure}[H]%
    \includegraphics[width=\textwidth]{images/js_survey_welcome.png}
    \caption[Survey application landing page]{The parking survey web application's welcoming dialog window.}%
    \label{fig:js_survey_welcome}%
\end{figure}

\begin{hyphenrules}{nohyphenation}
    \begin{table}[H]
        \centering
        \def\arraystretch{1.2}
        \setlength\tabcolsep{1.2ex}
        \caption[Research survey web application components]{The research survey web application components (\gls{lamp}).} 
        \label{tab:survey_components}
        \begin{tabular}{ @{} >{\raggedright\arraybackslash}p{3cm} >{\raggedright\arraybackslash}p{3cm} >{\raggedright\arraybackslash}p{5.5cm} @{} }
            \toprule
            Component & Version & Description \\
            \midrule
            Ubuntu & 16.04.6 & Linux distribution, the operating system for the virtual machine \\
            Apache HTTP Server & 2.4.18-2ubuntu3.9 & Web server software, management of website requests and responses \\
            MySQL & 5.7.25-0ubuntu0.16.04.2 & Relational database management system, survey database operations \\
            PHP & 7.0.33-0ubuntu0.16.04.1 & Programming language, used for server side scripting \\
            Parking survey front-end & 16.5.2019 & Survey visible to user, graphical user interface \\        
            \bottomrule
        \end{tabular}
    \end{table}
\end{hyphenrules}

Setting up the virtual machine for the use of the survey was a process of a few stages. The \gls{lamp} was installed on the fresh virtual machine with the command \code{sudo apt install lamp-server}. After the successful installation, the \gls{mysql} tables were formed and relevant users created. The last step before a fully functioning web server was using root access to give the survey components permission to access relevant system directories. The parking survey's GitHub repository (\textcolor{blue}{\url{https://github.com/sampoves/parking-in-helsinki-region}}) may be viewed for the full step-by-step install procedure used to set up the web server for this survey research.

The survey front-end was programmed in NetBeans \gls{ide} 8.2 in mostly JavaScript using an open-source mapping library Leaflet (software version 1.4.0) in January--May 2019. In the survey, the respondent was presented with a map view of the Helsinki Capital Region with its 167 postal code areas with the ability to drag the view, zoom in and out, search for places and addresses, choose the language between English and Finnish, and tweak various other settings to their liking. In this web survey, the respondent was asked to pick as many postal code areas as they could remember parking in in the last two years, and answer to five questions per each postal code area (table~\ref{tab:js_survey_questions} and figure~\ref{fig:js_survey_questions}). In each question, the respondent was asked to estimate their parking experience in that postal code area usually during the past two years. The last two years was chosen as the timeframe to allow respondents to comfortably recall parking events which happened during the subjective notion of "recent memory" while also forbidding the submission of out of date parking times. 

\begin{hyphenrules}{nohyphenation}
    \begin{table}[H]
        \centering
        \caption{Survey questions and question choices.} 
        \label{tab:js_survey_questions}
        \def\arraystretch{1.5}
        \setlength\tabcolsep{1.2ex}
        \begin{tabular}{ >{\raggedright\arraybackslash}p{5.5cm} >{\raggedright\arraybackslash}p{5cm} >{\raggedright\arraybackslash}p{2.3cm} }
            \toprule
            Question & Question choices & Abbreviation \\
            \midrule
            How long does it usually take for you to find a parking spot and park your car in this postal code area (in minutes)? & 0--99 & \code{parktime} \\
            How long does it usually take for you to walk from your parking spot to your destination in this postal code area (in minutes)? & 0--99 & \code{walktime} \\
            How familiar are you with this postal code area? & 1 -- Extremely familiar\linebreak2 -- Moderately familiar\linebreak3 -- Somewhat familiar\linebreak4 -- Slightly familiar\linebreak5 -- Not at all familiar & \code{likert} \\
            What kind of parking spot do you usually take in this postal code area? & 1 -- Parking space on the side of the street\linebreak2 -- Parking lot\linebreak3 -- Parking garage\linebreak4 -- Private or reserved spot\linebreak5 -- Other & \code{parkspot} \\
            At what time of the day do you usually park in this postal code area? & 1 -- Weekday, rush hour (07.00--09.00 and 15.00--17.00)\linebreak2 -- Weekday, other than rush hour\linebreak3 -- Weekend\linebreak4 -- None of the above, no usual time & \code{timeofday} \\
            \bottomrule
        \end{tabular}
    \end{table} 
\end{hyphenrules}

\begin{figure}[H]%
    \centering
    \subfloat[Survey questions in English.]{{\includegraphics[width=6.25cm]{js_survey_en.png} }}%
    \qquad
    \subfloat[Survey questions in Finnish.]{{\includegraphics[width=6.25cm]{js_survey_fi.png} }}%
    \caption[Research survey questions in the web application]{For each postal code area of their choosing the respondent would answer to these five questions. The survey was made available in English and Finnish.}%
    \label{fig:js_survey_questions}%
\end{figure}

\textcolor{red}{lisää kappale, jossa selitän kysymysten sisällön auki, tsekkaa apu surveystä} \\ 
The maximum values for searching for parking and walking to destination were consciously placed to 99 in an effort for the range to not feel restrictive for the survey respondent.

In the introduction to the survey, it was explained to respondents that all answers were meant to be estimates as the survey was not about an exact time and place. To mitigate confusion and errors made by respondents, a comprehensive help functionality and a location search tool were implemented in the parking survey. Once the respondent was finished with the survey, they would send their responses to the server. Respondents were welcomed to return to the survey to send additional data on any postal code areas they had missed the last time. Figure~\ref{fig:survey_process} visualises the the steps respondents would follow to send their data in the survey application.

\begin{figure}[H]%
    \centering
    \subfloat[Respondent arrives to the survey web application to see a map with the postal code areas of the Helsinki Capital Region lined out.]{{\includegraphics[width=7cm]{js_survey_process1.png} }}%
    \quad
    \subfloat[Respodent proceeds to fill out their parking experiences in freely chosen postal code areas.]{{\includegraphics[width=7cm]{js_survey_process2.png} }}%
    \quad
    \subfloat[\textit{Submit records} button activates when all questions in all selected postal code areas are completed.]{{\includegraphics[width=7cm]{js_survey_process3.png} }}%
    \quad
    \subfloat[Respondent receives a prompt to confirm that their submission was successful.]{{\includegraphics[width=7cm]{js_survey_process4.png} }}%
    \caption[Steps to fill out the survey]{A respondent would follow these steps to submit data through the survey web application.}%
    \label{fig:survey_process}%
\end{figure}

When data was received from the respondent, a script written in \gls{php} verified the data contents. This was an effort to prevent attacks on the web server running the study survey. Only specific variables of specific types were accepted from the front-end. Additionally, the \gls{php} verification made sure falsified or incomplete data would not be accepted into the database containing the verified results. If the server-side verification test failed in any way, the respondent was informed about it. 

In addition to the data verification, a PHP script tracked the IP addresses which accessed the survey web server. By using the survey, respondents agreed that their IP addresses were recorded for the use of this thesis solely to identify falsified or overlapping data and detect unique visits. All IP addresses were anonymised with a Python script and original sensitive was data deleted. The anonymisation script is available for viewing at the thesis data analysis repository at GitHub (\textcolor{blue}{\url{https://github.com/sampoves/thesis-data-analysis}}).

As a final survey component, the server side contained two MySQL datatables, one for received data (table~\ref{tab:mysql_records}) and another for survey web page hits (table~\ref{tab:mysql_visitors}). In the table \textit{records}, the following data was recorded: time of sending (column name \code{timestamp}), IP address (\code{ip}), postal area code (\code{zipcode}), a value in the sequence 1--5 for the likert question (\code{likert}), a value in the sequence 1--5 for the question what type of parking spot was used (\code{parkspot}), an integer value for how long it usually took to park in this location (\code{parktime}), an integer value for how long it usually took to walk from parking place to one's destination (\code{walktime}), and a value in the sequence 1--4 for the question at what time of the day one usually parks in the location (\code{timeofday}) (table~\ref{tab:mysql_records_str}). In the table \textit{records}, it is notable that in the case a respondent sent the web server data for multiple postal code areas, each of the postal code areas would take up their own row in the data table. Consequently, it was theoretically possible for one respondent to simultaneously submit 167 rows of data.

In the table \textit{visitors}, the following data was recorded: IP address (\code{ip}), the timestamp of the first visit of this IP address (\code{ts\_first}), the timestamp of the latest visit of this IP address (\code{ts\_latest}), and the count of visits (\code{count}). In this table, an IP address is only stored once. On the first visit of an IP address, the row for that IP address is created in the data table with \code{ts\_first} and \code{ts\_latest} being identical. On further visits of that IP address the original row is appended with updated information in the columns \code{ts\_latest} and \code{count} (table~\ref{tab:mysql_visitors_str}).

% \scalebox to prevent table going too wide
\begin{hyphenrules}{nohyphenation}
    \begin{table}[H]
        \centering
        \setlength\tabcolsep{2pt}
        \caption[MySQL table records]{An excerpt of the data content of the research survey MySQL table \textit{records}.} 
        \label{tab:mysql_records}
        \scalebox{0.9}
        {\begin{tabular}{ @{} >{\raggedright\arraybackslash}p{1.5cm} >{\raggedright\arraybackslash}p{4cm} >{\raggedright\arraybackslash}p{2.5cm} >{\raggedright\arraybackslash}p{2cm} >{\raggedright\arraybackslash}p{1.5cm} >{\raggedright\arraybackslash}p{1.5cm} >{\raggedright\arraybackslash}p{1.5cm} >{\raggedright\arraybackslash}p{1.5cm} >{\raggedright\arraybackslash}p{1.5cm} @{} }
            \toprule
            id & timestamp & ip & zipcode & likert & parkspot & parktime & walktime & timeofday \\
            \midrule
            3245 & 2019-06-06 21:41:21 & wro4qo8hv4 & 00510 & 1 & 4 & 0 & 3 & 1 \\
            3246 & 2019-06-06 21:41:54 & aonm72lyx3 & 00520 & 2 & 1 & 10 & 5 & 1 \\
            3247 & 2019-06-06 21:46:19 & n1982i4i2v & 00100 & 1 & 1 & 20 & 4 & 1 \\
            3248 & 2019-06-06 21:46:22 & sbhfz0uvsl & 00210 & 1 & 1 & 5 & 3 & 3 \\
            3249 & 2019-06-06 21:46:22 & sbhfz0uvsl & 00220 & 2 & 2 & 5 & 5 & 2 \\        
            \bottomrule
        \end{tabular}}
    \end{table} 
\end{hyphenrules}

\begin{hyphenrules}{nohyphenation}
    \begin{table}[H]
        \centering
        \setlength\tabcolsep{1pt}
        \caption[MySQL table visitors]{An excerpt of the data content of the research survey MySQL table \textit{visitors}.} 
        \label{tab:mysql_visitors}
        \begin{tabular}{ @{} >{\raggedright\arraybackslash}p{2cm} >{\raggedright\arraybackslash}p{3cm} >{\raggedright\arraybackslash}p{4cm} >{\raggedright\arraybackslash}p{4cm} >{\raggedleft\arraybackslash}p{1cm} @{} }
            \toprule
            id & ip & ts\_first & ts\_latest & count \\
            \midrule
            1780 & mvovd467a7 & 2019-05-26 15:25:23 & 2019-05-26 15:26:06 & 2 \\
            1781 & xgbgkkzxb3 & 2019-05-26 15:26:23 & 2019-05-26 15:26:23 & 1 \\
            1782 & c9qer4q99a & 2019-05-26 15:27:25 & 2019-05-26 15:27:25 & 1 \\
            1783 & cujhd0hng7 & 2019-05-26 15:27:29 & 2019-05-26 15:27:29 & 1 \\
            1784 & 3ja7gjtko6 & 2019-05-26 15:28:45 & 2019-05-26 15:29:20 & 2 \\        
            \bottomrule
        \end{tabular}
    \end{table} 
\end{hyphenrules}

\begin{hyphenrules}{nohyphenation}
    \begin{table}[H]
        \centering
        \setlength\tabcolsep{1.2ex}
        \caption[Structure of MySQL table records]{The structure of the survey MySQL table \textit{records} fetched with the statement \code{DESCRIBE records;}} 
        \label{tab:mysql_records_str}
        \begin{tabular}{ @{} >{\raggedright\arraybackslash}p{2cm} >{\raggedright\arraybackslash}p{2cm} >{\raggedright\arraybackslash}p{1cm} >{\raggedright\arraybackslash}p{1cm} >{\raggedright\arraybackslash}p{1.5cm} >{\raggedleft\arraybackslash}p{4cm} @{} }
            \toprule
            Field & Type & Null & Key & Default & Extra \\
            \midrule
            id & int(11) & No & PRI & NULL & AUTO\_INCREMENT \\
            timestamp & varchar(19) & Yes & & NULL & \\
            ip & TEXT & Yes & & NULL & \\
            zipcode & varchar(5) & Yes & & NULL & \\
            likert & int(1) & Yes & & NULL & \\
            parkspot & int(1) & Yes & & NULL & \\
            parktime & int(2) & Yes & & NULL & \\
            walktime & int(2) & Yes & & NULL & \\
            timeofday & int(1) & Yes & & NULL & \\
            \bottomrule
        \end{tabular}
    \end{table} 
\end{hyphenrules}

\begin{hyphenrules}{nohyphenation}
    \begin{table}[H]
        \centering
        \setlength\tabcolsep{1.2ex}
        \caption[Structure of MySQL table visitors]{The structure of the survey MySQL table \textit{visitors} fetched with the statement \code{DESCRIBE visitors;}} 
        \label{tab:mysql_visitors_str}
        \begin{tabular}{ @{} >{\raggedright\arraybackslash}p{2cm} >{\raggedright\arraybackslash}p{2cm} >{\raggedright\arraybackslash}p{1cm} >{\raggedright\arraybackslash}p{1cm} >{\raggedright\arraybackslash}p{1.5cm} >{\raggedleft\arraybackslash}p{4cm} @{} }
            \toprule
            Field & Type & Null & Key & Default & Extra \\
            \midrule
            id & int(11) & No & PRI & NULL & AUTO\_INCREMENT \\
            ip & TEXT & Yes & & NULL & \\
            ts\_first & DATETIME & Yes & & NULL & \\
            ts\_latest & DATETIME & Yes & & NULL & \\
            count & int(11) & Yes & & NULL & \\        
            \bottomrule
        \end{tabular}
    \end{table} 
\end{hyphenrules}

The parking survey was released to the public in May 2019 and the active phase of collecting data continued until 30th June 2019. However, the survey remained open after this active period, receiving the last row of data in October 2019. The majority of the respondents were found through Facebook groups. Invitations to participate in the survey were sent to 113 city district and neighborhood groups with a theoretical reach of tens of thousands of people. Of the 113 posts, 63 were Helsinki centric groups, while 23 were based in Espoo, 15 in Vantaa, and 12 in municipalities bordering the Helsinki Capital Region. In addition to these city district and municipal groups, invitations to participate were sent to two other Facebook groups, "Lisää kaupunkia Helsinkiin", a group for city planning ethusiasts in Helsinki, and the \acrshort{gis} profession group "GIS-velhot". In addition to Facebook, an effort was also made to get faculty members of geosciences and geography and students of University of Helsinki to participate in the survey. A small amount of answers were collected with a tweet sent from the Twitter account of Digital Geography Lab. After the initial invitation to participate, reminders were sent to the largest Facebook groups one month after the original posts.

Viewing figure~\ref{fig:thesis_visitors_app}, some response patterns arise when viewing gathered responses as a cumulative chart. It must be acknowledged that it is not possible to conclusively differentiate from which group or city the survey data originated from. This being said, the figure shows that the survey invitation is rapidly buried in the feeds of respective groups and most responses happened immediately after posting the invitations. From the figure \ref{fig:thesis_visitors_app} it may be observed that the Facebook groups have been very effective in gathering responses, while other channels, such as advertisements on Twitter and University of Helsinki email lists have been less so. The invitation posts on Facebook were sent over multiple days to the groups roughly in the following order: 

\begin{displayquote}
    $\text{Espoo}\rightarrow\text{Helsinki}\rightarrow\text{Vantaa}\rightarrow\text{bordering municipalities}\rightarrow\text{reminders to the largest groups}$
\end{displayquote}

\begin{figure}[H]%
    \includegraphics[width=\textwidth]{images/thesis_visitorsvis.png}
    \caption[Received responses viewed over time]{Received survey responses viewed cumulatively over time. This view show the active collection phase in the early summer 2019. Screen capture from the \textit{visitors} analysis application.}%
    \label{fig:thesis_visitors_app}%
\end{figure}

The source code for the survey described in this chapter and step-by-step information to set up an identical system is available at GitHub (\textcolor{blue}{\url{https://github.com/sampoves/parking-in-helsinki-region}}). As a side product, a variant of this survey was created where respondents pick precise points instead of areas. This point-based survey template is, too, available at GitHub (\textcolor{blue}{\url{https://github.com/sampoves/leaflet-map-survey-point}}). The parking survey web application as it was used in this thesis may be tested in the following web address: \textcolor{blue}{\url{https://parking-survey.socialsawblade.fi}}.

\subsubsection{Other approaches that were considered}
\justify

To collect the areal parking data, the study required an interactive survey which respondents could use to submit their parking habits in a spatial fashion. To attract the largest possible number of submissions, the survey also needed to be of modern design, easy to use and its purpose easy to understand. The survey would have to be clear-cut, effortless to internalise and short in length as to prevent users getting frustrated and leaving before submitting answers. Design-wise, the spatial resolution of the survey was in question. The particular concern was that in the case of insufficient amount of answers, what kind of area delineation would be at the same time detailed enough but also streamlined enough to realistically reach results of good quality? This chapter describes the process that would lead to the implemented web survey to accentuate the challenges this kind of research entail. The prototyping of the park survey followed these three steps:

\begin{enumerate}
    \item Review of the available ready-made map survey platforms;
    \item Development of the prototype map survey on Survey123, and;
    \item Programming of the parking survey used in this research.
\end{enumerate}

In the beginning of the process, a review of ready-made map survey platforms was conducted. It quickly became apparent that there were few alternatives available and even fewer free, sufficiently customisable alternatives. Out of the proprietary options, Maptionnaire by the Finnish company Mapita was considered. However, their fee was considered too steep and Maptionnaire was passed on. 

Next Survey123 for ArcGIS was evaluated. An Esri operated service, Survey123 is used to create and analyse form based surveys (\cite{Esri}). It is included in the contract between the University of Helsinki and Esri and thus was free to use for the study. One can quickly design a survey at the Survey123 website and share it immediately to respondents. Alternatively, the service is available as a desktop client, the Survey123 Connect, where Survey123 offers a wider range of possibilities for customisation with its adherence to the XLSForm standard. XLSForm is a standard to make authoring forms in Microsoft Excel easier. With the customisability of XLSForm, users can design Survey123 surveys to the dot while employing the support for Excel style scripting for complex survey behaviour (figure~\ref{fig:survey123_xlsform}). Furthermore, Survey123 provides online tools for collaboration, analysis, and data viewing with many options for exporting the collected data.

\begin{figure}[H]%
    \includegraphics[width=\textwidth]{images/survey123_xlsform.png}
    \caption[Survey123 XLSForm view]{Survey123 XLSForm view in Microsoft Excel. Some parameter columns here are hidden to provide a view to the essential inner workings of the Survey123 form.}%
    \label{fig:survey123_xlsform}%
\end{figure}

At its core, this survey asked respondents for specific parking events in the Helsinki Capital Region they had had (figure~\ref{fig:survey123}). Respondents would pick an exact location on a map view for the location of their parked car and separately on a second map view the location of their final destination they had reached on foot. In addition, respondents would fill the date and time of this parking event, how long it took for them to find this parking spot, how often they had parked to that area, and what kind of a parking spot they had taken. Respondents were asked repeat this process as many times as they had the will to do so.

The Survey123 survey was designed to reach the same spatial resolution as Travel Time Matrix 2018 with its MetropAccess-YKR-grid (abbreviated \textit{grid}, cell dimensions 250 x 250 meters). Using exact coordinates of parkings and final destinations, it would have been possible to allocate each event to possibly two different \textit{grid} cell codes, reaching excellent spatial resolution. As \textit{grid} contains 13 231 cells, there was not enough resources for this master's thesis research survey to accumulate events for every grid cell, or even for most grid cells. If the data gathering campaign had ended with insufficient amount of parking events, the backup plan was to employ an interpolation algorithm to generate approximate boundaries, contour lines, for the hypothetically varying parking search times in the Helsinki Capital Region. It was also considered that the exact coordinates of the parking events could be generalised to other boundaries, such as administrative areas like municipality subdivisions or postal code areas.

% collections of figures utilise the package subfig
\begin{figure}[H]%
    \centering
    \subfloat[Survey introduction and the date and time for the parking event.]{{\includegraphics[width=7.5cm]{survey123_1.png} }}%
    \quad
    \subfloat[Map panels for the parking location and the final destination.]{{\includegraphics[width=7.5cm]{survey123_2.png} }}%
    \quad
    \subfloat[Final questions of the survey.]{{\includegraphics[trim={0 2cm 0 0},width=7.5cm]{survey123_3.png} }}%
    \caption[The unused prototype parking survey created with Survey123]{An example parking event entered into the prototype parking research survey made with Survey123 Connect.}%
    \label{fig:survey123}%
\end{figure}

In January 2019, the prototype parking survey developed with Survey123 was deployed to friends and family, with a large scale marketing push on social media platforms planned for later. This original launch tested the feasibility of the collection of such high resolution public participation GIS data with the highly limited resources available to me. And, indeed, the survey proved itself unwieldy for the purposes of this research. The Survey123 software was difficult to use because of an assortment of inconvenient design choices, unfinished functionality and a helping of software bugs. It was not possible, for example, to have respondents enter multiple parking events at once in a full screen map view. They would have to create a single parking event, send it, and then reload the survey to start from the beginning -- something a majority of prospective respondents would not have the patience for. Survey123 Connect version available at the time, 3.1.126, did not allow customisation of the post-submission message and therefore it would not be possible to efficiently direct respondents back to the form. In addition, recording coordinates from two map views was only possible through a bypass. The coordinates of the final destination would have to be printed on the form (hence the section "Coordinates" on the form in figure~\ref{fig:survey123}) and then these second set of coordinates could be saved into the survey data table in string format. The technical limitations of Survey123 as a spatial survey were witnessed also in the fact that it was not possible to add custom polygons on top of the map views. It was therefore impossible to delineate the research area for the respondents and accurately detect attempts to add parking events outside of the Helsinki Capital Region.

The functionality of the survey form was not reliable on the most popular web browsers such as Google Chrome and Apple Safari. Survey123 supported multi-language strings but it proved problematic to ensure that the form would open in the system language of most respondents, Finnish. In addition to this, the field for entering the specific time for the parking event was restricted to the 12 hour clock preferred in the United States -- a time convention the target group of this thesis would frown upon. \textcolor{red}{liian detail? --> }To make matters worse, at that time there was a long persisting bug in Survey123 which produced unexpected behaviour, in some cases, with the use of \code{constraint}, the parameter that controls which entered values are deemed illegal and which are not (\cite{GeoNet-TheEsriCommunity2018}). If any type of constraint statement was added, the finalised form would always claim that the related question input was invalid. The parameter would have to be left empty and therefore it was not possible to automatically prevent insertion of parking events happening in the future and excessively long times for searching for parking, reducing the quality of the survey data and making the survey form more confusing for the respondent.

\begin{figure}[H]%
    \includegraphics[width=\textwidth]{images/survey123_dataview.png}
    \caption[The Survey123 web application dashboard]{Survey123 for ArcGIS, "data" tab view on the application website. The prototype research survey made with Survey123 received in total 104 parking events. The red dots are the final destinations of each parking event.}%
    \label{fig:survey123_dataview}%
\end{figure}

Despite the many technical uncertainties of Survey123, the prototype survey gathered more than one hundred parking events in one month (figure~\ref{fig:survey123_dataview}). This amount was achieved for the most part without advertising. Soon after the publication of the Survey123 parking survey it was decided, however, that the spatial resolution for this research would need to be lower than exact points in an attempt to gather more responses from the entire research area. An additional deciding factor was the fact that with Survey123, respondents could not send multiple parking events with one survey session, making the form unwieldy and outdated in its rigid structure. It was argued that a more general scale would still be accurate enough to provide good data and a more generalised scale would make the survey easier to answer to and a more pleasant experience for the respondent. Postal code areas were deemed an acceptable compromise in spatial detail.

After careful consideration, it was decided that the actual survey for this thesis would be programmed from the ground up.

\newpage
\subsection{Processing survey data}
\label{sec:c3-processdata} % labeling to enable hyperref to this chapter
\justify
%\begin{itemize} %processingin tärkeimmät kohdat
%    \item anonymisation of ip addresses
%    \item Read in spatial data sources
%    \item Read in survey data
%    \item Prepare source data (convert formats, remove some irregular erroneous answers from dataset)
%    \item Prepare shape files (remove islands not reachable by car)
%    \item give grid cells zipcodes (ykr grid does not have those of-the-shelf. Develop method to assign all cells zipcodes, take into account water and grid cells which are outside of research area)
%    \item respondent behaviour (see how each user has answered)
%    \item detect illegal data (first detect duplicate answers, produce report. Then remove data where parktime and/or walktime is 60 or over)
%    \item Add data to geodataframes (add columns for ykr\_vyoh, ua-forest, answer count, parktime and walktime mean
%    \item show statistics to user
%    \item Set percentage of urban zones and forest in each zipcode area (choose one urban zone and forest amount (jenks breaks) for every zipcode)
%    \item add subdivisions to data (all answer row gets corresponding subdivision value)
%    \item EXPERIMENTAL utilise travel-time matrix 2018, make comparisons
%    \item EXPERIMENTAL somehow create my own TTM18, with updated values
%    \item export results to R
%\end{itemize}

In this section, various data are refered to with abbreviated names as this makes it easier to follow the data processing workflow. Please see table~\ref{tab:used_data} for the key.

The main objective of the thesis data processing was to merge survey responses dataset (\textit{records}) with selected spatial data and prepare \textit{records}, survey visits dataset (\textit{visitors}), PAAVO postal code areas dataset (\textit{postal}), and MetropAccess-YKR-grid (\textit{grid}) for later analysis in R programming language environment. Using a selection of open spatial data (table~\ref{tab:used_data}), new explanatory variables would be available for use in the analysis. This opened opportunities to compare the newly gathered survey data against that in Helsinki Travel Time Matrix 2018 (\textit{TTM}).

As the first step in the survey data processing, all IP addresses were anonymised and replaced with identifiers of ten characters consisting of numbers 0--9 and letters of English alphabet (figure~\ref{fig:gen_workflow}, section 1). The anonymisation was carried out in such a way that the random identifiers for respondents matched in both \textit{records} and \textit{visitors}, preserving the possibility to associate survey responses with survey visits. 

The data processing proper started with loading the open spatial data presented in table~\ref{tab:used_data} and selecting only areas and attribute data relevant to the research (figure~\ref{fig:gen_workflow}, section 2). For \textit{CORINE}, this meant selecting only areas marked Level1Eng, Artificial surfaces. \textit{YKR zones}, a dataset that covers the entirety of Finland, was clipped with the spatial dimensions of \textit{postal} with an additional 500 meter buffer. \textit{postal} was processed to only include areas reachable by car from the mainland (figure~\ref{fig:paavo_resarea}). Islands not reachable by car were approximated visually using Google Maps and were removed from the data. However, some islands in the Helsinki Capital Region are technically accessible with a car from the mainland, but in practice the access is limited. In these cases, deliberation was used. For example, Suomenlinna islands and Korkeasaari were kept in the data. Conversely, some technically car-accessible islands like Staffan in Espoo, and Mustasaari and Seurasaari in Helsinki were removed from the data with the grounds of them containing only private property, or no public parking spaces. By removing the islands unreachable by car, we evade a source of uncertainty that could arise in the data analysis -- it would be misleading, for example, to include forested, inaccessible islands in the calculation of artificial surfaces carried out for each postal code area.

\begin{figure}[H]%
    \centering
    \subfloat[Unedited postal code areas for the Helsinki Capital Region.]{{\includegraphics[width=8.1cm]{resarea_unedited.png} }}%
    \quad
    \subfloat[Postal code areas of the Helsinki Capital Region, with islands unreachable by car removed.]{{\includegraphics[width=8.1cm]{resarea_edited.png} }}%
    \caption[Process to remove islands not reachable by car]{Islands unreachable by car were removed from the PAAVO postal code area dataset in the Python data processing.}%
    \label{fig:paavo_resarea}%
\end{figure}

Helsinki Region Travel Time Matrix 2018 and the survey data of this thesis operate in different spatial units. Travel time Matrix 2018 uses the MetropAccess-YKR-grid (\textit{grid}), a spatial dataset based on the Statistics Finland statistical grid with the cell size of 250 x 250 meters. The basic spatial unit of the survey data is the postal code area based on PAAVO open data (\textcolor{red}{lisää lähteet}). Using Python, postal codes were added to each \textit{grid} cell with the logic that the largest area in \textit{postal} (figure~\ref{fig:paavo_ykr}) assigns the postal code in each \textit{grid} cell. \textit{postal} polygons do not always intersect with the cells of \textit{grid} and because of this some cells were assigned a postal code of 99999 to denote missing data (\textcolor{red}{mieti vielä 99999:n käyttö}). As a side product of this postal code assignment, \textit{grid} was merged with data which tells how much of a cell was contained in the research area (\textit{postal}) and how large was the largest postal code area which dictated the postal code assignment of the current cell. \textcolor{red}{varmista että lukija ymmärtää PAAVO spatial datan ja tutkimusalueen yhteyden}

\begin{figure}[H]%
    \includegraphics[width=\textwidth]{images/paavo-ykr.png}
    \caption[Assigning MetropAccess-YKR-grid postal codes]{The MetropAccess-YKR-grid cell 6002625 (marked with the red square) is assigned postal code 02360 because in that grid cell, the largest segment of PAAVO open data (coloured warm purple and yellow) belongs in the postal code 02360 Soukka. \textcolor{red}{osm cite}}%
    \label{fig:paavo_ykr}%
\end{figure}

The data processing script created for this thesis contains detailed features to detect patterns in the survey data (figure~\ref{fig:gen_workflow}, section 3). To enhance pattern recognition, \textit{records} and \textit{visitors} were purged of known false data, which were namely responses and visits made by me.

The data processing script creates two distinct reports about \textit{records}. Firstly, the data processing script aggregates \textit{records} by IP address code, resulting in an Excel file where one row represents each respondent. It is then possible to review the behaviour of each respondent in detail. In addition to this report, the data processing script writes a text file report about IP address codes which submitted multiple responses from the same postal code area. The text file report also identifies whether the duplicate responses for each postal code area per each IP address code have identical values or if they have changed between responses. These two reports were used to determine what to do about the duplicates and values which appear anomalous.

It was decided that if the parking time or walking time value in a \textit{records} row was 60 minutes or greater, that data row would be deleted. This value is arbitrary. The research assumes that it is highly unlikely that anybody would generally park 60 minutes away from their final destination to which they would then proceed on foot. A hour of searching for parking is plausible in the center of Helsinki but because of its unlikeliness the same 60 minutes limit was utilised in searching for parking. It is not possible to determine why multiple survey responses contain the maximum value for parktime and walktime, 99, but it can not be ruled out that these data rows are protest votes meant to declare that reliable parking is hard to find in certain parts of the Helsinki Capital Region. When advertising the thesis survey on Facebook, some people took the opportunity to voice their displeasure at the perceivedly difficult parking conditions in the Helsinki Capital Region. In conclusion, even though the Python script has the capability to delete data rows deemed illegal, all of the illegal data in \textit{records} was preserved a more versatile analysis in R.

\textcolor{red}{lisää esimerkkikuva siitä miltä postal näyttää?}\\
Next in the survey data processing workflow additional spatial data, variables \code{ykr\_zone} and \code{articifial}, was added to \textit{postal} (figure~\ref{fig:gen_workflow}, section 4). \textcolor{red}{SEMI TURHA LAUSE: Utilising \textit{records}, answer count, and means and medians for \code{parktime} and \code{walktime} were produced for all postal code areas.} \textit{YKR zones} data was first simplified with the notation presented by the research group at the websites of the zones of urban structure (table~\ref{tab:ykr_zones_simplify}, \cite{FinnishEnvironmentInstitute2013}). Then, the percentage shares of each \textit{YKR zones} class in the postal code area were calculated. The zone with the largest percentage value was chosen for each postal code area to finalise the explanatory variable of \code{ykr\_zone}.

Using \textit{CORINE} data, the percentage of artificial surface in each postal code area was calculated. These values were then used to determine class breaks (Jenks natural breaks) for the explanatory variable \code{artificial} (table~\ref{tab:artificial_jenks_breaks}, figure~\ref{fig:postalvis_artificial}).

\begin{hyphenrules}{nohyphenation}
    \begin{table}[H]%
        \centering
        \def\arraystretch{1.2}
        \setlength\tabcolsep{1.2ex}
        \caption[YKR zones processing]{The logic by which the unedited source data for zones of urban structure was transformed for this thesis.}
        \label{tab:ykr_zones_simplify}
        \scalebox{0.9}
        {\begin{tabular}{ @{} >{\raggedright\arraybackslash}p{6cm} >{\raggedright\arraybackslash}p{6cm} @{} }
            \toprule
            Original definition & Definition for this thesis \\
            \midrule
            Keskustan jalankulkuvyöhyke & Keskustan jalankulkuvyöhyke \\
            \greyrule
            % Manually set the position of multirow label
            Keskustan reunavyöhyke & \multirow{3}{*}[-4.5ex]{Keskustan reunavyöhyke} \\
            Keskustan reunavyöhyke/intensiivinen joukkoliikenne & \\
            Keskustan reunavyöhyke/joukkoliikenne & \\
            \greyrule
            Alakeskuksen jalankulkuvyöhyke & \multirow{3}{*}[-4.5ex]{Alakeskuksen jalankulkuvyöhyke} \\
            Alakeskuksen jalankulkuvyöhyke/intensiivinen joukkoliikenne & \\
            Alakeskuksen jalankulkuvyöhyke/joukkoliikenne & \\
            \greyrule
            Intensiivinen joukkoliikennevyöhyke & Intensiivinen joukkoliikennevyöhyke \\ 
            \greyrule
            Joukkoliikennevyöhyke & Joukkoliikennevyöhyke \\
            \greyrule
            Autovyöhyke & Autovyöhyke \\
            \greyrule
            \textit{Areas not in the YKR zones data} & novalue \\
            \bottomrule
        \end{tabular}}
    \end{table} 
\end{hyphenrules}

\begin{figure}[H]%
    \centering
    \includegraphics[width=\textwidth]{images/thesis_postalvis_ykrzone.png}
    \caption[Calculated zones of urban structure in the research area]{Each postal code area was assigned to belong to a group in the variable \code{ykr\_zone}, explained in table~\ref{tab:ykr_zones_simplify}, according to which group was largest in area in any given postal code area.\textcolor{red}{need larger font, remove decimal}}%
    \label{fig:postalvis_ykrzone}%
\end{figure}

\begin{figure}[H]%
    \centering
    \includegraphics[width=\textwidth]{images/thesis_postalvis_artificial.png}
    \caption[Calculated zones of built surfaces in the research area]{Artificial surfaces are based on CORINE Land Cover 2018.\textcolor{red}{need larger font, remove decimal, move scalebar away from delineation}}%
    \label{fig:postalvis_artificial}%
\end{figure}

\begin{hyphenrules}{nohyphenation}
    \begin{table}[H]
        \centering
        \def\arraystretch{1.2}
        \setlength\tabcolsep{1.2ex}
        \caption[CORINE data levels]{CORINE land cover 2018 data hierarchy under attribute data column Level1Eng, Artificial surfaces.}
        \label{tab:corine_artificial}
        \scalebox{0.85}
        {\begin{tabular}{ @{} >{\raggedright\arraybackslash}p{4cm} @{} >{\raggedright\arraybackslash}p{4cm} @{} >{\raggedright\arraybackslash}p{4.25cm} @{} >{\raggedright\arraybackslash}p{4cm} @{} }
            \toprule
            Level1, Level1Eng & Level2, Level2Eng & Level3, Level3Eng & Level4, Level4Eng \\
            \midrule
            \multirow{16}{4cm}[-15ex]{1 Artificial surfaces} & \multirow{2}{4cm}[-4ex]{11 Urban fabric} & 111 Continuous urban fabric 
 & 1111 Continuous urban fabric \\
            \arrayrulecolor{black!30}\cmidrule(lr){3-4}
            & & 112 Discontinuous urban fabric & 1121 Discontinuous urban fabric \\
            \arrayrulecolor{black!30}\cmidrule(lr){2-4}
            & \multirow{2}{4cm}[-0.5ex]{12 Urban fabric} & \multirow{2}{4cm}{121 Industrial or commercial units} & 1211 Commercial units \\
            \arrayrulecolor{black!30}\cmidrule(lr){4-4}
            & & & 1212 Industrial units \\
            \arrayrulecolor{black!30}\cmidrule(lr){2-4}
            & \multirow{3}{4cm}[-4ex]{12 Industrial, commercial and transport units} & 122 Road and rail networks and associated land & 1221 Road and rail networks and associated land \\
            \arrayrulecolor{black!30}\cmidrule(lr){3-4}
            & & 123 Port areas & 1231 Port areas \\
            \arrayrulecolor{black!30}\cmidrule(lr){3-4}
            & & 124 Airports & 1241 Airports \\
            \arrayrulecolor{black!30}\cmidrule(lr){2-4}
            & \multirow{4}{4cm}[-4ex]{13 Mine, dump and construction sites} & \multirow{2}{4cm}[-2.5ex]{131 Mineral extraction sites} & 1311 Mineral extraction sites \\
            \arrayrulecolor{black!30}\cmidrule(lr){4-4}
            & & & 1312 Open cast mines \\
            \arrayrulecolor{black!30}\cmidrule(lr){3-4}
            & & 132 Dump sites & 1321 Dump sites \\
            \arrayrulecolor{black!30}\cmidrule(lr){3-4}
            & & 133 Construction sites & 1331 Construction sites \\
            \arrayrulecolor{black!30}\cmidrule(lr){2-4}
            & \multirow{5}{4cm}[-2ex]{14 Artificial, non-agricultural vegetated areas} & 141 Green urban areas & 1411 Green urban areas \\
            \arrayrulecolor{black!30}\cmidrule(lr){3-4}
            & & \multirow{4}{4cm}[-3ex]{142 Sport and leisure facilities} & 1421 Summer cottages \\
            \arrayrulecolor{black!30}\cmidrule(lr){4-4}
            & & & 1422 Sport and leisure areas \\
            \arrayrulecolor{black!30}\cmidrule(lr){4-4}
            & & & 1423 Golf courses \\
            \arrayrulecolor{black!30}\cmidrule(lr){4-4}
            & & & 1424 Race courses \\
            \bottomrule
        \end{tabular}}
    \end{table} 
\end{hyphenrules}

\begin{hyphenrules}{nohyphenation}
    \begin{table}[H]%
        \centering
        \def\arraystretch{1.2}
        \setlength\tabcolsep{1.2ex}
        \caption[Built surfaces Jenks breaks classes]{Using a custom algorithm, the variable \code{artificial} was divided to classes using Jenks natural breaks method.}
        \label{tab:artificial_jenks_breaks}
        \scalebox{0.9}
        {\begin{tabular}{ @{} >{\raggedright\arraybackslash}p{2.5cm} >{\raggedright\arraybackslash}p{3cm} >{\raggedright\arraybackslash}p{2.5cm} @{} }
            \toprule
            \code{artificial} value & Description & Amount of postal code areas \\
            \midrule
            < 100 \% & Fully built & 77 \\
            < 89.6 \% & Predominantly built & 39 \\
            < 69.0 \% & Moderately built & 28 \\ 
            < 44.3 \% & Some built & 14 \\
            < 16.6 \% & Scarcely built & 9 \\
            \bottomrule
        \end{tabular}}
    \end{table}
\end{hyphenrules}

\textcolor{red}{histogram noista \code{artificial}-valueista? (pythonissa yks versio, R:ssä helppo classIntervals(postal\$artificial, 5, style = "jenks"))}

% https://www.spatialanalysisonline.com/HTML/index.html?classification_and_clustering.htm
In the finalising section, \textit{records} was prepared for analysis and visualisation in R (figure~\ref{fig:gen_workflow}, section 5). The software library for plotting in R, \textit{ggplot2}, prefers data inputted in long format. To study charasteristics of postal code areas in this research, it meant adding repetitive data columns in \textit{records}, where values for CORINE land cover 2018 artificial surfaces, YKR zone and subdivision remained unchanged for all rows in the same postal code area. For artificial surfaces, a custom Jenks natural breaks function (GitHub user Drewda, \textcolor{red}{add cite, add jenks cite, kato kommenttilinkki}) with five classes were utilised to find the applicable Jenks breaks class for each postal code area. For YKR zones, the most common urban structure type in percentage was selected for each postal code area. In addition, \textit{records} was inserted with municipality subdivision information (figure~\ref{fig:subdiv_placement}). This was achieved by collecting data from the web sites of the municipalities of the Helsinki Capital Region (\cite{Espoonkaupunki2020}, \cite{Helsinginkaupunkiymparistontoimiala2019}, \cite{Vantaankaupunki2019}). In these sources, each municipality broke the subdivisions down to city district level, from where it was possible to allot each postal code area with a subdivision. This was for the most part simplistic work, but in some cases the postal code areas and city districts did not align and author's own deliberation was used to help the placement. Some of the most glaring discrepancies between PAAVO postal code areas and subdivision boundaries occur in Espoo. In the case of Lippajärvi-Järvenperä, a postal code area north of Kauniainen, the subdivision Vanha-Espoo was chosen because Lippajärvi-Järvenperä as a whole does not fit into the charasteristics of Suur-Leppävaara, and at the same time the city districts Lippajärvi and Järvenperä do not fit into the distinctive features of the subdivision Pohjois-Espoo. In the same spirit the postal code area Sepänkylä-Kuurinniitty south of Kauniainen lies troublingly in the area of four subdivisions of Espoo. In the end Vanha-Espoo was chosen as Sepänkylä-Kuurinniitty lies for the most part in its area. Similar complications occurred in Helsinki and Vantaa (the partial placement of Kirkonkylä-Veromäki and Ruskeasanta-Ilola in subdivision of Tikkurila) and using my best judgement, the classification shown in figure~\ref{fig:subdiv_placement} was used in the survey results analysis of this thesis.

The source code for the data processing described in this chapter is available at GitHub (\textcolor{blue}{\url{https://github.com/sampoves/Msc-thesis-data-analysis}}).

\begin{figure}[H]%
    \includegraphics[width=\textwidth]{images/thesis_subdiv_place.png}
    \caption[Placing postal code areas in subdivisions]{For the purposes of analysis in R, all postal code areas in \textit{postal} were assigned with subdivision information. In this figure, distinct colors depict the postal code areas with the subdivision classification chosen for this thesis.}%
    \label{fig:subdiv_placement}%
\end{figure}

\newpage
\subsection{Creating applications and conducting analyses}
\subsubsection{Analysis application}
\justify
%\begin{itemize}
%    \item Prepare data to R compliant format
%    \item ShinyApp descriptive statistics
%    \item shinyapp histogram for parktime and walktime
%    \item shinyapp boxplot, show outliers
%    \item shinyapp barplot, show amounts
%    \item shinyapp levene test
%    \item shinyapp one-way anova
%    \item shinyapp map, nice to have, not at all important
%    \item visitor shinyapp, see the accumulation of visits and received records
%\end{itemize}

\textcolor{red}{kerro paremmin mitä nämä analysisin työkalut ovat ja miksi valittu tähän appiin!}
Once the data processing in Python was completed, \textit{records} and \textit{visitors} were carried over to R to utilise its easy to access statistical analysis functionality. For this thesis, this meant namely packages \textit{onewaytests} for ANOVA and Brown-Forsythe test, \textit{plotrix} for standard error, and \textit{moments} for quantiles (table~\ref{tab:used_soft}). To help study the large datasets, three Shiny applications were written (Shiny is a web application framework software package for R), one for \textit{records} and a second for \textit{visitors}, and a third one to study differences between the thesis survey results and Helsinki Region Travel Time Matrix 2018. Benefits in creating these applications were twofold. Firstly, approaching the survey results from an interactive perspective allowed countless combinations of active and inactive variables -- without constant tweaking of code -- which would be beneficial for the analysis of \textit{records}. Secondly, programming the applications using Shiny enabled the use of shinyapps.io, a service where one can host Shiny applications on the internet without charge. Combination of these two factors made it effortless to analyse results of the survey in a visual way and at the same time, publish the tools and results to the public, upholding the thesis' mission of openness and transparency.

In the Shiny analysis application for \textit{records}, users can view the survey responses from many different angles (figure~\ref{fig:shinyapps_analysis}). Users are given control which variables are active at any moment. Users control the variables through the side panel, with settings taking effect in the main panel. The variables currently viewed are selected through two dropdown menus, Response (continuous) and Explanatory (ordinal). Continuous variables are \code{parktime} and \code{walktime} with an integer range 0--99. Available ordinal variables are \code{likert}, \code{parkspot}, \code{timeofday}, \code{artificial}, \code{ykr\_zone}, and \code{subdiv} with the values that can not be unequivocally ordered in a sequence in the same way as continuous variables. One variable from each variable group can be selected at the same time. Any and all groups of values in the ordinal variables can be deactivated to better understand the significance of each value group. In addition to the selection of the continuous and ordinal variable, users can deactivate \textit{records} data rows based on their spatial location in municipality subdivisions assigned in \hyperref[sec:c3-processdata]{\fullref{sec:c3-processdata}}. Most importantly, the analysis application allows selection of maximum allowed value for \code{parktime} and \code{walktime}. The default value for both is set at 59 minutes, as discussed in the \fullref{sec:c3-processdata}, but the user is free to choose any value between zero and 99.

\begin{figure}[H]%
    \centering
    \includegraphics[width=.75\textwidth]{images/shinyapps_analysis.png}
    \caption[Records analysis application screen capture]{A segment of the shinyapps.io deployment of \textit{records} analysis application. The web application provides a wide array of analysis and visualisation tools for the results of the thesis survey research. The application is available for use at \textcolor{blue}{\url{https://sampoves.shinyapps.io/records}}.}%
    \label{fig:shinyapps_analysis}%
\end{figure}

% levene, anova, boxplot, lue: https://www.itl.nist.gov/div898/handbook/eda/section3/eda35a.htm. On legit lähde
\begin{table}[H]
    \centering
    \caption[Records Shiny application features]{\textit{records} Shiny application features. All features are affected by the maximum permitted \code{parktime} and \code{walktime} values, currently active response and explanatory variables and inactive subdivisions. In addition, certain exclusive settings are found in some of the features.}
    \label{tab:records_shiny_features}
    \scalebox{0.8}
    {\def\arraystretch{1.3}
    \setlength\tabcolsep{1.2ex}
    \begin{tabular}{ @{} >{\raggedright\arraybackslash}p{3cm} >{\raggedright\arraybackslash}p{2cm} >{\raggedright\arraybackslash}p{6cm} >{\raggedright\arraybackslash}p{6cm} @{} }
        \toprule
        Feature & Type & Outputs & Feature exclusive settings \\
        \midrule
        1 Descriptive statistics & Analysis, table & n, median, mean, standard deviation, standard error, confidence interval for mean, lower bound, confidence interval for mean, min, max, 25th quartile, 75th quartile, skewness, kurtosis & None \\
        2 Histogram & Analysis, chart & Histogram, kernel density estimate, mean, median & Histogram binwidth \\
        3 Distribution of ordinal variables & Analysis, chart & Distribution plot by explanatory variable value group & Explanatory variable for the distribution plot Y axis \\
        4 Boxplot & Analysis, chart & Quartile data & None \\
        5 Test of homogeneity of variances (Levene's test) & Analysis, table & Equality of variances for a variable calculated for the currently active response and explanatory variable & None \\
        6 Analysis of variance (ANOVA) & Analysis, table & Analysis of differences among group means in a sample & None \\
        7 Brown-Forsythe test & Analysis, table & Analysis of equality of group variances & None \\
        8 Interactive map & Visualisation, map & Choropleth map with Jenks breaks classifiction, descriptive data per postal code area (answer count, mean and median for parktime and walktime, forest amount percentage, largest YKR zone percentage) & - Selection of active municipalities \linebreak - Jenks breaks parameter column \linebreak - Amount of Jenks breaks classes \linebreak - Possibility to visualise the map with boundaries and labels \\
        \bottomrule
    \end{tabular}}
\end{table} 

When the user has selected a continuous and an ordinal variable to compare, they are presented a thorough set of descriptive statistics for the currently active data rows with n, median, mean, standard deviation, standard error, confidence interval for lower and upper bound, minimum and maximum, 25 \% and 75 \% quantiles, skewness, and kurtosis (table~\ref{tab:records_shiny_features}). For the continuous variables, a histogram is available to visualise the distribution of \code{walktime} and \code{parktime}. Distribution of ordinal variables \code{likert}, \code{parkspot}, and \code{timeofday} can be compared against other ordinal variables in a barplot. To study quartiles, a boxplot is available. Importantly, users can test their selection of variables with the test of homogeneity of variables (Levene's test), analysis of variance (ANOVA), and the Brown-Forsythe test \textcolor{red}{laajenna selityksiä kappaleen alun kommentin mukaisesti}. Lastly, a versatile interactive map of the research area is provided. This map, divided in postal code areas, reveals the survey results in a spatial fashion. The interactive map is affected by the maximum parking time and walking time, selection of an ordinal variable and any inactive subdivisions to provide a flexible view into the details of the data. In addition, this interactive map is controlled by some exclusive settings of its own. The interactive map settings offers six distinct parameters for viewing the research area through Jenks natural breaks classification, alongside with the possibility to select the amount of classes in the map view. Hovering the cursor over the map reveals a tooltip which the application users can use to view mean, median, and percentage data about each postal code area in the Helsinki Capital Region. Tooltips are also available for the barplot of distribution of ordinal variables and the boxplot.

Much additional work was put into the analysis application to make it as clear and easy to use as possible. The application features a number of links to move between the features and the settings, while smooth scrolling and animations help in directing the attention of the user. Each application feature can be switched on and off to make space for exactly the topic the user wants to examine. The analysis application allows downloading all the results, outputting tables into comma separated value files (CSV). Charts and the map are outputted into high resolution images (PNG). The files are intuitively named informing of the used application settings and the date of file download. Attention was given to ensure the usage of the analysis application on mobile phones. To this end, the CSS style sheet of the application detects mobile phone screen sizes and adjusts the application content accordingly. The sidebar tends to block the view of the main panel on mobile screens and for this situation a switch is provided to hide the sidebar at any given time. All graphical elements of the application are in SVG (Scalable Vector Graphics) format which supports effortless zooming without loss of detail.

The source code for the \textit{records} analysis application is available at GitHub (\textcolor{blue}{\url{https://github.com/sampoves/thesis-records-shinyapps}}). The application may be viewed on shinyapps.io (\textcolor{blue}{\url{https://sampoves.shinyapps.io/records}}).

\subsubsection{Visitors application}

In the Shiny application for \textit{visitors}, users can examine events in the timeline of the survey research (figure~\ref{fig:shinyapps_visitors}). In this interactive view, cumulative charts are presented for received survey responses and survey page first visits. The charts reveal the effect and importance of advertisement on actual received responses and survey traffic. While not completely verifiable, the significance of different sources of responses can be viewed in the application.

\begin{figure}[H]%
    \centering
    \includegraphics[width=.75\textwidth]{images/shinyapps_visitors.png}
    \caption[Visitors analysis application screen capture]{The shinyapps.io deployment of \textit{visitors} analysis application. In this web application users may examine how the amounts of submitted responses and unique first visits to the thesis web survey developed over time. The application is available for use at \textcolor{blue}{\url{https://sampoves.shinyapps.io/visitors}}.}%
    \label{fig:shinyapps_visitors}%
\end{figure}

Compared to the other two analysis applications programmed for this thesis, the \textit{visitors} application is relatively simple in its function and features. The user controls the chart view with mouse button presses or dragging the cursor and no additional settings are provided.

The source code for the \textit{visitors} analysis application is available at GitHub (\textcolor{blue}{\url{https://github.com/sampoves/thesis-visitors-shinyapps}}). The application may be viewed on shinyapps.io (\textcolor{blue}{\url{https://sampoves.shinyapps.io/visitors}}).

\subsubsection{Travel time comparison application}

Despite potential for extensive analysis, the applications described in previous chapters do not provide means to study the third research question of this thesis:

\begin{displayquote}
    III What is the significance of the parking process to the overall travel time?
\end{displayquote}

To answer this research question, an application to compare travel time datasets was programmed (figure~\ref{fig:shinyapps_comparison}). In this application, the user can view a variety of descriptive values calculated from Helsinki Region Travel Time Matrix 2018, the thesis survey data, and a dataset created by comparing the two datasets. The user is given control a set of features, such as a selection of the travel times origin postal area code, the parameter to visualise on the map, and the amount of classes to show on the map. The map view can be customised with a number of additional spatial data for the purpose of visualisation, such as regional boundaries and physical features (inland water, main roads), and options for the labelling of postal code areas. The application supports downloading any map view in high resolution png format image.

\begin{figure}[H]%
    \includegraphics[width=\textwidth]{images/shinyapps_comparison.png}
    \caption[Comparison application screenshot]{Helsinki Region Travel Time Matrix 2018 and thesis survey data comparison application's shinyapps.io deployment. The cursor hovers over 01200 Hakunila with the data tooltip active. The application is available for use at \textcolor{blue}{\url{https://sampoves.shinyapps.io/comparison}}.}%
    \label{fig:shinyapps_comparison}%
\end{figure}

It was decided that the basic spatial unit for the comparison application would be the PAAVO postal code area as the thesis survey results exist in that resolution. This decision necessitated extensive processing of \textit{TTM} data. Firstly, the application needed to be able to recalculate the map view as quickly as possible. Secondly, the original \textit{TTM} dataset is unwieldy to be used in its original format in a web application (data stored in uncompressed txt files, data scope much too detailed for the application). Thirdly, the hosting service shinyapps.io places technical limitations on the resource intensity of the application. For these three main reasons, the library \textit{fst} (table~\ref{tab:used_soft}) was first used to convert the \textit{TTM} private car columns into the data format used by the library (table~\ref{tab:comparison_fst1}), and then using the dataset \textit{grid} preprocessed in Python to aggregate all \textit{TTM} grid cell values to the PAAVO postal code area level and writing the results using the optimised \textit{fst} format (table~\ref{tab:comparison_fst2}). For data completeness, \textit{TTM} searching for parking (0.42 minutes) and walking to destination (2.0--2.5 minutes) data was added to these aggregated \textit{TTM} files. This aggregation method for \textit{TTM} data drastically reduces unnecessary real-time processing, minimises disk space needed, and keeps application memory footprint in manageable figures for the deployment of the application to the internet. The other main dataset used in the comparison application, the thesis survey data, is aggregated each time the application initialises, as the complete survey results dataset is miniscule in size compared to \textit{TTM}.

\begin{hyphenrules}{nohyphenation}
    \begin{table}[H]
        \centering
        \setlength\tabcolsep{1pt}
        \caption[Comparison application fst structure I]{An excerpt of the data content of \textit{TTM} converted to fst format for further processing (table~\ref{tab:comparison_fst2}). Original file 5785xxx/travel\_times\_to\_ 5785640.txt.} 
        \label{tab:comparison_fst1}
        \begin{tabular}{ @{} >{\raggedright\arraybackslash}p{1cm} >{\raggedright\arraybackslash}p{2cm} >{\raggedright\arraybackslash}p{2cm} >{\raggedright\arraybackslash}p{2cm} >{\raggedright\arraybackslash}p{2cm} >{\raggedright\arraybackslash}p{2cm} @{} }
            \toprule
            & from\_id & to\_id & car\_r\_t & car\_m\_t & car\_sl\_t \\
            \midrule
            10 & 5787549 & 5785640 & 22 & 21 & 16 \\
            11 & 5787550 & 5785640 & 22 & 21 & 16 \\
            12 & 5789447 & 5785640 & 10 & 9 & 8 \\
            13 & 5789448 & 5785640 & 10 & 9 & 8 \\
            14 & 5789449 & 5785640 & 11 & 10 & 9 \\
            \bottomrule
        \end{tabular}
    \end{table} 
\end{hyphenrules}

\begin{hyphenrules}{nohyphenation}
    \begin{table}[H]
        \centering
        \setlength\tabcolsep{5pt}
        \caption[Comparison application fst structure II]{An excerpt of the data content of \textit{TTM} aggregated to postal code area level for the use of the comparison application. The origin postal code area of the shown data table is 00100 Helsinki Keskusta -- Etu-Töölö.} 
        \label{tab:comparison_fst2}
        \begin{tabular}{ @{} >{\raggedright\arraybackslash}p{1cm} >{\raggedright\arraybackslash}p{1.5cm} >{\raggedright\arraybackslash}p{1.5cm} >{\raggedright\arraybackslash}p{2cm} >{\raggedright\arraybackslash}p{2cm} >{\raggedright\arraybackslash}p{2cm} >{\raggedright\arraybackslash}p{2cm} >{\raggedright\arraybackslash}p{2cm} @{} }
            \toprule
            & zipcode & from\_zip & ttm\_r\_avg & ttm\_m\_avg & ttm\_sl\_avg & ttm\_wtd & ttm\_sfp \\
            \midrule
            5 & 00150 & 00100 & 14.89 & 13.24 & 9.48 & 2.33 & 0.42 \\
            6 & 00160 & 00100 & 15.48 & 13.97 & 9.77 & 2.50 & 0.42 \\
            7 & 00170 & 00100 & 14.41 & 13.10 & 9.09 & 2.50 & 0.42 \\
            8 & 00180 & 00100 & 12.77 & 11.34 & 8.40 & 2.50 & 0.42 \\
            9 & 00190 & 00100 & 24.27 & 22.31 & 17.04 & 2.00 & 0.42 \\
            \bottomrule
        \end{tabular}
    \end{table} 
\end{hyphenrules}

The survey data gathered for this thesis does not contain any additional data about the driving segment of travel chains in the Helsinki Capital Region. To gain this information, all \textit{grid} cells were associated with a postal code area to enable the aggregation of \textit{TTM} data by that variable. Then, total travel times were extracted from every postal code area to all other postal code areas using the newly aggregated \textit{TTM} data, and substracting those values with the length of the \textit{TTM} parking process (2.42--2.92 minutes). These driving time segment values were then added up with the thesis survey \code{parktime} and \code{walktime} data, creating fully comparable travel chains from the realistic durations of \textit{TTM} data and the newly collected parking survey data. In the travel time comparison application the user can access all calculated values by hovering the cursor over a postal code area (figure~\ref{fig:shinyapps_comparison_detail}). The calculation formulas and abbreviated column names are further explained in table~\ref{tab:comparison_tooltip_content}.

With the help of this travel time comparison application it is effortless to get an overall picture of private car parking time charasteristics in the Helsinki Capital Region and most importantly, to answer the third research question of this thesis, view the share of the driving segment and parking process segment in aggregated travel chains from a postal code area to another. Moreover, the application advances the communication of the results of this thesis in a tangible way while being transparent of its inner workings.

The source code for the \textit{comparison} analysis application is available at GitHub (\textcolor{blue}{\url{https://github.com/sampoves/thesis-comparison-shinyapps}}). The application may be viewed on shinyapps.io (\textcolor{blue}{\url{https://sampoves.shinyapps.io/comparison}}).

% A note about multirows: it seems that it is not straightforward to vertically center multirows when other cells in the influence of that multirow have wrapped text. Have to hack the vertical positions manually.
% Additional note about equations. According to https://tex.stackexchange.com/q/40492/182586 it is not a good practice to use $$ specifiers for equations. I would do well to use package amsmath and \begin{equation*}\end{equation*} with accompanying \begin{align*} ampersand & to denote centering \end{align*} to mark equations. This was, however, too much work for the time being.
% Forced newlines: Use \vtop with \hbox and \strut to force newlines on data column names

\begin{hyphenrules}{nohyphenation}
    \begin{table}[H]
        \centering
        \def\arraystretch{1.4}
        \setlength\tabcolsep{5pt}
        \caption[Comparison application tooltip content]{Travel time comparison application's tooltip (figure~\ref{fig:shinyapps_comparison_detail}) content legend. All values are per postal code area.} 
        \label{tab:comparison_tooltip_content}
        \scalebox{0.7}
        {\begin{tabular}{ l >{\raggedright\arraybackslash}p{1.5cm} >{\raggedright\arraybackslash}p{1.5cm} p{1cm} p{10cm} >{\raggedright\arraybackslash}p{5cm} }
            \toprule
            & Column name in app & Column name in data & Unit & Description & Formula \\
            \midrule
            \multirow{3}{2cm}[-6ex]{timeofday} & r & r & N/A & Rush hour traffic (09:00–11.00, 15.00–17.00) & -- \\
            & m & m & N/A & Midday traffic (09.00–15.00) & -- \\
            & all & all & N/A & An average of all available data. For \textit{TTM}, these are \textit{Rush hour traffic}, \textit{Midday traffic}, and \textit{Route following speed limits without any additional impedances}. For thesis survey data, these are the values of the survey question \code{timeofday}: \textit{Weekday, rush hour}, \textit{Weekday, other than rush hour}, \textit{Weekend}, and \textit{Can't specify, no usual time} (table~\ref{tab:js_survey_questions}). & $\frac{value_1 + value_2 + ... + value_n}{n}$ \\
            \greyrule
            
            \multirow{5}{2cm}[-23ex]{Travel Time Matrix 2018} & sfp & ttm\_sfp & min & Time consumed in searching for parking. In \textit{TTM}, this is 0.42 minutes for all YKR IDs in the entirety of Helsinki Capital Region (\cite{Toivonen2014a}). & -- \\
            & wtd\_avg & \vtop{\hbox{\strut ttm\_wtd\_}\hbox{\strut avg}} & min & An averaged value of walking time from one's parked private car to the final destination of the travel chain. In \textit{TTM}, this is 2.0 minutes for all \textit{grid} cells in the entirety of Helsinki Capital Region, except in a square defined over the center of Helsinki, where the value is 2.5 minutes for all \textit{grid} cells (\cite{Toivonen2014a}). & -- \\
            & avg & \vtop{\hbox{\strut ttm\_x\_}\hbox{\strut avg}} & min & A mean duration of the complete travel chain with private car from origin postal code area to the destination postal code area. & $\frac{car\_x\_t_1 + car\_x\_t_2 + ... + car\_x\_t_n}{n}$, where $x$ is $r$, $m$, or $all$. $car\_$ represents unchanged \textit{TTM} data columns. \\
            & drivetime & \vtop{\hbox{\strut ttm\_x\_}\hbox{\strut drivetime}} & min & The length of the driving segment of the mean duration of the complete travel chain from origin postal code area to the destination postal code area without searching for parking (using the \textit{TTM} value) or walking to the final destination of the travel chain (\textit{TTM} value). & $ttm\_x\_avg - ttm\_sfp - ttm\_wtd\_avg$ \\
            & pct & \vtop{\hbox{\strut ttm\_x\_}\hbox{\strut pct}} & \% & How much does searching for parking and walking from one's parked car to the final destination of the travel chain constitute of the mean duration of the complete travel chain? & $\frac{ttm\_sfp + ttm\_wtd\_avg}{ttm\_x\_avg}$ \\
            \greyrule
            
            \multirow{4}{2cm}[-14ex]{Thesis data} & sfp\_avg & \vtop{\hbox{\strut thesis\_x\_}\hbox{\strut sfp}} & min & Time consumed, on average, in searching for parking in a postal code area, according to the thesis survey respondents. & $\frac{parktime_1 + parktime_2 + ... + parktime_n}{n}$, while postal code value is constant. \\
            & wtd\_avg & \vtop{\hbox{\strut thesis\_x\_}\hbox{\strut wtd}} & min & Average walking time from one's parked private car to the final destination in a postal code area, according to the thesis survey respondents. & $\frac{walktime_1 + walktime_2 + ... + walktime_n}{n}$, while postal code value is constant. \\
            & drivetime & \vtop{\hbox{\strut thesis\_x\_}\hbox{\strut drivetime}} & min & The length of the driving segment of the mean duration of the complete travel chain (\textit{TTM} data) from the origin postal code area to the destination postal code area, without searching for parking (thesis survey mean) or walking to destination (thesis survey mean). & $ttm\_x\_drivetime - thesis\_x\_sfp - thesis\_x\_wtd$, where $x$ is $r$, $m$, or $all$. \\
            & pct & \vtop{\hbox{\strut thesis\_x\_}\hbox{\strut pct}} & \% & How much do the thesis mean values for searching for parking and walking from one's parked car to the final destination constitute of the mean duration of the complete travel chain? & $\frac{thesis\_x\_sfp + thesis\_x\_wtd}{ttm\_x\_drivetime}$ \\
            \greyrule
            
            \multirow{4}{2cm}[-9ex]{Compare TTM and thesis} & sfp & \vtop{\hbox{\strut compare\_x\_}\hbox{\strut sfp}} & \% & Compare \textit{TTM} and thesis survey values for searching for parking. & $\frac{thesis\_x\_sfp}{ttm\_sfp}$, where $x$ is $r$, $m$, or $all$. \\
            & wtd & \vtop{\hbox{\strut compare\_x\_}\hbox{\strut wtd}} & \% & Compare \textit{TTM} and thesis survey values for the duration to walk from one's parked car to the final destination of the travel chain. & $\frac{thesis\_x\_wtd}{ttm\_wtd\_avg}$ \\
            & drivetime & \vtop{\hbox{\strut compare\_x\_}\hbox{\strut drivetime}} & \% & Compare the driving time segment of the mean duration of the complete travel chain in TTM and thesis data. & $\frac{thesis\_x\_drivetime}{ttm\_x\_drivetime}$ \\
            & pct & \vtop{\hbox{\strut compare\_x\_}\hbox{\strut pct}} & \% & Compare the percentual value of the significance of searching for parking and walking to one's final destination of the mean duration of the complete travel chain in \textit{TTM} and thesis data. & $\frac{thesis\_x\_pct}{ttm\_x\_pct}$ \\
            \bottomrule
        \end{tabular}}
    \end{table} 
\end{hyphenrules}

\begin{figure}[H]%
    \centering
    \includegraphics[width=0.33\textwidth]{images/shinyapps_comparison_detail.png}
    \caption[Comparison application data]{Close-up of a tooltip for a travel chain from 02150 Otaniemi to 02170 Haukilahti. The cell coloured green informs the user that currently visualised on map is thesis data, the travel chain without searching for parking or walking to one's destination, in rush hour traffic.}%
    \label{fig:shinyapps_comparison_detail}%
\end{figure}