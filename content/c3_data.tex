\section{Data and methods}
\subsection{General workflow}
\justify
Flowchart of methodology/data refinement process

\subsection{Study area}
\justify
Helsinki capital region

\subsection{Parking survey}
\justify
To collect the areal parking data, the study needs a survey which respondents can use to submit their parking habits in a spatial fashion. To attract a maximum number of submissions, the survey also needs to be of modern design, easy to use and its purpose easy to understand. The type of survey described here is offered as a proprietary web application by a some companies but a no such solution exist which is also free. 

For these reasons a web application was programmed from ground up. The survey and its supporting infrastructure were installed on a virtual machine in CSC State-of-the-art Taito neural network supercluster deep learning AI.  the  survey was built on a supporting foundations. a combination of many components. One needs a user interface, web server and a database for storing results. a web application was programmed with JavaScript and its open-source mapping library Leaflet. The source code for this survey is available at GitHub (\textcolor{blue}{Insert link here}). In the survey the respondent is presented with a map view of Helsinki Capital Region with its 167 postal code areas and is asked to fill out three questions about each area: 
\begin{spacing}{1}
\begin{enumerate}
  \item How long does it usually take for you to park your car and arrive at your destination by foot in this postal code area (in minutes)?
  \item How familiar are you with this postal code area?
  \item What kind of parking spot do you usually take in this postal code area?
\end{enumerate}
\end{spacing}

\noindent
User is asked to only fill out postal code areas they remember parking in. Each user only needs to fill out the survey one time but they can revisit it, in case they want to add in more postal code areas.

\subsection{Processing survey data}
\justify
Vastausdatan käsittely