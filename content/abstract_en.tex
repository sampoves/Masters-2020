\justify
\textit{Accessibility} – what can be reached from a given point in space and how – is an essential field of study to measure the physical structure of cities, travel mode choices of residents, and the competitiveness of areas. Researchers increasingly acknowledge that accessibility is a fundamental concept on understanding how urban regions work and its position in future development of cities is paramount. Travel time is considered an intuitive measure to indicate accessibility and a strong predictor of mode choice, and usually, private car is the fastest mode of transport in urban environments. 

A central issue which stems from private cars and accessibility is the process of \textit{searching for parking}. An understudied issue, the rather stressful activity is engaged in when arriving by car at the general area of desired parking, but no space is available. Motorists are then forced to continue search for parking, significantly contributing to urban congestion. In catering to mobility rather than accessibility, the modern urban planning has made it challenging to move away from private cars toward alternative, often more sustainable, modes of transport. Travel time studies, and more specifically, parking studies, can produce accurate data to aid in this transformation.  

In this thesis, a parking related research survey was developed and conducted in the Helsinki Capital Region, Finland. Adhering to the \textit{door-to-door approach}, the survey respondents were enquired how long it took for them to find a parking place and park their car, and walk from the car to the destination in different postal code areas of Helsinki Capital Region. To explain a hypothetical variation in \textit{parking process} durations (searching for parking, and walking to one's destination) in different areas, additional questions, such as the time of the day of parking, were presented. The invitation to respond to the survey was mostly spread on the social media platform Facebook. The survey, filled out with a web application specifically programmed for this thesis, received 5200 data rows from over 1000 unique visitors.

The survey results indicate that there are spatial differences in parking process durations in different postal code areas of the Helsinki Capital Region. The inner city of Helsinki was experienced as the most difficult location to park in with regional subcenters such as Matinkylä, Espoo and Tikkurila, Vantaa, receiving relatively long parking process durations. Short parking process durations were reported from scarcely built areas but more often than not these areas had extreme values reported. Interestingly, area familiarity did not necessarily translate to faster parking process, while the type of the usual parking place was a better indicator. Out of the spatial explanatory variables added in the survey data processing, zones of urban structure (yhdyskuntarakenteen vyöhykkeet) could be used to find statistically significant differences in the parking process between variable groups and study area municipalities.

Making use of the Helsinki Region Travel Time Matrix, a dataset developed by the research group Digital Geography Lab of the University of Helsinki, the thesis survey data was compared to total travel chain durations. The thesis survey data indicates that the proportion of time it takes to park one's car and walk to one's destination is a much larger part of the entire travel chain than previously estimated in the dataset. The parking process times are proportionally largest in the inner city of Helsinki, where the reported parking process duration exceeds that of the actual driving segment.

This thesis, its entire version history, and all of the scripts developed for it have been made available at GitHub (\textcolor{blue}{\url{https://github.com/sampoves/thesis-data-analysis}}).