\section{Discussion}
\subsection{Measuring survey success}
\justify

\textcolor{red}{Research questions as chapter backbone} \\
\textcolor{red}{Discuss hypotheses stated in introduction, keep intro in mind at all times}\\
\textcolor{red}{selitä auki parktime ja walktime kerran ennen lyhenteiden käyttöä}\\
This study shows that there are, in fact, spatial differences in the time that it takes to park one's car in the Helsinki Capital Region. This was the hypothesis going in to the research, as it seemed to be a clear oversimplification on the part of Helsinki Region Travel Time Matrix to assume that one static value would represent the parking process length in the functionally diverse urban region.

\textcolor{red}{Some words about thesis result findings (expected likert to be better descriptions, disparity between mean and median values in explanatory variables [skewness of gathered data], finding parking spot has prominent luck factor based on weak significance of likert variable)}

The survey successfully withstood accidential and intentional misuse. Using safeguards such as serverside filtering of incoming data prevented possible attempts at mischief. Later, the analysis Python script made sure the amount of questionable responses were at a minimum. The survey program front-end made a great effort to appear easy to use to prevent frustration and premature exiting from the site.

\textcolor{red}{Discuss feedback I received during the process of gathering survey data (pitää kaivaa fabosta) In essence, how well did this survey succeed in collection of spatial data?}

\newpage
\subsection{Survey and analysis uncertainties}
\justify

A major source of uncertainty in this study stems from the design of the survey research. The phrasing of questions, available answering options to questions, and the technical design are of major relevance (\textcolor{red}{cite?}). The research survey application contained an extensive help functionality to instruct respondents how to respond with valid answers. When it was decided that the survey should not inquire specific parking events, a factor of uncertainty was accepted in the sense that as parking habits were disconnected from time and place, respondents were given leeway in question interpretation. Despite efforts to communicate essential information to respondents, I can not be sure that everyone understood what it meant when respondents were asked to report about their parking experiences \textit{usually}. Respondents were supposed to only think back two years to weed out estimations about parking infrastructure that not necessarily exists anymore. In addition, as the survey was provided in Finnish and English, there is a possibility for differently understood questions based on the language used.

A manner of uncertainty stems from the possibility to answer to \code{parktime} and \code{walktime} questions with an inplausibly large values, such as with the maximum of 99 minutes, as in, it took the respondent 99 minutes to walk from their car to the final destination of their travel chain. In this thesis the issue was solved by simply excluding all \code{parktime} and \code{walktime} values over 59 minutes, an entirely arbitrarily chosen value. Although these high values are improbable and their exclusion relatively inconsequential, some respondents may have reported real parking events which have now been left unexamined. Through feedback received in Facebook, I am able to deduce that some of the large values, especially the ones reporting 99 for both \code{parktime} and \code{walktime}, may have been made as protest to signify that parking is not possible or is highly unpleasant in the postal code areas concerned.

Regarding the single-answer questions in the survey, such as \textit{What kind of parking spot do you usually take in this postal code area?}, several of the available choices were problematic. Especially problematic is the choice \textit{Private or reserved}, which was meant to describe a parking space that is located, for example, in the parking lot owned by one's employer where one has a right to park but no specific reserved spot. Without consulting the survey help, it is entirely possible to believe this choice can be used for reporting home yard parking, an unwanted result. The choice \textit{Other} could, too, potentially be used to report parking at home. If these problematic \code{parkspot} answer scenarios happened, they are of relatively minimal consequence as 284 responses reported \textit{Private or reserved}, and only 39 responses reported \textit{Other}.

Even though the survey application provided tools to locate places and addresses accurately, it can't be ruled out that some respondents have sent answers about a postal code area when they meant to sent those answers about a neighboring postal code. The open numeric fields for \code{parktime} and \code{walktime} introduces a source of distortion in that people seemed to prefer reporting easy, common durations of time, such as five minutes, ten minutes, or twenty minutes. This trend can be viewed in the analysis application histograms and in the figure~\ref{fig:parktime_hist}. A behaviour pattern such as this is to be expected when people are required to make estimations of past durations of time and then the end results, the variables \code{parktime} and \code{walktime} have to be treated in same fashion. One additional matter to consider is the fallibility of respondents' memory. To what extent we can assume the survey results are from the correct time period, or location?

\textcolor{red}{consequences of my own assumptions such as classification of postal code areas into one YKR zone and artificial surface value}

\textcolor{red}{The significance of construction sites during these years to the responses?}

The survey data processing script written in Python has a feature to detect duplicate responses in the survey results. Using this tool, it is possible to conclude that 19 IP addresses sent responses with the same postal code areas. It can be discussed what has happened in these cases but no definite answer can be drawn. It is possible that people sharing a connection to the Internet may have legitimately sent their own responses, and indeed, in these cases the individual questions should have varying answers. The tool detects if the duplicate responses have identical answers to the individual questions. This could possibly indicate an attempt to influence the results of the survey. Mostly the duplicates have none or some identical answers, with some having the maximum of five identical answers. If the timestamps of these identical answers are close to each other, we can deduce that these are real duplicates. Four fully identical answers were found. In the case of one IP address, the postal code 00790 Viikki was entered a total of nine times. It is possible, as no identically answered questions were detected, that these responses originated from a public network such as a campus of the University of Helsinki. With no definite way to answer the questions regarding the duplicate responses, these answers were left in in the survey results. Furthermore, the 19 suspect IP address codes is ultimately a negligible fraction of the total visits.

\newpage
\subsection{Avenues for future research}
\justify

The research survey carried out for this thesis can be considered a success with the total of 5579 visits to the survey website and the 5183 received data rows. The collected datasets \textit{visitors} and \textit{records} are voluminous troves of data and this study observed them from a carefully defined angle leaving many other approach vectors unexplored. Combining the collected datasets with pre-existing spatial datasets and other sources of data could potentially lead to new findings about the parking habits of people that answered to the survey. In particular, the dataset \textit{visitors} was used in a superficial manner because of the arrangement of research questions and the survey's statement of privacy, which promised minimal use of personal data. In another context, the respondent behaviour could be used to find additional patterns in the \textit{records} dataset. Considering extended statistical analysis, it is possible to test a countless combination of explanatory variables against the response variables with equally many freely chosen spatial extents with exclusion of municipalities or municipality subdivisions. Much of this additional analysis can be carried out at any time with the analysis applications programmed for this thesis.

Aside from the basic identifying data, the PAAVO dataset includes dozens of columns of demographic data for each postal code area in Finland. The study did not make use of these components of the dataset. Employing these variables in parking research could bear deeper understanding about the parking behaviour in Helsinki Capital Region. One could, for example, attempt to find links between postal code areas of many work places and long parking times, or areas with high amount of buildings and long walking times.

Helsinki Region Infoshare (HRI) has released a spatial dataset which contains the locations of parking spots in Southern Helsinki, in a total of 13 postal code areas (\textcolor{red}{cite}). HRI states that the material contains errors and is not in actively supported. If this dataset encompassed a larger share of the Helsinki Capital Region and more work put into it, the parking process research could descend from the abstract level for a more grounded take on parking private cars in urban setting and the challenges it involves. The amount of parking spaces, even if it would only be an estimate, would shed an informative light into the possible connection of cruising for parking and parking places.

\newpage
\subsection{Conclusions}
\justify

\textcolor{red}{Conclude with private cars and urban environment talk. Realistic and optimistic thoughts.}

The source code repositories for \textit{records}, \textit{visitors}, \textit{comparison} analysis applications are available at GitHub (\textcolor{blue}{\url{https://github.com/sampoves/thesis-records-shinyapps}}, \textcolor{blue}{\url{https://github.com/sampoves/thesis-visitors-shinyapps}}, \textcolor{blue}{\url{https://github.com/sampoves/thesis-comparison-shinyapps}}). The applications may be viewed on shinyapps.io (\textcolor{blue}{\url{https://sampoves.shinyapps.io/records}}, \textcolor{blue}{\url{https://sampoves.shinyapps.io/visitors}}, \textcolor{blue}{\url{https://sampoves.shinyapps.io/comparison}}).