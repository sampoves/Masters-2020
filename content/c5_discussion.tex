\section{Discussion}
\justify

%\begin{itemize}
%  \item Discuss hypotheses stated in introduction here
%  \item In discussion have dialogue with introduction chapter
%  \item The kind of feedback I received in the process of gathering my survey data
%  \item HRI data about Helsinki center parking places, what would it mean if this data would encompass the entire HCR?
%  \item Conclusions
%  \item discussioniin: tulevaisuuden mahdollisuudet käyttää paavo-dataa (paljon kaikkea demografiatietoa) laajemmin keräämieni tutkimustulosten kanssa
%\end{itemize}

The survey data processing script written in Python has a feature to detect duplicate responses in the survey results. Using this tool, it is possible to conclude that 19 IP addresses sent responses with the same postal code areas. It can be discussed what has happened in these cases but no definite answer can be drawn. It is possible that people sharing a connection to the Internet may have legitimately sent their own responses, and indeed, in these cases the individual questions should have varying answers. The tool detects if the duplicate responses have identical answers to the individual questions. This could possibly indicate an attempt to influence the results of the survey. Mostly the duplicates have none or some identical answers, with some having the maximum of five identical answers. If the timestamps of these identical answers are close to each other, we can deduce that these are real duplicates. Four fully identical answers were found. In the case of one IP address, the postal code 00790 Viikki was entered a total of nine times. It is possible, as no identically answered questions were detected, that these responses originated from a public network. With no definite way to answer the questions regarding the duplicate responses, these answers were left in in the survey results. Furthermore, the 19 suspect IP addresses is ultimately a negligible fraction of the total visits. \textcolor{red}{liitteisiin tuplat esille?}

\begin{itemize}
    \item Expected \code{likert} to be a better descriptor
    \item disparity between mean and median in explanatory variables
    \item likert familiarity: finding parking spots has a prominent luck factor on likert results?
\end{itemize}

\textcolor{red}{Add separate links to all three shinyapps}