\section{Discussion}
\subsection{Measuring survey success}
\justify

%Research questions as chapter backbone
%Discuss hypotheses stated in introduction, keep intro in mind at all times
%Some words about thesis result findings (expected likert to be better descriptions, disparity between mean and median values in explanatory variables [skewness of gathered data], finding parking spot has prominent luck factor based on weak significance of likert variable)
%selitä auki parktime ja walktime kerran ennen lyhenteiden käyttöä
This study shows that there are, in fact, spatial differences in the time that it takes to park one's car in the Helsinki Capital Region. This was the hypothesis going in to the research, as it seemed to be a clear oversimplification on the part of Helsinki Region Travel Time Matrix to assume that one static value would represent the parking process length in the functionally diverse urban capital region. The thesis survey was designed with a good sense of what is essential in urban parking, with much of the collected data providing statistically significant explanation why are there spatial differences between areas. An additional hypothesis, in which the parking process was predicted to be significant, was confirmed. The proportion of the parking process is significant all over Helsinki Capital Region.

Long searching for parking were found especially in the inner city of Helsinki, in Matinkylä, Espoo, and in Tikkurila and Veromiehenkylä, Vantaa. Long walking times were spread around Helsinki Capital Region more evenly than parking times. In particular, inner Helsinki was again found to have long walking times, but so were Kulosaari, Malmi and Kaitalahti. In Espoo, the long walking times concentrated in Tapiola and maybe surprisingly, in Nupuri-Nuuksio and in Kauklahti. In Vantaa, a wedge-like area from Tikkurila to Seutula formed the areas with long walking times in Vantaa. It is useful to remember, however, that in general postal code areas in Espoo and Vantaa have significantly smaller amount of responses compared to Helsinki. As many postal code areas inside the inner Helsinki received hundred or more responses, Nupuri-Nuuksio and Kauklahti received 11 and 10 responses, respectively. Before the survey data collection, I expected a more spatially even distribution of responses. It now seems either most of the private car traffic heads for the services and locations in Helsinki, with much less activity in other municipalities, or, most of the respondents live in Helsinki and mostly only drive inside their home municipality. The fact that majority of the publicity work for the survey was done in Facebook groups based in Helsinki supports the latter assumption.

Parking times and walking times vary substantially between postal code areas and this study determined that the variation can be explained with location inside the cities (explanatory variable \code{subdiv}), at what time of the day one attempted to find a parking spot (\code{timeofday}), and what type of parking spot was used (\code{parkspot}). Perhaps the strongest explanation to spatial differences could be found from zones of urban structure (\code{ykr\_zone}). 

This study also shows that the significance of the parking process in complete travel chains was substantial. In most postal code areas close to the origin postal code area the parking process proportion is effortlessly over 20 percent of the total travel chain duration, and 50 percent is certainly not unheard of. The situation where the parking process is longer than the driving time occasionally makes an appearance. 
\textcolor{red}{Some words about thesis result findings}

The survey successfully withstood accidential and intentional misuse. Using safeguards such as serverside filtering of incoming data prevented possible attempts at mischief. Later, the analysis Python script made sure the amount of questionable responses were at a minimum. The survey program front-end made a great effort to appear easy to use to prevent frustration and premature exiting from the site.

During the data collection phase, I received over 50 comments regarding the thesis reseach survey on the social media platform Facebook. The overall helpfulness of the comments was often poor, but a few highlights helped shine light on the shortcomings of the survey. Some comments verified my suspicions about potential problems I had in my mind already when programming the survey.

About a fifth of the comments contained a positive message. Some people found the survey well designed and others thought the research subject was interesting and topical. Some more than fifth of the comments were critical of the survey. A few commenters stated that they felt intimidated by the breadth of the task asked of them. Suggestions were offered to make the survey seem less like a chore: inquire only a few postal code areas per person, or shorten the two year timeframe respondents were supposed to think back on. A handful of commenters felt that they were not given the tools to convey the experience that some areas in Helsinki Capital Area are sometimes in dire shortage of parking places, but othertimes the same areas are easy to park in. Areas with large event venues were brought up. Two commenters talked about parking events that do not happen because of perceivedly bad parking opportunities or failed parking events. These circumstances could function as foundations for later research. Some commenters had problems comprehending the user interface or encountered technical problems. These cases were difficult to deal with as there was no way of knowing if respondents had really read the survey instructions or if the situation was a genuine issue in the programming of the survey. As a matter of fact, I received a comment which declared the survey non-relevant, as they had misunderstood that the survey is only available in English. Finally, one person conveyed the dissappointment of the unavailability of Swedish language in the survey.

In a particularly useful comment, a person noted that it was not possible to create records for Vantaa's Nikinmäki. This prominent issue had stayed hidden from me until the survey was in production and Nikinmäki's exclusion may have costed me some data about Vantaa. In an other comment a person brought forward their confusion about the mainline instruction that parking made in personally reserved parking places should not be reported in the survey, but at the same time one of the parking place types in the variable \code{parkspot} was \textit{Private or reserved parking spot}. This comment made me aware of this wording issue and increased my alertness for other issues such as this.

More than a fifth of the comments contained a message of dissatisfaction about the current state of private car parking in Helsinki Capital Region. In some cases, the invitation to participate in the research was met with blunt retorts without any connections to the survey. A person said that they could not receive visitors because of the situation with the parking spaces. Another said that parking in the center is starting to resemble an utopian dream, while a third person disclosed that they had abandoned all ideas of parking in the center since 2013. In these messages the expensiveness and difficulty of parking in the center of Helsinki was lamented. In this research the center was indeed found to contain some of the longest times to search for parking but it was seemingly forgotten by these persons that the survey was interested in a vast expanse of areas beside the center of Helsinki. These comments of vexation lead me to believe that at least some of the anomalous \code{parktime} and \code{walktime} values that reached the maximum value 99 were made in protest.

The survey comments received on Facebook painted an interesting picture of at least a few types of potential survey respondents. Even though most people stayed quiet and participated, there are lessons to be learned in these voiced opinions. For example, it became ever clearer that the user experience has to be a top priority. If the survey landing view looks intimidating, respondents may be lost. If there is too much to read in the beginning, respondents are probably lost. Finally, the potential respondents have zero patience for failing programming. As can be viewed from the \textit{visitors} dataset, a majority of survey visitors stopped by for only a single time. Valuable data is lost if survey administrator can't respond to these demands. I consider the thesis survey a success as it gathered a considerable amount of comments and likes in the Facebook groups, the main channel of promotion, and gathered a large amount of valid data. Even if only one second was spent with each data row, the combined time it would take to fill out the survey 5183 times totals in nearly a hundred hours.

\newpage
\subsection{Survey and analysis uncertainties}
\justify

A major source of uncertainty in this study stems from the design of the survey research. The phrasing of questions, available answering options to the questions, and the technical design are of major relevance (\textcolor{red}{löytyiskö cite?}). The research survey application contained an extensive help functionality to instruct respondents how to respond with valid answers. When it was decided that the survey should not inquire specific parking events, but a general experience in a predetermined timeframe, a factor of uncertainty was accepted in the sense that as parking habits were disconnected from specific time and place, respondents were given leeway in question interpretation. Despite efforts to communicate essential information to respondents, I can not be sure that everyone understood what it meant when respondents were asked to report about their parking experiences \textit{usually}. Respondents were supposed to only think back two years to weed out estimations about parking infrastructure that potentially no longer exists or has changed considerably. A question with no definitive answer can be asked about the significance of rapid construction in different parts of Helsinki Capital Region. How much weight the large shifting construction sites of 00220 Jätkäsaari and 01700 Kivistö bear in the minds of the respondents? In addition, as the survey was provided in both Finnish and English languages, there is a possibility for differently understood questions based on the language used.

Regarding PAAVO dataset, a source of additional uncertainty rises from the fact that postal code areas were used as the main spatial unit. The benefit was streamlined survey design and lower possibility for data collection failure as the study would have needed a very large amount of responses to be valid. However, using the postal code areas obscures the actual hotspots of parking activity in the Helsinki Capital Region. We are now unable to see if responses have spread out to whole postal code areas or concentrated on a few popular activity locations. Using postal code areas also forces this study to use averages when comparing thesis survey data and Helsinki Region Travel Time Matrix 2018 data, making the southern islands of Suvisaaristo "as close to" Espoonlahti shopping mall as northern part of Suvisaaristo, located on the mainland, is.

A manner of uncertainty stems from the possibility to answer to \code{parktime} and \code{walktime} questions with an inplausibly large values, such as with the maximum of 99 minutes, i.e., it took the respondent 99 minutes to walk from their car to the final destination of their travel chain. In this thesis the issue was solved by simply excluding all \code{parktime} and \code{walktime} values over 59 minutes, an entirely arbitrarily chosen value. Although these high values are improbable and their exclusion relatively inconsequential, some respondents may have reported real parking experiences which have now been left unexamined. Through feedback received in Facebook, I am able to deduce that some of the large values, especially the ones reporting 99 for both \code{parktime} and \code{walktime}, may have been made as protest to signify that parking is not possible or is highly unpleasant in the postal code areas concerned.

Regarding the single-answer questions in the survey, such as \textit{What kind of parking spot do you usually take in this postal code area?}, several of the available choices were problematic. Especially problematic is the choice \textit{Private or reserved}, which was meant to describe a parking space that is located, for example, in the parking lot owned by one's employer where one has a right to park but no specific reserved spot. Without browsing the survey help, it is entirely possible to believe this choice can be used for reporting home yard parking, an unwanted result. The choice \textit{Other} could, too, potentially be used to report parking at home. If these problematic \code{parkspot} answer scenarios happened, they are of relatively minimal consequence as 284 responses reported \textit{Private or reserved}, and only 39 responses reported \textit{Other}.

The thesis survey experienced an issue originating from PAAVO dataset that was only revealed after the survey had gone public. The postal code area boundaries do not completely follow the boundaries of municipalities. Nevertheless, PAAVO dataset considers each postal code area to belong in a single municipality, leading to the situation where the neighborhood of Nikinmäki, Vantaa was completely excluded from the research (missing Nikinmäki in figure~\ref{fig:thesis_resarea}, northeastern Vantaa). In surface area more than a half of the postal code area 01490 Nikinmäki is indeed located in the city of Sipoo, but the majority of urban activity in that postal code area is located in Vantaa. Regardless, this research lost some responses because of this oversight.

Even though the survey application provided tools to locate places and addresses accurately, it can't be ruled out that some respondents could have sent answers about a postal code area when they meant to sent those answers about a neighboring postal code. Furthermore, the open numeric fields for \code{parktime} and \code{walktime} introduces a source of distortion in that people seemed to prefer reporting easy, common amounts of time, such as five minutes, ten minutes, or twenty minutes. This trend can be viewed in the analysis application histograms and in the figures~\ref{fig:parktime_hist} and \ref{fig:walktime_hist}. A behaviour pattern such as this is to be expected when people are required to make estimations of past durations of time and then the variables \code{parktime} and \code{walktime} as well as the end results in the comparison application have to be treated in same fashion. One additional matter to consider is the fallibility of respondents' memory. To what extent we can assume the survey results are from the correct time period, or location?

% likert-inspistä haettu: https://www.extension.iastate.edu/Documents/ANR/LikertScaleExamplesforSurveys.pdf
The calculation methods for some of the variables in the survey analysis application is an additional matter for consideration. In the case of explanatory variables \code{artificial} and \code{ykr\_zone}, a rather arbitrary approach was used for the originally continuous data. The variable \code{artificial}, was simplified from a percentage value 0--100 \% to five Jenks natural breaks classes. A source of uncertainty in \code{artificial} is the naming of these classes, which was based on the best practices by \textcolor{red}{cite, tsek kommentti}, but ultimately, the naming was subjectively decided. Additionally, it could be considered an oversimplification to say that, for example, 40 percent of artificial surface in a postal code area means that it can be characterised as \textit{Some built}. The variable \code{ykr\_zone} was subjected to arguably more drastic simplification. In the original \textit{YKR zones} data, each postal code area could contain as much as six different classes of urban structure and the leftover classes, which I have designated \textit{novalue} in this thesis. It, too, can be considered an oversimplification to choose only a singular largest zone of urban structure in each postal code area, causing some structurally diverse postal code areas, like 02270 Espoon keskus, to appear as \textit{novalue}, even when the area consists of only 28 percent of \textit{novalue} and contains four actual classes of \textit{YKR zones} data. For a future reference in similar contexts, it could be an informed decision to set specific conditions to minimise the presence of \textit{novalue}.

The survey data processing script written in Python has a feature to detect duplicate responses in the survey results. Using this tool, it is possible to conclude that 19 IP addresses sent responses that contained responses from same postal code areas. It can be discussed what has happened in these cases but no definite answer can be drawn and these cases can not be labeled as unwanted duplicates. It is possible that people sharing an internet connection may have legitimately sent their own responses, and indeed, in these cases the individual questions should have varying answers. The tool detects if the suspect IP addresses have identical answers in the individual questions. This could possibly indicate an attempt to influence the results of the survey. Mostly the suspect IP addresses have none or some identical answers, with a few IP addresses having the maximum of five identical answers. If the timestamps of these identical answers are close to each other, we can deduce that these are real duplicates. Identical responses were sent from four IP addresses and these identical responses were indeed sent only minutes apart. In the case of one separate IP address, the postal code 00790 Viikki was entered a total of nine times. It is possible, as no identically answered questions were detected in this case, that these responses originated from a public network such as a campus of the University of Helsinki. With no definite way to answer the questions regarding the duplicate responses, these suspected responses were included in the survey results. Furthermore, the 19 suspect IP address codes is ultimately a negligible fraction of the total visit count.

\newpage
\subsection{Avenues for future research}
\justify

The research survey carried out for this thesis can be considered a success with the total of 5579 visits to the survey website and the 5183 received data rows. The collected datasets \textit{visitors} and \textit{records} are voluminous troves of data and this study observed them from a carefully defined angle leaving many other approach vectors unexplored. Combining the collected datasets with pre-existing spatial datasets and other sources of data could potentially lead to new findings about the parking habits of people that answered to the survey. In particular, the dataset \textit{visitors} was used in a superficial manner because of the arrangement of research questions and the survey's statement of privacy, which promised minimal use of personal data. In another context, the respondent behaviour could be used to find additional patterns in the \textit{records} dataset. Considering extended statistical analysis, it is possible to test a countless combination of explanatory variables against the response variables with equally many freely chosen spatial extents with exclusion of municipalities or municipality subdivisions. Much of this additional analysis can be carried out at any time with the analysis applications programmed for this thesis.

Aside from the basic identifying data, the PAAVO dataset includes dozens of columns of demographic data for each postal code area in Finland. The study did not make use of these components of the dataset. Employing these variables in parking research could bear deeper understanding about the parking behaviour in Helsinki Capital Region. One could, for example, attempt to find links between postal code areas of many work places and long parking times, or areas with high amount of buildings and long walking times.

Helsinki Region Infoshare (HRI) has released a spatial dataset which contains the locations of parking spots in Southern Helsinki, in a total of 13 postal code areas (\textcolor{red}{cite}). HRI states that the material contains errors and is not in actively supported. If this dataset encompassed a larger share of the Helsinki Capital Region and more work put into it, the parking process research could descend from the abstract level for a more grounded take on parking private cars in urban setting and the challenges it involves. The amount of parking spaces, even if it would only be an estimate, would shed an informative light into the possible connection of cruising for parking and parking places.

\newpage
\subsection{Conclusions}
\justify

%rakentaminen, palveluiden laajeneminen, arvojen muutos, nuorten käytös, vanhojen käytös

\textcolor{red}{Conclude with private cars and urban environment talk. Realistic and optimistic thoughts.}

The source code repositories for \textit{records}, \textit{visitors}, \textit{comparison} analysis applications are available at GitHub (\textcolor{blue}{\url{https://github.com/sampoves/thesis-records-shinyapps}}, \textcolor{blue}{\url{https://github.com/sampoves/thesis-visitors-shinyapps}}, \textcolor{blue}{\url{https://github.com/sampoves/thesis-comparison-shinyapps}}). The applications may be viewed on shinyapps.io (\textcolor{blue}{\url{https://sampoves.shinyapps.io/records}}, \textcolor{blue}{\url{https://sampoves.shinyapps.io/visitors}}, \textcolor{blue}{\url{https://sampoves.shinyapps.io/comparison}}).