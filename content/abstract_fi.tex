\justify
\textit{Saavutettavuus} -- mitä tietystä paikasta voidaan saavuttaa ja miten -- on keskeinen käsite, kun tutkitaan kaupunkirakennetta, asukkaiden liikkumistapavalintoja sekä alueiden keskinäistä kilpailukykyä. Tutkijat ovat entistä enemmän yhtä mieltä siitä, että kaupunkien tulevaisuudentutkimuksessa tulee ottaa huomioon saavutettavuus. Matka-aikaa pidetään intuitiivisena tapana mitata saavutettavuutta ja muuttujana se ennustaa vahvasti liikkumistapavalintaa. Kaupunkiympäristöissä henkilöauto on usein nopein kulkumuoto.

\textit{Pysäköintipaikan etsiminen} on kaupunkien autoilijoille tuttu prosessi, jossa yhdistyy henkilöautojen ja saavutettavuuden yhteensovittamisen haastavuus. Autoilija päätyy pysäköintipaikan etsintään silloin, kun autolle ei ole saatavilla pysäköintipaikkaa sillä alueella, jonne olisi tahdottu pysäköidä. Autoilija joutuu kiertämään aluetta niin kauan, kuin pysäköintipaikka vapautuu. Tällainen liikenne on omiaan pahentamaan kaupunkien ruuhkia. Perinteisesti kaupunkisuunnittelua on toteutettu mobiliteetin (mitä voidaan saavuttaa tietyssä aikamääreessä) ehdoilla, kiinnittäen vähemmän huomiota saavutettavuuteen. Vuosikymmenten mobiliteettipainotetun kaupunkisuunnittelun jälkeen on haasteellista kehittää kaupunkia, jossa henkilöautoliikenteen osuus on pienempi. Henkilöautoliikenteen kasvun rajoittaminen vaihtoehtoisilla liikumistavoilla on kaupunkialueiden edun mukaista. Matka-aikatutkimukset ja tarkemmin, pysäköintiaikatutkimukset, ovat eräs keino edelläkuvatun muutoksen edesauttamiseksi.

Tässä pro gradu -tutkielmassa kehitettiin ja toteutettiin henkilöautojen pysäköintiä koskeva kyselytutkimus. Pääkaupunkiseudulle sijoittuneessa tutkimuksessa kysyttiin, kuinka kauan vastaajalla yleensä kestää pysäköidä autonsa ja kävellä autolta matkan lopulliseen määränpäähän seudun eri postinumeroalueilla (\textit{pysäköintiprosessi}). Jotta hypoteettinen pysäköintiprosessissa tapahtuva ajallinen vaihtelu voitaisiin selittää tutkimuksen analyysivaiheessa, kyselyssä esitettiin joitain lisäkysymyksiä, esimerkiksi minä vuorokaudenaikana vastaaja yleensä pysäköi alueelle. Kyselyä mainostettiin valtaosin sosiaalisessa mediassa, Facebookin kaupunginosaryhmissä. Kysely täytettiin tätä tutkimusta varten ohjelmoidussa verkkosovelluksessa. Kyselystä saatiin yli 5200 vastausta yli tuhannelta tutkimukseen osallistuneelta henkilöltä.

Kyselyn tutkimustulokset viittaavat siihen, että pysäköintiajoissa sekä kävelyajoissa autolta määränpäähän on eroavaisuuksia pääkaupunkiseudun postinumeroalueiden välillä. Pisimmät pysäköintiprosessit mitattiin Helsingin kantakaupungissa. Alueellisesti merkittävän pitkiä pysäköintiprosesseja löytyi muun muassa Matinkylästä Espoosta sekä Vantaan Tikkurilasta. Lyhyitä pysäköinti- ja kävelyaikoja mitattiin enimmäkseen pääkaupunkiseudun harvasti asutuilla postinumeroalueilla, mutta myös näillä alueilla esiintyi merkittäviä eroavaisuuksia pysäköintiaikojen pituuksissa. Huomionarvoista tuloksissa oli, että lähialuetuntemus ei nopeuttanut pysäköintitapahtumaa. Sen sijaan pysäköintipaikan tyyppi oli parempi indikaattori, tuottaen pisimmät pysäköintiprosessit kadunvarsipysäköinnissä ja lyhyimmät pysäköintihalleissa. Kyselyaineistoon lisättiin prosessointivaiheessa kaksi spatiaalista selittävää muuttujaa, keinotekoisen maanpeitteen prosenttiosuus sekä vallalla oleva yhdyskuntarakenteen vyöhyke. Yhdyskuntarakenteen vyöhyke tuotti tilastollisesti merkitseviä eroja tutkimusalueen kaupunkien sekä vyöhykkeiden välille.

Tässä tutkimuksessa hyödynnettiin Helsingin yliopiston Digital Geography Labin tuottamaa Helsingin alueen matka-aikamatriisia, josta laskettiin kokonaismatka-ajat postinumeroalueiden välille. Kyselyaineisto ja matka-aikamatriisi yhdistettiin pysäköintiprosessin osuuden selvittämiseksi kokonaismatka-ajasta. Tästä saatiin selville, että matka-aikamatriisin arvio pysäköintiprosessin pituudesta oli huomattavasti alhaisempi verrattuna pysäköintikyselyssä tuotettuun aineistoon. Pysäköintiprosessi oli koko matkaan suhteutettuna pisin Helsingin kantakaupungissa, missä pysäköintipaikan löytämiseen, pysäköintiin ja määränpäähän kulunut aika oli monin paikoin pidempi kuin matka-aikamatriisista laskettu kokonaismatka-aika.

Tämä pro gradu -tutkielma, sen koko versiohistoria ja sitä varten tehdyt skriptit ovat julkaistu GitHubissa (\textcolor{blue}{\url{https://github.com/sampoves/thesis-data-analysis}}).