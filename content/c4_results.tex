\section{Results}
\justify

\subsection{Survey results in summary}
\justify
%- Present results in the order of the research questions
%- Amount of records, amount of faulty records
%- Uncertainties
%- Present logically all results and interesting specific questions
%- Compare parking with original Travel Time Matrix values and values collected by me
%- Charts, statistics
%-Just show your findings, no nonsense
The survey received visits from 4309 unique IP addresses with a total of 5561 visits. 843 visitors visited the survey more than one time. This signals that a majority of survey respondents were finished with the survey in one sitting. 24.6 percent of all visitors submitted at least one answer. On average one respondent submitted answers for 4.89 postal code areas.

The survey received in total 5222 answers of which 39 were deemed invalid using the criteria described in \hyperref[sec:processdata]{Processing survey data}. The raw survey data included multiple answers where the values for searching for parking or walking from parking to destination, or both, were 60 minutes or more. It was impossible to determine the reasoning behind these answers and it was concluded that these answers would be excluded. The maximum values in these fields were consciously placed to 99 in an effort for the range to not feel restrictive for user. Additionally the raw data included some answers submitted by the author which were discarded. In one case an individual contacted the author to report an erroneous submission which was also detected and deleted. 

The survey data processing script written in Python has a feature to detect duplicate responses in the survey results. Using this tool it is possible to conclude that 19 IP addresses sent responses with the same postal code areas. It can be discussed what has happened in these cases but no definite answer can be drawn. It is possible that people sharing a connection to the Internet may have legitimately sent their own responses, and indeed, in these cases the individual questions should have varying answers. The tool detects if the duplicate responses have identical answers to the individual questions. This could possibly indicate an attempt to influence the results of the survey. Mostly the duplicates have none or some identical answers, with some having the maximum of five identical answers. If the timestamps of these identical answers are close to each other, we can deduce that these are real duplicates. Four fully identical answers were found. In the case of one IP address the postal code 00790 Viikki was entered a total of nine times. It is possible, as no identically answered questions were detected, that these responses originated from a public network. With no definite way to answer the questions regarding the duplicate responses, these answers were left in in the survey results. (\textcolor{red}{liitteisiin tuplat esille?})

In summary, 5183 valid answers were sent from 1060 unique IP addresses. All 167 postal code areas received answers, with the median amount being 17 answers per area. On average, one postal code area received 31.03 answers. Five areas received over one hundred answers while 60 postal code areas received less than ten answers.

\textcolor{red}{Insert taulukko or teemakartta about answers (convey spatial distribution)}

\subsection{Statistics}
\justify
Insert statistics things