\section{Introduction}
\justify
% --- what to include in intro ---
% Glimpse to essence of the current scientific discussion \par
% terminology \par
% viewpoint of previous studies \par
% gaps in current scientific knowledge \par
% how study relates to existing ones \par
% main themes and challenges dealt with the present study \par

% In the center of this thesis is the work of \citeauthor{Salonen2013} where they presented their door-to-door approach (\citeyear{Salonen2013}). 

Amidst the raised awareness of climate change, the number of private cars is globally on the rise. According to one estimation, the world reached one billion cars in 2020 (\cite{Sperling2009}). \textcolor{red}{OICA sanoo, että 1 mrd kaaraa 2015}. With no slowing down in sight for the production of private cars, eyes must turn to managing the vast quantity of personal transportation in cities and in their surroundings. It is a question of mitigation of climate change but also maximising the quality of life for urban citizens everywhere. (\cite{StatisticsFinland2019})

Cities face many challenges, particularly in relation to mobility of people and structure of land use. Parking management is a way to link land use and transportation (\cite{Marsden2006}; \cite{Diallo2015}).

Goals of parking policy are numerous, for example optimal accessibility and traffic flow and maximising turn-over for shops (\cite{Marsden2006}).

Travel time is considered an intuitive measure to indicate accessibility and a strong predictor of mode choice (\cite{Frank2008}).

Donald Shoup's Crusing for Parking and The High Cost of Free Parking, Beyond Mobility book?

To realistically model travel times, one needs to take into account the whole process of private car journeys, including the entire parking process (\cite{Salonen2013}).

This thesis is connected with the work of Digital Geography Lab (\textcolor{blue}{\url{https://www.helsinki.fi/en/researchgroups/digital-geography-lab}}) and Accessibility Research Group (\textcolor{blue}{\url{https://blogs.helsinki.fi/accessibility}}) within University of Helsinki.

This thesis strives for maximum transparency and repeatability. All parts of this thesis are available online at GitHub (\textcolor{blue}{\url{https://github.com/sampoves/Masters-2020}} and \textcolor{blue}{\url{https://github.com/sampoves/Msc_thesis_data_analysis}}). This includes the entire thesis in LaTeX format, the parking survey programmed in JavaScript, instructions to set up the web server as used in this research survey, and the survey data and analysis scripts programmed in Python and R. Complete development histories of all components are included. All of the work is released using \textcolor{red}{MIT or something} license. In addition to the work proper, as side products, an empty template of this thesis (\textcolor{blue}{insert link}) and a point based variant of the park survey (\textcolor{blue}{insert link}) have been made available.

I would like to thank Harri Lampi, Johannes Nyman, Samuli Pitzén and Panu Vesanen for their invaluable help in various aspects of this thesis. \textcolor{red}{mihin väliin kuuluu acknowledgements?}

\bigskip
\noindent
The research questions for this thesis are:

\begin{spacing}{1}
    % use Roman numerals, uses enumitem
    \begin{enumerate}[label=\Roman*]
      \item What are the spatial differences in the time that it takes to find a parking spot and park one’s car in the study area?
      \item If spatial differences are detected, what explains them?
      \item What is the significance of the parking process to the overall travel time?
    \end{enumerate}
\end{spacing}
\bigskip
In addition to these research questions, this thesis explores how well the map survey created for this thesis worked in collecting user data in a spatial manner.